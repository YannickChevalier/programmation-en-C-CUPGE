\titre{Fractions}


\section{Introduction}


Le but de ce thème est de traduire des réels de l’intervalle \([0,
1]\) en fractions à une précision donnée près. On va travailler sur 5
fichiers :
\begin{itemize}
\item le fichier \url{src/Fractions/fraction.mk} pour compiler automatiquement
  les fichiers ;
\item le fichier \url{src/Fractions/fraction.c} qui contiendra
  l’implémentation des fractions ;
\item le fichier \url{include/fraction.h} qui contiendra les en-têtes des
  fonctions et types définis ;
\item le fichier \url{src/Fractions/test_fraction.c} qui contiendra les
  tests sur les fractions ;
\item le fichier \url{src/Fractions/recherche.c} qui contiendra les sources
  du programme bin/fraction de recherche de fraction.
\end{itemize}


\section{Compilation}

Pour effectuer les tests, on va réutiliser les fichiers \texttt{tests.h} et la librairie \texttt{entrees}. Assurezvous qu’ils sont bien dans le répertoire \url{~/src/include}. Cela signifie que parmi les options de
gcc, lors de la compilation, il faut ajouter :
\begin{center}
  \tt -I "\url{~/programmation-en-C/src/include}"
\end{center}
De plus, pour voir plus facilement le résultat des tests, il faut compiler avec :
\begin{center}
  \tt -DUSECOLORS
\end{center}
Ensuite, quand on en sera à la version finale, on enlèvera les tests en compilant avec l’option :
\begin{center}
  \tt -Dproduction
\end{center}
Enfin, pour utiliser la fonction \(\text{\tt fabsl} : \text{\tt
  double} \rightarrow \text{\tt double}\) qui calcule la valeur
absolue d’un double et est définie dans la bibliothèque
\texttt{<math.h>}, il faut compiler avec l’option : 
\begin{center}
  \tt -lm
\end{center}
 On paramétrise
le fichier  \texttt{fraction.mk} avec des variables en le commençant par les lignes
suivantes : 
\begin{lstlisting}[language=make]
GCC=gcc 
LIBS=-lm -lentrees 
INCLUDE=-I "~/programmation-en-C/src/include" 
FLAGS=-DUSECOLORS 
\end{lstlisting}

On peut utiliser ces variables dans les règles. Par exemple :
\begin{lstlisting}[language=make]
test_fraction : test_fraction.c
       fraction.c $(GCC) $(FLAGS) -o $@ $@.c $(INCLUDE) $(LIBS) 
\end{lstlisting}


\paragraph{Test du Makefile.} On peut préciser ce qui va être fait en
marquant le but d’une règle. Par exemple :
\begin{lstlisting}[language=bash]
  make -f fraction.mk test_fraction
\end{lstlisting}

\section{Fichier \texttt{fraction.c}}

\exo{Base sur les structures}

\question Déclarer une structure \texttt{fraction\_s} qui contient
deux entiers, un numérateur \(p\) et un dénominateur \(q\)

\begin{solutioncachee}
  \begin{lstlisting}[language=C]
struct fraction_s {
  int p ;
  int  q ;
} ;
  \end{lstlisting}
\end{solutioncachee}

\question Déclarer un nouveau type fraction qui est le type des structures \texttt{fraction\_s}.
\begin{solutioncachee}
  \begin{lstlisting}[language=C]
typedef struct fraction_s fraction ;
\end{lstlisting}
\end{solutioncachee}

\question Écrire une fonction \texttt{nouvelle\_fraction} qui prend en
entrée deux entiers, et qui renvoie une fraction avec ces entiers
comme, respectivement, le numérateur et le dénominateur.  

\begin{solutioncachee}
  \begin{lstlisting}[language=C]
fraction nouvelle\_fraction ( int p , int q )
{
  fraction r ;
  r.p = p ;
  r.q = q ;
  return r ;
}
\end{lstlisting}
\end{solutioncachee}

\question Écrire deux fonctions \texttt{numerateur} et
\texttt{denominateur} qui renvoient respectivement le numerateur et le
dénominateur de la fraction passée en entrée.

\begin{solutioncachee}
  \begin{lstlisting}[language=C]
int
numerateur ( fraction f )
{
return f.p ;
}
int
denominateur ( fraction f )
{
return f.q ;
}
\end{lstlisting}
\end{solutioncachee}

\paragraph{Note de bonne programmation.} Dans la suite, utilisez
uniquement les fonctions \texttt{nouvelle\_fraction}, \texttt{numerateur}, et
\texttt{denominateur}.

\exo{Opérations simples sur les fractions}

\question Écrire une fonction \texttt{fractioncmp} qui prend en entrée
deux fraction \(f_1\) et \(f_2\) et rend un entier du même signe que
\(f_1 - f_2\) (pour l’opération habituelle de soustraction) et \(0\)
si les deux fractions sont égales.

\begin{solutioncachee}
  \begin{lstlisting}[language=C]
int fractioncmp ( fraction a , fraction b )
{
  return numerateur ( a ) ∗ denominateur ( b ) 
  - numerateur ( b ) ∗ denominateur ( a ) ;
}
\end{lstlisting}
\end{solutioncachee}


\question {\ 'E}crire une fonction \texttt{print\_fraction} qui affiche une
fraction.
\begin{solutioncachee}
  \begin{lstlisting}[language=C]
void print_fraction ( fraction f )
{
  printf ( "  %d\n----------\n   %d\n" , numerateur ( f ) , denominateur ( f ) ) ;
}
\end{lstlisting}
\end{solutioncachee}


\question Écrire une fonction \texttt{addition\_cancre} qui fait
l’addition des cancres de 2 fractions :
\[
  \frac pq \oplus \frac{p'}{q'} = \frac{p+p'}{q+q'}
\]

On note qu’il s’agit aussi de l’addition des professeurs, puisque
\[
  \frac 58 \oplus \frac{8}{12} = \frac{13}{20}
\]

\begin{solutioncachee}
  \begin{lstlisting}[language=C]
fraction
addition_cancre ( fraction f1 , fraction f2 )
{
return nouvelle_fraction ( numerateur ( f1 ) + numerateur ( f2 ) ,
denominateur ( f1 ) + denominateur ( f2 ) ) ;
}
\end{lstlisting}
\end{solutioncachee}

\question Écrire une fonction \texttt{eval\_fraction} qui renvoie le
nombre (de type double) qui est la valeur de la fraction passée en
argument.
\begin{solutioncachee}
  \begin{lstlisting}[language=C]
double eval_fraction ( fraction f ) 
{ 
  return ( (  double ) numerateur ( f ) ) / denominateur ( f ) ; 
}
\end{lstlisting}
\end{solutioncachee}

\exo{Recherche par dichotomie d’une fraction}

L’addition des cancres permet de trouver par dichotomie la forme réduite d’une fraction
approchant un nombre positif. La justification mathématique est donnée par le lemme suivant:
\begin{trivlist}
\item \textbf{Lemme 1.} \em
Si \(\frac ab < \frac pq < \frac cd\) et \(ad-bc = -1\) alors:
\[
  \left\lbrace
    \begin{array}{rcl}
      p&\ge&a+c\\
      q&\ge&b+d\\
    \end{array}
  \right.
\]
\end{trivlist}

Donc si on cherche un rationnel entre deux autres nombres qu’on aura calculé, soit il est
égal à leur somme des cancres, soit il est entre leur somme des cancres et l’un des deux, et ses
numérateurs et dénominateurs sont plus grand que ceux de la somme des cancres.


\question En déduire une fonction \texttt{ftofraction} qui prend en
entrée un double et rend une fraction irréductible égale à ce double à
une précision absolue \(\varepsilon\) près.  
\paragraph{Aide.} Il peut commencer par considérer uniquement
le cas d’un nombre entre \(0\) et \(1\)  et faire une recherche
dichotomique en partant de \(0 = \frac 01\) et \(1 = \frac 11\). On peut généraliser ensuite aux nombres positifs en posant \(+\infty=\frac 10\).  
\paragraph{Aide pour le C.} Pour
calculer la valeur absolue, utilisez la fonction \texttt{fabsl} définie dans
\texttt{math.h}.  

\begin{solutioncachee}
  \begin{lstlisting}[language=C]
fraction ftofraction ( double x , double epsilon ) 
{ 
  double e ;

  fraction min = nouvelle_fraction ( 0 , 1 ) , //zero
  max = nouvelle_fraction(1 , 0 ) ; //+infini
  fraction courant ;

  do {
    courant = addition_cancre ( min , max ) ;
    e = eval ( courant ) ;
    if ( e < x )
    min = courant ;
    else
    max = courant ;
  } while ( fabsl ( e − x ) > epsilon ) ;
  return courant ;
}
\end{lstlisting}
\end{solutioncachee}



\section{Écriture d’un programme}

On veut maintenant écrire un programme complet qui va effectuer la
mise sous forme de fraction d’un décimal. Lors de
l’appel---\textit{i.e.}, au début de la fonction main---le programme
commence par regarder combien il a d’arguments :
\begin{itemize}
\item s’il en a 0, il demande (en utilisant \texttt{LireDecimal}) à
  l’utilisateur un nombre à chercher et une précision ;
\item s’il en a 1, on suppose que c’est le nombre à chercher, et on
  utilise la précision par défaut de \(0.0000001\) ;
\item s’il en a 3, le premier doit être la chaîne de caractères
  \texttt{"-p"}, le second la précision, et le troisième est le nombre
  à chercher.

\exo{Utilisation des arguments}

Implémentez ce programme dans un fichier recherche.c.
\begin{solutioncachee}
  \begin{lstlisting}[language=C]
#include <stdio.h>
#include <stdlib.h>
#include <math.h>
#include "fraction.h"
#include "entrees.h"
void
demande_utilisateur ( double ∗ x , double ∗ epsilon )
{
  printf ( " Entrez un nombre à traduire en fraction:\n " ) ;
  lire_decimal_long ( x ) ;
  printf ( " Entrez ␣ l'erreur admissible epsilon :\n " ) ;
  do {
    lire_decimal_long ( epsilon ) ;
  }  while ( ∗ epsilon < 0 ) ;
}

int
main ( int argc , char ∗ argv [ ] )
{
  double x , epsilon ;
  switch ( argc )
  {
    case 1 :
    demande_utilisateur ( &x , &epsilon ) ;
    break ;
    case 2 :
    x = atof ( argv [ 1 ] ) ;
    epsilon = 0.0000001 ;
    break ;
    case 4 :
    i f ( strcmp ( argv [ 1 ] , "−p " ) == 0 )
    {
      x = atof ( argv [ 3 ] ) ;
      epsilon = atof ( argv [ 2 ] ) ;
    }
    else
    goto erreur_argument ;
    break ;
    default :
    goto erreur_argument ;
  }
  print_fraction ( ftofraction ( x , epsilon ) ) ;
  return 0 ;
  erreur_argument :
  printf ( " Utilisation: %s [[ --p <precision>] <x >]\n " , argv [ 0 ] ) ;
  return 1 ;
}

\end{lstlisting}
\end{solutioncachee}

\section{Programmation modulaire}

Le fichier fraction.h a un défaut, qui est minime pour nous, mais qui
est handicapant pour les gros programmes: les programmeurs qui
l’utilisent peuvent créer de nouvelles structures en dehors des
fonctions qu’on a défini car ils ont accès au format de la structure.

La technique permettant d’éviter cela consiste à déclarer le type
\texttt{fraction} non plus comme une structure fraction, mais comme un pointeur
vers une telle structure. Si les fichiers ont été écrits correctement,
ce changement est très rapide.

\exo{Changement de type des fractions}

\question Changez le type fraction en un type qui pointe sur des structures dans \texttt{fraction.h}
et \texttt{fraction.c}.

\begin{solutioncachee}
  \begin{lstlisting}[language=C]
typedef struct fraction_s * fraction ;
  \end{lstlisting}
\end{solutioncachee}

\question Changez les fonctions \texttt{nouvelle\_fraction},
\texttt{numerateur}, et \texttt{denominateur} du fichier
\texttt{fraction.c} pour les adapter au nouveau type.

\question Recompilez \texttt{test\_fraction} et \texttt{recherche}.

\question Enlever la déclaration de la structure \texttt{fraction\_s}
du fichier \texttt{fraction.h}. Recompilez \texttt{test\_fraction} et
\texttt{recherche}.

\begin{solutioncachee}
  \begin{lstlisting}[language=make]
GCC=gcc
LIBS=-lm ../lib/cours.o
INCLUDE=-I "../include/ "
FLAGS=-DUSECOLORS

fraction.o : fraction.c
    $(GCC) $(FLAGS) −c $< $(INCLUDE)

test_fraction : test_fraction.c fraction.c
    $(GCC) $(FLAGS) −o $@ $@. c $(INCLUDE) $(LIBS)

fraction : recherche.c fraction.o
    $(GCC) $(FLAGS) −o $@ recherche.c $(INCLUDE) $(LIBS) fraction.o

recherche : recherche.c fraction.o
    $(GCC) $(FLAGS) −o $@ recherche.c $(INCLUDE) $(LIBS) fraction.o
    
  \end{lstlisting}
\end{solutioncachee}
