\exo{Matrices}





Le but de cet exercice est d'impl{\'e}menter des fonctions op{\'e}rant sur des
matrices. Une \emph{matrice} est un tableau de \cfloat {\`a} deux
dimensions. 

\question (1 pt) D{\'e}clarez une structure \ctype{Matrice\_s} qui
contient:
\begin{itemize}
\item le nombre de lignes et de colonnes ;
\item un tableau de lignes, chaque ligne {\'e}tant un tableau de \cfloat ;
\end{itemize}
ainsi que le type \ctype{Matrice} des pointeurs vers ces
structures. 

Dans la suite, une matrice ayant $l$ lignes et $c$ colonnes est dite
$l\times c$. Une matrice $l\times c$ peut {\^e}tre {\'e}crite
\matrice alc ; dans ce cas on dit que $a_{i,j}$ est son coefficient
{\`a} la ligne $i$ et {\`a} la colonne $j$. Enfin, on dit qu'une matrice est
\emph{carr{\'e}e} si $l = c$.


\question (2 pts) {\'E}crivez une fonction \cfun{matrice} cr{\'e}ant une
nouvelle matrice ayant $l$ lignes et $c$ colonnes et dont tous les
coefficients sont {\`a} $0$.

\question (1 pt) L'addition de deux matrices $A=\matrice a{l_a}{c_a}$
et $B=\matrice a{l_b}{c_b}$ est d{\'e}finie si, et seulement si, $l_a=l_b$
et $c_a=c_b$. {\'E}crire une fonction indiquant si deux matrices peuvent
{\^e}tre additionn{\'e}es.

\question (2 pts) Si deux matrices \matrice alc et \matrice blc sont
additionnables, le r{\'e}sultat de leur addition est la matrice \matrice
rlc avec:
$$
r_{i,j}=a_{i,j}+b_{i,j}
$$
{\'E}crire une fonction rendant le r{\'e}sultat de l'addition de deux matrices.


\question (2 pts) Le produit de deux matrices \matrice alc et \matrice
b{l'}{c'} est d{\'e}fini lorsque $c=l'$, et dans ce cas, le r{\'e}sultat du
produit est la matrice $\matrice rl{c'}$ d{\'e}finie par:
$$
r_{i,j} = \Sigma_{k=1}^c a_{i,k} \cdot b_{k,j}
$$
{\'E}crire une fonction rendant le produit de deux matrices.

\question (2 pts) On donne ci-dessous un sch{\'e}ma pour effectuer un
calcul de puissance rapide d'une matrice carr{\'e}e $n\times n$  $A$. Dans
ce sch{\'e}ma, $I_n$ d{\'e}signe la matrice identit{\'e} $n\times n$ (ses
coefficients $r_{i,i}$ valent $1$, tous les autres valent $0$).

\begin{verbatim}
// Calcul de A^k pour une matrice n x n A
R:= I_n
x := A ;
Tant que k > 0 faire
   Si k mod 2 == 0 Alors
       k := k / 2 ; x := x * x
   Sinon
       k := k - 1 ; R := R * x ;
   Fin Si
Fin tant que
retourner R.
\end{verbatim}
{\'E}crire une fonction impl{\'e}mentant ce calcul rapide de puissance d'une
matrice carr{\'e}e.

\question (3 pts) Soient $A=\matrice alc$ et $B=\matrice bc{c'}$ deux
matrices telles que le produit $A\times B$ soit d{\'e}fini. On peut alors
d{\'e}couper $A$ et $B$ en blocs tels que toutes les op{\'e}rations ci-dessous
(additions et produits de matrices) soient d{\'e}finies:
$$
\left(
  \begin{array}[b]{c|c}
    A_1 & A_2\\
    \hline
    A_3 & A_4
  \end{array}
\right) \times
\left(
  \begin{array}[b]{c|c}
    B_1 & B_2\\
    \hline
    B_3 & B_4
  \end{array}
\right) =
\left(
  \begin{array}[b]{c|c}
    A_1\times B_1 + A_2\times B_3 & 
    A_1\times B_2 + A_2 \times B_4 \\
    \hline
    A_3\times B_1 + A_4\times B_3 & 
    A_3\times B_2 + A_4 \times B_4 
  \end{array}
\right) 
$$ 
{\'E}crire une fonction effectuant le produit de deux matrices en
commen{\c c}ant par les s{\'e}parer en blocs de taille ad{\'e}quate, puis en
regroupant les diff{\'e}rents r{\'e}sultats. \emph{Il est rappel{\'e} que la
  compr{\'e}hension de l'{\'e}nonc{\'e} fait partie de la note, et que les
  derni{\`e}res questions peuvent {\^e}tre longues.}
