\exo{Matrices (6 pts)}

\utilisation{partiel}{pcp}{2016}


Une \emph{matrice} est un tableau de lignes, et chaque ligne est un
tableau de \cfloat. Pour cet exercice, on suppose que les matrices
sont carrées, c'est-à-dire qu'il existe un entier \texttt{N} fixé (par
un \verb+#define+) tel que chaque matrice a exactement \texttt{N}
lignes qui contiennent chacune \texttt{N} éléments de type \cfloat.

\question (1 pt) Quel est le type en C d'une matrice ?

\question (1 pt) Écrire une fonction \cfun{zero} qui crée une matrice
dont tous les éléments sont à \(0\).

\question (2 pts) Écrire une fonction qui calcule la somme de 2 matrices.
La somme de \(A = ( a_{i,j})_{0\le i,j < N} \) et \(B = ( b_{i,j})_{0\le i,j < N} \)
est la matrice \(C = ( c_{i,j})_{0\le i,j < N} \) avec, pour tout \(0\le i,j < N\):
\[
c_{i,j} = a_{i,j} + b_{i,j}
\]

\question (2 pts) Écrire une fonction qui calcule le produit de 2 matrices.
Le produit de \(A = ( a_{i,j})_{0\le i,j < N} \) et \(B = ( b_{i,j})_{0\le i,j < N} \)
est la matrice \(C = ( c_{i,j})_{0\le i,j < N} \) avec, pour tout \(0\le i,j < N\):
\[
c_{i,j} = \Sigma_{k=0}^{N-1} a_{i,k} \cdot b_{k,j}
\]
