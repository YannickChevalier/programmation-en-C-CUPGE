\exo{Compréhension}

\begin{lstlisting}[language=C]
#include <stdio.h>

int
main ( int argc , char * argv[] )
{
  int x[3] ;
  printf ( "l'adresse de x[0] est %p.\n" , & x[0] ) ;
  printf ( "l'adresse de x[1] est %p.\n" , & x[1] ) ;
  printf ( "l'adresse de x[2] est %p.\n" , & x[2] ) ;
  printf ( "la valeur de x est %p.\n" , x ) ;
  printf ( "l'adresse de x est %p.\n" , & x ) ;
  printf ( "& x[1] - & x[0] = %ld\n" , & x[1] - & x[0] ) ;
  printf ( "& x[2] - & x[1] = %ld\n" , & x[2] - & x[1] ) ;
  printf ( "& x[2] - & x[0 = %ld\n" , & x[2] - & x[0] ) ;
  * ( & x[0] - 2 ) = 0 ;
  * ( & x[0] + 1 ) = 1 ;
  * ( & x[0] + 2 ) = 2 ;
  * ( & x[0] + 3 ) = 3 ;
  printf ( "x[0] = %d, x[1] = %d, x[2] = %d\n" , x[0] , x[1] , x[2] ) ;
  printf ( "sizeof ( x ) = %ld\n" , sizeof ( x ) ) ;
  printf ( "sizeof ( & x ) = %ld\n" , sizeof ( & x ) ) ;
  return 0 ;
}    
\end{lstlisting}

\question Dire ce qui sera affiché quand il est possible de le
déterminer, et quels sont les liens avec les autres valeurs quand on
peut seulement relier des valeurs entre elles (par exemple, savoir que
\(x = y\) même si on ne connaît pas la valeur de \(x\)).
