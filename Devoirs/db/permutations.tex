\exo{Permutations}

Le but de cet exercice est d'{\'e}crire des fonctions qui r{\'e}alisent des
permutations d'un tableau d'entiers. On veut faire une permutation
al{\'e}atoire en combinant les deux op{\'e}rations de base suivantes:
\begin{itemize}
\item Une transposition entre le premier {\'e}l{\'e}ment et le second {\'e}l{\'e}ment;
\item Un cycle dans lequel tous les {\'e}l{\'e}ments du tableau sauf le
  dernier sont d{\'e}plac{\'e}s d'une case vers la droite, et le dernier
  {\'e}l{\'e}ment est mis en premier.
\end{itemize}
On donne un exemple de combinaison et de cycles dans la Fig.~\ref{fig:permutation}.

\begin{figure}[hbp]
  \centering
  \begin{tabular}{|c|c|c|c|r}
    \cline{1-4}
    0&1&2&3& \hspace*{3em}tableau intial\\
    \cline{1-4}
    \multicolumn 5c {$\downarrow$}\\
    \cline{1-4}
    3&0&1&2& cycle\\
    \cline{1-4}
    \multicolumn 5c {$\downarrow$}\\
    \cline{1-4}
    0&3&1&2& transposition\\
    \cline{1-4}
    \multicolumn 5c {$\downarrow$}\\
    \cline{1-4}
    2&0&3&1& cycle\\
    \cline{1-4}
  \end{tabular}
  \caption{\label{fig:permutation}Exemple de combinaison de permutations}
\end{figure}

\question (1pt) Donnez la d{\'e}claration d'une structure \ctype{perm} qui contient:
\begin{itemize}
\item un pointeur \cvar{val} vers un tableau d'entiers ;
\item un entier \cvar{taille} qui est la taille de ce tableau ;
\end{itemize}

\question (2pts) {\'E}crivez une fonction \cfun{nouveau} qui prend en
entr{\'e}e un entier $N$ et rend une permutation dans laquelle \cvar{val}
pointe vers un tableau d'entiers de taille $N$.

\question (2pts) {\'E}crivez une fonction \cfun{initialise} qui prend en
entr{\'e}e une permutation et qui rend une permutation dans lequel la case
$i$ du tableau vaut $i$.

\question (2pts) {\'E}crivez une fonction \cfun{transpose} qui {\'e}change les
deux premiers {\'e}l{\'e}ments de la permutation donn{\'e}e en entr{\'e}e.

\question (2pts) {\'E}crivez une fonction \cfun{cycle} qui effectue une
permutation circulaire sur les {\'e}l{\'e}ments de la permutation donn{\'e}e en

entr{\'e}e.

\question (3pts) {\'E}crivez une fonction \cfun{aleatoire} qui effectue
une permutation circulaire sur les {\'e}l{\'e}ments de la permutation donn{\'e}e
en entr{\'e}e. Pour cela, elle prend en entr{\'e}e une permutation de taille
$N$, et effectue $N$ tirages al{\'e}atoires d'entier. {\`A} chaque tirage, si
l'entier tir{\'e} est pair, la fonction fait une transposition, et s'il
est impair, elle applique une permutation cyclique.

