\exo{Bicha{\^\i}nes}

On fixe la structure suivante pour contenir les couples
($\langle$clef$\rangle$,$\langle$valeur$\rangle$).
\begin{lstlisting}
  struct bichaine_s {
  int l1 ;
  int l2 ;
  char * clef ;
  char * valeur ;
};
\end{lstlisting}


\question (1 pt) D{\'e}finir le type \texttt{bichaine} des pointeurs vers
une structure \texttt{bichaine\_s}.

\question (2 pts) {\'E}crire une fonction \texttt{nouvelle\_bichaine} qui
prend en entr{\'e}e 2 cha{\^\i}nes de caract{\`e}res et leur longueur, et rend une
valeur de type \texttt{bichaine}.

\question (2 pts) {\'E}crire deux fonctions \texttt{clef} et
\texttt{valeur} qui prennent en entr{\'e}e une bicha{\^\i}ne et rendent
respectivement sa clef et sa valeur.

\question (1 pt) {\'E}crire une fonction qui affiche une bicha{\^\i}ne dans le
format \texttt{$\langle$clef$\rangle$: $\langle$valeur$\rangle$} suivi
dans cet ordre par les deux caract{\`e}res \verb+\n+ et \verb+\f+.

\question (3 pts) {\'E}crire une fonction qui prend en entr{\'e}e
\textbf{les adresses} de 2 valeurs de type \texttt{bichaine}, et rend 
un entier positif, nul, ou n{\'e}gatif si la premi{\`e}re est plus grande,
{\'e}gale, ou plus petite que la seconde. Pour comparer 2 bichaines:
\begin{itemize}
\item si les deux sont \texttt{NULL}, elles sont {\'e}gales ;
\item si une seule des deux est \texttt{NULL}, c'est la plus
  grande ;
\item si aucune n'est nulle, alors le r{\'e}sultat est celui de la
  comparaison de leur \textbf{clef}.
\end{itemize}
En particulier, on note que si deux bicha{\^\i}nes ont la m{\^e}me clef, mais
des valeurs diff{\'e}rentes, elles seront consid{\'e}r{\'e}es comme {\'e}gales.

