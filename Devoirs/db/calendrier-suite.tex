
\exo{Calendrier (bis) (5pts)}

Cette partie est plus complexe que
les pr{\'e}c{\'e}dentes ; il est d{\'e}conseill{\'e} de l'aborder avant d'avoir termin{\'e}
ce qui pr{\'e}c{\`e}de. On cherche {\`a} nouveau {\`a} mod{\'e}liser des dates, mais cette
fois dans le format jour-semaine-an:
\begin{flist}
\item le dimanche de la semaine 0 de 2012 est le 01/01/2012 ;
\item le mercredi de la semaine 2 de 2012 est le 10/01/2012 ;
\item pour plus d'exemples, voir les deux calendriers en annexe.
\end{flist}
Chaque ann{\'e}e est divis{\'e}e en 53 ou 54 semaines, dont une ou deux
peuvent {\^e}tre incompl{\`e}tes. Une semaine commence soit un lundi, soit le
premier janvier (semaine 0), et se termine soit le dimanche, soit le
31 d{\'e}cembre (semaine 52).  Une ann{\'e}e normale dure $52\times 7 + 1$
jours, et une ann{\'e}e bissextile dure $52 \times 7 + 2$ jours. Dans les
questions suivantes, il est conseill{\'e} de d{\'e}finir des fonctions
auxiliaires pour faire les calculs. Il est possible de r{\'e}utiliser les
fonctions de l'exercice \textsc{II}.

\question (1pt) Donnez les types de donn{\'e}es que vous allez utiliser.

\question (2pts) {\'E}crire une fonction rendant la date du jour de l'an
d'une ann{\'e}e $n$.

\question (2pts) {\'E}crire deux fonctions transformant une date
jour-mois-an en une date jour-semaine-an et r{\'e}ciproquement.
