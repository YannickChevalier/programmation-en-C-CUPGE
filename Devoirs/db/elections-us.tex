
\exo{Algorithmique}

On organise des {\'e}lections pour lesquelles il y a $C$ candidats et $S$
{\'E}tats. Les r{\'e}sultats des candidats dans chaque {\'E}tat sont stock{\'e}s dans
un tableau. S'il y a $C=3$ candidats et $S=2$
{\'e}tats, ce tableau contient par exemple:
\begin{center}
  \begin{tabular}{|l|c|c|c|}
    \hline
    {\'e}tat$\bslash$candidat & 0 & 1 & 2 \\
    \hline
    0 & 150 & 200 & 10 \\
    \hline
    1 & 33 & 120 & 3 \\
    \hline
  \end{tabular}
\end{center}


\question (2 pts) {\'E}crire une fonction \cfun{initialisation} qui prend en
  entr{\'e}e le nombre $C$ de candidats et le nombre $N$ d'{\'e}tats, et rend
  l'adresse d'un tableau pouvant stocker les
  r{\'e}sultats du vote pour une {\'e}lection {\`a} $C$ candidats et $S$ {\'e}tats.

\question (1 pt) {\'E}crire une fonction \cfun{affichageCandidatEtat} qui prend
  en entr{\'e}e le tableau stockant les r{\'e}sultats du vote, un num{\'e}ro de
  candidat $c$ et un num{\'e}ro d'{\'e}tat $s$ et qui affiche le nombre de voix
  obtenu par le candidat $c$ dans l'{\'e}tat $s$.

\question (1 pt) {\'E}crire une fonction \cfun{entreeCandidatEtat} qui prend en
  entr{\'e}e le tableau stockant les r{\'e}sultats du vote, un num{\'e}ro de
  candidat $c$ et un num{\'e}ro d'{\'e}tat $s$, qui demande {\`a} l'utilisateur le
  nombre de voix obtenu par le candidat $c$ dans l'{\'e}tat $s$, et qui
  met {\`a} jour le tableau de r{\'e}sultats.

\question (2 pts) {\'E}crire une fonction \cfun{totalVoixCandidat} qui rend le
  nombre total des voix obtenues par ce candidat.

\question (2 pts) {\'E}crire une fonction \cfun{totalVoix} qui rend le nombre
total de votes.


\question (2 pts) {\'E}crire une fonction \cfun{president} qui rend le candidat
  ayant remport{\'e} le plus d'{\'e}tats.

