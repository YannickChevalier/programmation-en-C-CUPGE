
On rappelle qu'un nombre:
\[
x=(-1)^s\cdot 2^{e-127}\cdot (1+\frac M{2^{23}})
\]
est codé sous la forme:
\begin{center}
  \begin{tabular}{|c|c|c|}
    \multicolumn 1c 0& \multicolumn 1c{\hbox to 8em{1\hfill 8}}&  \multicolumn 1c{\hbox to 23em {9\hfill 31}}\\
    \hline
    s &  e & M \\
    \hline
  \end{tabular}
\end{center}

\question (2 pts) Donnez le code hexadécimal du signe \(s\), de l'exposant \(e\), et de la mantisse \(M\), ainsi que le code entier sur 32 bits, du nombre \(-10,5\).

\begin{solution}
  On a:
  \begin{eqnarray*}
    -10,5&=&(-1)^1\cdot 2^3\cdot (1 + \frac{2,5}{8})\\
         &=& (-1)^1\cdot 2^{130-127}\cdot (1 + \frac{5}{16})\\
         &=& (-1)^1\cdot 2^{130-127}\cdot (1 + \frac{5}{2^4})\\
         &=& (-1)^1\cdot 2^{130-127}\cdot (1 + \frac{5\cdot 2^{19}}{2^{23}})\\
  \end{eqnarray*}
  Donc:
  \[
    s = 1 \hspace*{3em}e=130\hspace*{3em}M=5\cdot 2^{19}
  \]
  En base 16 (hexadécimal):
  Donc:
  \[
    s = 1 \hspace*{3em}e=128+2=82\hspace*{3em}M=5\cdot 8\cdot 2^{16}=280000
  \]
  Pour écrire en binaire le code entier, on écrit tout en base 2:
  \[
    \overbrace{\underbrace{1}_1}^s\,\,
    \overbrace{\underbrace{1000}_8\,\underbrace{0010}_2}^e\,\,
    \overbrace{\underbrace{010}_2\,\underbrace{1000}_8\,\underbrace{0000}_0\,\underbrace{0000}_0\,\underbrace{0000}_0\,\underbrace{0000}_0}^{M\text{ sur 23 bits}}
  \]
  et on regroupe par groupe de 4, qu'on traduit en hexadécimal:
  \[
    \underbrace{1100}_{12=C}\,\underbrace{0001}_1\,\underbrace{0010}_2\,\underbrace{1000}_8\,\underbrace{0000}_0\,\underbrace{0000}_0\,\underbrace{0000}_0\,\underbrace{0000}_0
  \]
  Donc le code entier en hexadécimal est \(C1280000\).
\end{solution}

\question (2 pts) Donnez le code hexadécimal du signe \(s\), de l'exposant \(e\), et de la mantisse \(M\), ainsi que le code entier sur 32 bits, du nombre \(3,25\).

\question (2 pts) Donnez le code hexadécimal du signe \(s\), de l'exposant \(e\), et de la mantisse \(M\), ainsi que le code entier sur 32 bits, du nombre \(-3,875\).
