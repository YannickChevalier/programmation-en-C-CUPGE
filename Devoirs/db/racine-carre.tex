\exo{Calcul d'une racine carr{\'e}e (4,5 pts)}

On peut calculer la racine carr{\'e}e d'un nombre positif $d$ en calculant
la limite d'une suite r{\'e}cursive:
$$
\left\lbrace
\begin{array}{rcl}
u_0 &=& 1 \\
u_{n+1}&=& \frac 12 \cdot ( u_n + \frac d {u_n} )\\
\end{array}
\right.
$$
Pour la suite des questions, il peut {\^e}tre utile de d{\'e}finir d'autres
fonctions que celles demand{\'e}es.

\question (2,5 pts) {\'E}crire une fonction \cfun{racine17} qui
retourne la racine carr{\'e}e de $17$ {\`a} $10^{-3}$ pr{\`e}s. On
admettra que cette pr{\'e}cision est atteinte lorsque deux {\'e}l{\'e}ments
cons{\'e}cutifs de la suite r{\'e}cursive sont distants de moins de $10^{-3}$.

\question (1 pt) {\'E}crire une fonction \cfun{racine} qui calcule la racine
  carr{\'e}e d'un nombre de type \ctype{double} {\`a} $\varepsilon$
  pr{\`e}s.

\question (1 pt) {\'E}crire une fonction qui demande un nombre de type
\ctype{double} {\`a} l'utilisateur et affiche la racine carr{\'e}e de
ce nombre (on supposera que les entr{\'e}es de l'utilisateur forment
du premier coup un nombre de type \ctype{double} strictement positif.)

