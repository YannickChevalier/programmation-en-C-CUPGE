\exo{Conversion vers d'autres bases}

\question (2 pts) Traduire \(1,23\) en base 2 avec 10 chiffres
significatifs (en base 2).


\begin{solution}
  Pour avoir 10 chiffres en base 2, il en faut 3 en base 16 (et on
  aura une estimation trop précise ayant 12 chiffres en base 2):
  \begin{eqnarray*}
    0,23 \cdot 16 &=& \fbox{3},68\\
    0,68\cdot 16 &=& \fbox{10},88\\
    0,88\cdot 16 &=& \fbox{14},08\\
  \end{eqnarray*}
  Donc:
  \begin{eqnarray*}
    1,23 &=& (1,3AE)_{16}\\
    &=&(1,0011\,1010\,1110)_2\\
    &=&\fbox{\((1,0011\,1010\,11)_2\)}\hspace*{2em}\text{avec 10 chiffres significatifs}\\
  \end{eqnarray*}
  
\end{solution}

\question (2 pts) Traduire exactement \(1,25\) en base 2 et en base 16.


\begin{solution}
  On commence par la traduction en base 16:
  \begin{eqnarray*}
    0,25 \cdot 16 &=& \fbox{4},0\\
  \end{eqnarray*}
  Donc:
  \begin{eqnarray*}
    1,25 &=& (1,4)_{16}\\
    &=&(1,01)_2\\
  \end{eqnarray*}
  
\end{solution}


\question (2 pts) Traduire exactement \(44,125\) en base 2 et en base 16.


\begin{solution}
  On commence par la traduction en base 16:
  \begin{eqnarray*}
    0,125 \cdot 16 &=& \fbox{2},0\\
    44=2\cdot 16 + \underbrace{12}_{C}
  \end{eqnarray*}
  Donc:
  \begin{eqnarray*}
    44,125 &=& (2C,2)_{16}\\
    &=&(101100,0010)_2\\
  \end{eqnarray*}
  
\end{solution}

