

\exo{Algorithmique (4 pts)}

Ecouter de la musique peut être très agréable mais lorsqu'un morceau
est vraiment très répétitif, il arrive parfois qu'on s'ennuie un
peu. Aussi le professeur de composition musicale du conservatoire a
décidé d'imposer une règle très stricte : quand il relit les morceaux
composés par ses élèves, dès qu'il voit deux notes identiques côte à
côte, il les efface toutes les deux ! Il continue ainsi d'effacer tant
qu'il existe deux notes égales consécutives.

Ce travail étant long et fastidieux, il se demande s'il n'est pas
possible de l'automatiser.


\question (4 pts) Écrire une fonction C prenant en argument une chaîne
de caractères représentant le morceau de musique (on supposera qu'elle
est correcte) et qui affiche version du morceau "corrigée" où tous les
doublons sont supprimés tant qu'il en existe. Les notes de musiques
sont représentées par les lettres 'a', 'b', 'c', 'd', 'e', 'f' et 'g'.
