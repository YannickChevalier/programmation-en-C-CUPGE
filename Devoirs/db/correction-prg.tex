\exo{Correction d'un programme C}

\question (5 pts) Corriger le programme suivant pour qu'il affiche le maximum
  de 3 entiers (r{\'e}pondre sur le sujet)

\begin{Ccode}
\ctab{}\cinclude{\syslib{stdio}} 
\ctab{}\cint \cfun{main} (\cint \cvar{argc}, \cchar *\cvar{argv}[])\lb
\ctab{}  \cint \cvar{nb1},\cvar{nb2},\cvar{nb3},\cvar{n1},\cvar{n2},*
\cvar{n3},
\ctab{}  \cint \cvar{max};
\ctab{}  \ccomment{Demande des nombres} 
\ctab{}  \cfun{printf}(\cstring{Donner votre premier nombre : });
\ctab{}  \cdo \{ \cvar{nb1} = \cfun{scanf}(\cstring{\%f},\&\cvar{n1}) ; 
\cwhile (\cfun{getchar}() != '\n' ) ; \} \cwhile ( \cvar{n1} == 0 );    
\ctab{}  \cfun{printf}(\cstring{Donner votre deuxieme nombre : });
\ctab{}  \cdo \{ \cvar{nb2} = \cfun{scanf}(\cstring{\%d},\&\cvar{n2}) ; 
\cwhile (\cfun{getchar}() != '\n' ) ; \} \cwhile ( \cvar{nb2} = 0 );
\ctab{}  \cfun{printf}(\cstring{Donner votre troisieme nombre : });     
   
\ctab{}  \cdo \{ \cvar{nb3} = \cfun{scanf}(\cstring{\%s},\cvar{n3}) ; 
\cwhile (\cfun{getchar}() != '\n' ) ; \} \cwhile ( \cvar{nb3} = 0 );      
                         
\ctab{}  \cif (\cvar{n1}>\cvar{n2}) ;                                   
    
\ctab{}    \lb
\ctab{}      \cvar{max} = \cvar{n1};
\ctab{}      \cif (\cvar{n3} > \cvar{max} )
\ctab{}          \hspace*{2em}\cvar{max} = *\cvar{n3};
\ctab{}    \rb
\ctab{}  \celse
\ctab{}    \lb
\ctab{}      \cvar{max} = \cvar{n2};
\ctab{}      \cif (\cvar{n3} > \cvar{max} )
\ctab{}        \lb
\ctab{}          \cvar{max} = *\cvar{n3}
\ctab{}        \rb 
\ctab{}    \rb
\ctab{}  \cfun{printf} (\cstring{Le maximum de \%d \%d \%d est \%d \n},
\&\cvar{n1},\&\cvar{n2},\cvar{n3},\&\cvar{max}); 
\ctab{}  \creturn 0;
\ctab{}\rb
\end{Ccode}

