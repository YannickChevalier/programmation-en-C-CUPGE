\exo{Algorithmique (7 pts)}

La bibliothèque universitaire a actuellement des difficultés à gérer
son stock car elle est très fréquentée en ce moment et ne dispose pas
d'un nombre suffisant d'employés. Vous décidez de l'aider en écrivant
un programme informatique permettant de soulager le travail des
employés.

\paragraph{Ce que doit faire votre programme:}
La bibliothèque possède \textit{nbLivres} livres indexés de 0 à
\(\text{\it nbLivres} - 1\). Chaque jour, un certain nombre de clients
demandent à emprunter des livres pour une certaine durée. Si le livre
est disponible, la requête du client est satisfaite, sinon le client
repart sans livre.

Votre programme doit d'abord lire sur une première ligne deux entiers:
\(\text{\it nbLivres} \le 10000\) et \textit{nbJours}. Pour chacun des
jours, votre programme lira un entier \textit{nbClients} sur une ligne puis
\textit{nbClients} lignes de deux entiers. Le premier entier correspond à
l'indice du livre et le second la durée correspondante. (voir
l'exemple d'entrée). Il affichera ensuite, sur des lignes séparées,
pour chaque client un 1 si le livre peut être prêté et 0 dans le cas
contraire.

On remarquera que si un client emprunte un livre le jour
\textit{iJour} pendant une durée \textit{duree} alors celui-ci ne sera
de nouveau disponible qu'au jour \(\text{\it iJour} + \text{\it
  duree}\). De plus, si plusieurs personnes demandent le même livre
pendant une journée, seule la première a une chance d'être satisfaite.

\paragraph{Exemple d'entrée.} Les lignes commen\c cant par \texttt{//}
sont des commentaires décrivant le contenu de la ligne suivante.

{\obeylines
\texttt{// nombre de livres}
10
\texttt{// nombre de jours à traiter}
2
\texttt{// nombre de clients le premier jour}
1
\texttt{// emprunt du livre 3 pour 2 jours}
3 2
\texttt{// nombre de clients le second jour}
2
\texttt{// emprunt du livre 3 pour 1 journée}
3 1
\texttt{// emprunt du livre 0 pour 2 jours}
0 2
}
Le programme doit afficher:
{\obeylines
\texttt{// premier emprunt accepté}
1
\texttt{// deuxième emprunt refusé car le livre a été prêté la veille}
0
\texttt{// troisième emprunt accepté}
1
}

\question (1 pt) Écrire une fonction qui lit le nombre de livres et
alloue un tableau contenant, pour chaque livre, le jour (un entier)
o\`u il sera livre. Initialement, aucun livre n'est emprunté.

\question (2 pts) Écrire une fonction qui prend entrée le tableau des
disponibilités, lit une ligne correspondant à un emprunt (2 entiers), et:
\begin{itemize}
\item affiche 0 si l'emprunt est refusé, et 1 s'il est accepté;
\item met à jour le tableau si l'emprunt est accepté.
\end{itemize}

\question (2 pts) Écrire une fonction qui prend entrée le tableau des
disponibilités, lit le nombre d'emprunts sur une journée, et met le
tableau à jour pour cette journée.

\question (2 pts) Écrire le programme résolvant le problème.


