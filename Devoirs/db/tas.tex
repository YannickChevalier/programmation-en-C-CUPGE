\exo{Tas}

\paragraph{Pr{\'e}liminaires.}
Un \emph{arbre} est soit un arbre vide, soit est un n\oe{}ud (la
\emph{racine} de cet arbre) qui est le p{\`e}re de deux \emph{fils}, son
fils droit et son fils gauche, qui sont eux-m{\^e}mes des arbres.

Les \emph{tas} sont des tableaux qui codent certains arbres de la
mani{\`e}re suivante:
\begin{itemize}\setlength\itemsep {-3pt}
\item la valeur de la racine de l'arbre est dans la case $0$;
\item si la valeur d'un n\oe ud est stock{\'e}e dans la case $i$, alors
  la valeur de la racine de son fils gauche est stock{\'e}e dans la case
  $2\times i + 1$, et la valeur de la racine de son fils droit est
  stock{\'e}e dans la case $2\times i + 2$;
\item la valeur d'un n\oe ud est toujours plus petite que la valeur
  de ses fils.
\end{itemize}
Pour savoir quand s'arr{\^e}tent les valeurs du tas (les cases blanches de
la Figure~\ref{fig:graph}), on a besoin, en plus du tableau, du nombre
d'{\'e}l{\'e}ments dans le tas. Pour d{\'e}crire des tas en C, on utilise les
d{\'e}clarations suivantes:
\begin{Ccode}
  \ctab\cstruct \ctype{tas\_s} \lb
  \ctab \cint \cvar{nb\_elts} ; \ccomment{nombre d'{\'e}l{\'e}ments dans le tas}
  \ctab \cint * val ; \ccomment{tableau des valeurs}
  \ctab\rb ;
  \ctab\ctypedef \cstruct \ctype{tas\_s} * \ctype{tas} ;
\end{Ccode}

\begin{figure}
  \centering
        \begin{subfigure}[b]{0.5\textwidth}
                \centering
                \begin{tikzpicture}[->,>=stealth',level/.style={sibling distance = 4cm/#1,
                    level distance = 1cm}] 
                  \node [arn_n] {5}
                  child{ node [arn_n] {7}
                    child{ node [arn_n] {36} 
                      child{ node [arn_n] {38}}
                      child{ node [arn_n] {39}}
                    }
                    child{ node [arn_n] {51}
                  }
                }
                child{ node [arn_n] {10} 
                    child{ node [arn_n] {27} 
                    }
                    child{ node [arn_n] {49}}                             
                  }
                ; 
                \end{tikzpicture}
                \caption{Un tas, dont la racine a pour valeur $5$, son
                  fils gauche a une racine de valeur $7$ et son fils
                  droit a pour racine un n\oe ud dont la  valeur est 10. Les
                  arbres vides ne sont pas repr{\'e}sent{\'e}s.}
                \label{fig:arbre}
        \end{subfigure}%
        \qquad
        \begin{subfigure}[b]{0.4\textwidth}
                \centering
                %\small
                \begin{tabular}[c]{|c|c|c|c|c|c|c|c|c|c|c|c|}
                  \hline
                  5 & 7 & 10 & 36 & 51 & 27 & 49 & 38 & 39 &  &  &  \\
                  \hline
                  \multicolumn 2l{0} & \multicolumn 3l{2} &
                  \multicolumn 7l {5} \\
                \end{tabular}
                \vspace*{2em}
                \caption{Le tableau correspondant {\`a} ce tas: la valeur
                  de la racine est dans la case $0$, celle de son fils
                droit dans la case $2=2\times 0+2$, et celle du fils
                gauche de ce fils droit est dans la case $5=2\times 2
                + 1$.}
                \label{fig:tableau}
        \end{subfigure}
  \caption{Tas et leur repr{\'e}sentation par un tableau}
  \label{fig:graph}
\end{figure}

\begin{figure}
  \centering
          \begin{subfigure}[b]{0.45\textwidth}
                \centering
                \begin{tikzpicture}[->,>=stealth',level/.style={sibling distance = 4cm/#1,
                    level distance = 1cm}] 
                  \node [arn_n] {7}
                  child{ node [arn_n] {36}
                    child{ node [arn_n] {38} 
                      child{ node [arn_n] {5}}
                      child{ node [arn_n] {39}}
                    }
                    child{ node [arn_n] {51}
                  }
                }
                child{ node [arn_n] {10} 
                    child{ node [arn_n] {27} 
                    }
                    child{ node [arn_n] {49}}                             
                  }
                ; 
                \end{tikzpicture}
                \caption{Le tas de la Figure~\ref{fig:arbre} apr{\`e}s
                  l'{\'e}tape 2 de la suppression.}
                \label{fig:tas:suppression:etape2}
        \end{subfigure}%
        \qquad
          \begin{subfigure}[b]{0.45\textwidth}
                \centering
                \begin{tikzpicture}[->,>=stealth',level/.style={sibling distance = 4cm/#1,
                    level distance = 1cm}] 
                  \node [arn_n] {7}
                  child{ node [arn_n] {36}
                    child{ node [arn_n] {38} 
                      child{ node [arn_n] {39}}
                      child{ node [arn_x] {}}
                    }
                    child{ node [arn_n] {51}
                  }
                }
                child{ node [arn_n] {10} 
                    child{ node [arn_n] {27} 
                    }
                    child{ node [arn_n] {49}}                             
                  }
                ; 
                \end{tikzpicture}
                \caption{Le tas de la Figure~\ref{fig:arbre} apr{\`e}s
                  l'{\'e}tape 5 de la suppression. Le n\oe ud vide
                  repr{\'e}sente la d{\'e}cr{\'e}mentation du nombre
                  d'{\'e}l{\'e}ments. $39$ est plus grand que $38$, donc
                  l'{\'e}tape 6 ne fait rien}
                \label{fig:tas:suppression:etape2}
        \end{subfigure}%
  \caption{Tas aux diff{\'e}rentes {\'e}tapes de la suppression.}
  \label{fig:tas:suppression:fig}
\end{figure}

\question (0.5 + 0.5 + 1 = 2 pts) {\'E}crire trois fonctions
\cfun{fils\_gauche}, \cfun{fils\_droit}, et \cfun{pere} qui, en
fonction d'un indice $i$, calculent l'indice correspondant
respectivement au fils gauche, au fils droit, et au p{\`e}re du n\oe ud
dont la valeur est stock{\'e}e dans la case $i$. \emph{Par exemple,
  \cfun{pere}(5)=2, et \cfun{fils\_gauche}(2)=5.}

\question (1 pt) {\'E}crire une fonction qui prend en entr{\'e}e un nombre
maximal d'{\'e}l{\'e}ments pour un tas, et qui rend un tas contenant $0$
{\'e}l{\'e}ments mais \emph{pouvant} contenir ce nombre maximal d'{\'e}l{\'e}ments.

L'algorithme permettant d'ins{\'e}rer un {\'e}l{\'e}ment dans un tas est donn{\'e}
informellement dans la Figure~\ref{fig:tas:insertion}.
\begin{figure}
  \begin{subfigure}[t]{0.45\textwidth}
    \centering
    \fbox{\parbox{\textwidth}{\begin{enumerate}\setlength\itemsep {-3pt}
        \item On note $i$ l'indice courant. Au d{\'e}part, l'indice courant
          est l'indice de la premi{\`e}re case vide dans le
          tableau (dans la Figure~\ref{fig:tableau}, $i$ vaut au d{\'e}part $9$);
        \item mettre l'{\'e}l{\'e}ment {\`a} ajouter dans la case d'indice $i$;
        \item Tant que:
          \begin{itemize}\setlength\itemsep {-3pt}
          \item l'indice courant est strictement positif;
          \item et la valeur du p{\`e}re de l'indice courant est plus grande que
            la valeur de l'indice courant
          \end{itemize}
          {\'e}changer ces deux valeurs, et positionner l'indice courant sur son
          p{\`e}re;
        \item rendre le tas;
        \end{enumerate}}\quad}
    \caption{Algorithme d'insertion d'un {\'e}l{\'e}ment.}
    \label{fig:tas:insertion}
  \end{subfigure}
  \qquad
  \begin{subfigure}[t]{0.45\textwidth}
    \centering
    \fbox{\parbox{\textwidth}{\begin{enumerate}\setlength\itemsep {-3pt}
        \item On note $i$ l'indice courant. Au d{\'e}part, l'indice courant
          vaut $0$;
        \item Tant que le fils droit de l'indice courant est dans le
          tableau:
          \begin{itemize}
          \item {\'e}changer la valeur du n\oe ud courant avec celle du plus
            petit de ses fils;
          \item l'indice courant devient celui du fils avec lequel
            l'{\'e}change a {\'e}t{\'e} fait.
          \end{itemize}
        \item Si l'{\'e}l{\'e}ment courant a un fils gauche, on {\'e}change ces deux
          {\'e}l{\'e}ments, et l'{\'e}l{\'e}ment courant devient l'indice du fils gauche;
        \item on {\'e}change ensuite la valeur de l'{\'e}l{\'e}ment courant et celle
          du dernier {\'e}l{\'e}ment du tas;
        \item on d{\'e}cr{\'e}mente de $1$ le nombre d'{\'e}l{\'e}ments dans le tas;
        \item on remonte, comme pour l'insertion, l'{\'e}l{\'e}ment courant tant
          qu'il est plus petit que son p{\`e}re.  
        \end{enumerate}}\quad}
    \caption{Algorithme de suppression du plus petit {\'e}l{\'e}ment.}
    \label{fig:tas:suppression}
  \end{subfigure}
  \label{fig:tas:algos}
  \caption{Algorithmes sur les tas}
\end{figure}

\question (1 pt) Justifiez (informellement) que l'insertion d'un
{\'e}l{\'e}ment dans un tas permet d'obtenir un tas, \textit{i.e.}, que dans
le tas obtenu, tout {\'e}l{\'e}ment est plus petit que ses fils.

\question (2 pts) {\'E}crire la fonction d'insertion d'un {\'e}l{\'e}ment en C. On suppose
qu'il y a strictement moins d'{\'e}l{\'e}ments dans le tas que le nombre
maximal qu'il ne peut en contenir.

\question (2 pts) {\'E}crire la fonction de suppression du plus petit
{\'e}l{\'e}ment en C en se basant sur l'algorithme de la
figure~\ref{fig:tas:suppression}. On suppose qu'il y a au moins un
{\'e}l{\'e}ment dans le tas.


