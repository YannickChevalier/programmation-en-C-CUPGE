\exo{Algorithmique}

Pour une formation, nous disposons de ta\-bleaux de notes (de type
\emph{float}) contenant les moyennes des {\'e}tudiants
(num{\'e}rot{\'e}s {\`a} partir de $0$) et d'un tableau contenant le
nom complet de chaque {\'e}tudiant Par exemple, si le nom complet d'un
{\'e}tudiant est \texttt{Bernard Pivot}, qu'il a eu \texttt{12.5} en
math{\'e}matiques, et que son num{\'e}ro est 5:
\begin{flist}
\item la cinqui{\`e}me case du tableau de noms vaut \cstring{Bernard
    Pivot}, et
\item la cinqui{\`e}me case du tableau de notes correspondant aux
  math{\'e}matiques vaut \texttt{12.5}.
\end{flist}

\begin{center}
  \fbox{\begin{minipage}{.9\linewidth} Dans les questions suivantes,
      pour chaque fonction, et dans cet ordre:
    \begin{flist}
    \item Indiquez le profil de la fonction que vous allez {\'e}crire ;
    \item D{\'e}crivez en une phrase ses arguments et la valeur qu'elle
      retourne ;
    \item {\'E}crivez la fonction en C.
    \end{flist}
  \end{minipage}}
\end{center}


\question (1 pt) Quel est le type d'un tableau de noms ?

\question (2 pts) {\'E}crire une fonction \cfun{Affiche} qui prend en
entr{\'e}e un tableau de notes, un tableau de noms, et le nombre
d'{\'e}tu\-diants, et qui affiche les notes (avec l'{\'e}tudiant
correspondant) avec une pr{\'e}cision de 3 chiffres apr{\`e}s la
virgule.

\question (2 pts) {\'E}crire une fonction \cfun{Moyenne} qui prend en
entr{\'e}e un tableau de \cfloat repr{\'e}sentant les notes d'une
mati{\`e}re, le nombre d'{\'e}tu\-diants, et qui retourne la moyenne
de la classe pour la mati{\`e}re repr{\'e}sent{\'e}e par le tableau.


\question (2 pts) {\'E}crire une fonction \cfun{Major} qui, a partir
d'un tableau de notes, retourne le num{\'e}ro de l'{\'e}tudiant ayant
obtenu la meilleure note

\question (2 pts) Faire une fonction \cfun{Passe} qui prend en
entr{\'e}e les un tableau de notes et le nombre d'{\'e}tudiants, et
rend en sortie un tableau $t$ d'entiers avec $t[i]=1$ si
l'{\'e}l{\`e}ve de num{\'e}ro $i$ a la moyenne, et $t[i]=0$ sinon.

\question (4 pts) {\'E}crire une fonction \cfun{MiseAJour} qui prend un num{\'e}ro
  d'{\'e}tudiant et propose de mettre {\`a} jour la moyenne de cet {\'e}tudiant
  dans une mati{\`e}re. Pour cela, cette fonction:
  \begin{flist}
  \item affiche la note courante courante de l'{\'e}tudiant dans cette
    mati{\`e}re;
  \item demande une nouvelle note, puis :
    \begin{flist}
    \item si l'utilisateur rentre une note, la fonction met {\`a} jour la
      note de l'{\'e}tudiant ;
    \item sinon, l'ancienne note est inchang{\'e}e.
    \end{flist}
  \end{flist}
