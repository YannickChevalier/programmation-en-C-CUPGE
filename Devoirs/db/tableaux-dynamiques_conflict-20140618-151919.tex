\exo{Tableaux dynamiques (12 points)}

On veut cr{\'e}er une structure permettant d'adapter la taille d'un
tableau au nombre d'{\'e}l{\'e}ments qu'il contient. En th{\'e}orie, une mani{\`e}re
efficace d'adapter la taille est, pour un tableau de $n$ {\'e}l{\'e}ments
pouvant en contenir $N$:
\begin{quote}
  \begin{itemize}\itemsep=-3pt
  \item Si $n = N$, \textit{i.e.}, le tableau est plein, alors pour ajouter
    un élément au tableau il faut commencer par le changer en un tableau de
    taille $2\times N$ ;
  \item Si $n\le N$, \textit{i.e.}, le tableau est aux $\nicefrac 34$
    vide et si $N>10$, alors avant d'enlever un élément il faut commencer
    par changer le tableau en un tableau de taille
    $\nicefrac N 2$.
  \end{itemize}
\end{quote}


\question (1 pt) {\'E}crire une fonction qui prend en entr{\'e}e un entier
positif $n$ et qui rend la plus petite puissance de $2$ qui est plus
grande que $n$: $f(n) = \min \lbrace N \,\vert\, N=2^k \text{ et } N\ge n\rbrace$

\question (1 pt) D{\'e}clarer le type \ctype{td} des pointeurs vers une
structure \ctype{td\_s} qui contient:
\begin{quote}
  \begin{itemize}\setlength\itemsep {-3pt}
  \item un nombre maximal d'{\'e}l{\'e}ments;
  \item le nombre courant d'{\'e}l{\'e}ments;
  \item un tableau \cvar{val} d'{\'e}l{\'e}ments.
  \end{itemize}
\end{quote}

\question (2 pts) {\'E}crire une fonction rendant un tableau dynamique
pouvant contenir $N$ {\'e}l{\'e}ments mais n'en contenant aucun. Le nombre $N$
est pass{\'e} en argument.

\question (2 pts) {\'E}crire une fonction qui prend en entr{\'e}e deux
tableaux dynamiques tels que le premier peut contenir plus d'{\'e}l{\'e}ments
que n'en poss{\`e}de le second, et qui recopie tous les {\'e}l{\'e}ments du second
tableau dans le premier. Cette fonction rend le nombre d'{\'e}l{\'e}ments
copi{\'e}s.


\question (2 pts) {\'E}crire deux fonctions:
\begin{itemize}
\item une qui ins{\`e}re un {\'e}l{\'e}ment {\`a} la fin d'un tableau dynamique pass{\'e}
  en argument;
\item une qui supprime le dernier {\'e}l{\'e}ment d'un tableau dynamique pass{\'e}
  en argument.
\end{itemize}
Si $N$ d{\'e}signe le nombre d'{\'e}l{\'e}ments que peut contenir le tableau
dynamique, et $n$ le nombre d'{\'e}l{\'e}ments contenus apr{\`e}s l'insertion ou
la suppression, on recopiera le tableau dans un tableau pouvant
contenir $\nicefrac N2$ {\'e}l{\'e}ments si $n\le \nicefrac N4$, ou $2\times
N$ si $n=N$


\question (1 pt) Donnez une mani{\`e}re de repr{\'e}senter un tableau
dynamique dans un fichier. \emph{Cette question est une pr{\'e}paration
  aux deux questions suivantes.}

\question (1.5 pts) {\'E}crire une fonction prenant un tableau dynamique
et un argument de type \ctype{FILE *} correspondant {\`a} un fichier
ouvert en {\'e}criture, et stockant ce tableau dynamique suivant le format
d{\'e}fini {\`a} la question pr{\'e}c{\'e}dente. 

\question (1.5 pts) {\'E}crire une fonction prenant un argument de type
\ctype{FILE *} correspondant {\`a} un fichier ouvert en lecture, et lisant
un tableau dynamique qui y est stock{\'e} suivant le format d{\'e}fini {\`a} la
question pr{\'e}c{\'e}dente. Cette fonction renvoie ce tableau dynamique.
