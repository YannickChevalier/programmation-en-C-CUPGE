\exo{Tableaux de bicha{\^\i}nes}

Pour l'envoi de messages \textsf{HTTP}, on a juste besoin d'ajouter
des bicha{\^\i}nes dans un tableau initialement vide. Pour cela, on va
cr{\'e}er des tableaux d'une taille fix{\'e}e
(\textsc{Taille\_initiale\_tableau\_bc}), et s'il y a besoin, lors de
l'ajout d'une bicha{\^\i}ne, on augmentera cette taille.

\question (1 pt) D{\'e}finir une structure \texttt{tab\_bc\_s} qui contient
l'adresse d'un tableau de bicha{\^\i}nes et le nombre d'{\'e}l{\'e}ments dans ce
tableau.

\question (1 pt) D{\'e}finir le type \texttt{tab\_bc} des pointeurs vers
une structure \texttt{tab\_bc\_s}.

\question (1 pt) D{\'e}finir la constante
\textsc{Taille\_initiale\_tableau\_bc} comme valant 10.

Dans le tableau de bicha{\^\i}nes d'une structure \texttt{tab\_bc\_s}, un
{\'e}l{\'e}ment peut {\^e}tre soit \texttt{NULL} (pour indiquer une case vide),
soit une valeur de type \texttt{bichaine}.


\question (2 pts ) {\'E}crire une fonction \texttt{nouveau\_tab\_bc} qui
renvoie une valeur de type \texttt{tab\_bc} dont le tableau contient
\textsc{Taille\_initiale\_tableau\_bc} cases \textbf{vide}.

\question (3 pts) {\'E}crire une fonction \texttt{ajoute\_tab\_bc} qui
ins{\`e}re une bicha{\^\i}ne au premier emplacement \texttt{NULL} d'un tableau
de bicha{\^\i}nes. S'il n'y a aucun emplacement disponible, cette fonction
va cr{\'e}er un nouveau tableau dont la taille est 2 fois l'ancienne
taille, correctement initialis{\'e}, puis recopier l'ancien dans le nouveau, et
enfin ins{\'e}rer la nouvelle bicha{\^\i}ne dans ce tableau (toujours au
premier emplacement non-nul).

\vspace*{3em}

Lors de la lecture d'un message, la norme \textsf{RFC 2616} indique
que si deux ``bicha{\^\i}nes'' avec la m{\^e}me clef apparaissent, il faut
fusionner les valeurs en les s{\'e}parant par des points-virgule. Par
exemple, il faut fusionner les bicha{\^\i}nes \texttt{("codage","8859-1")}
et \texttt{("codage","UTF-8")} en une seule bicha{\^\i}ne
\texttt{("codage","8859-1;UTF-8")}. Pour effectuer cette fusion, on
va:
\begin{enumerate}
\item Trier le tableau de bicha{\^\i}nes en utilisant la fonction
  \texttt{qsort} d{\'e}finie dans \texttt{stdlib.h}:
$$
\text{\tt qsort}: \text{\tt void} * \times \text{\tt int} \times
\text{\tt int} \times (\text{\tt int} ()( \text{\tt void} * \times
\text{\tt void} * )) \to \text{\tt void}
$$
avec par exemple \texttt{qsort ( t , 10 , sizeof ( int ) , intcmp )}
qui trie un tableau $t$ de $10$ {\'e}l{\'e}ments (des \texttt{int}) dont
chacun est de la taille \texttt{sizeof ( int )} en comparant les
{\'e}l{\'e}ments avec une fonction \texttt{intcmp} qui prend en entr{\'e}e les
adresses de deux entiers et renvoie leur diff{\'e}rence.
\item Pour chaque {\'e}l{\'e}ment non-nul du tableau tri{\'e}:
  \begin{itemize}
  \item Pour chaque {\'e}l{\'e}ment suivant qui a la m{\^e}me clef:
    \begin{itemize}
    \item copier la valeur de l'{\'e}l{\'e}ment suivant {\`a} la suite de celle de
      l'{\'e}l{\'e}ment courant  (en r{\'e}servant de la m{\'e}moire suppl{\'e}mentaire si
      n{\'e}cessaire) ;
    \item mettre l'{\'e}l{\'e}ment suivant {\`a} \texttt{NULL}
    \item passer {\`a} l'{\'e}l{\'e}ment d'apr{\`e}s
    \end{itemize}
  \item passer au premier {\'e}l{\'e}ment suivant non-nul.
  \end{itemize}
\item Re-trier le tableau pour mettre les {\'e}l{\'e}ments nuls en fin de
  tableau.
\end{enumerate}

\paragraph{Exemple d'ex{\'e}cution.}
\begin{center}
  \begin{tabular}[c]{|l|l|}\hline
    encodage & 8859-1\\ \hline
    referer & "http://www.irit.fr/"\\ \hline
    encodage & UTF-8\\ \hline
  \end{tabular}

  \vspace*{1em}

  $\downarrow$ \makebox[0pt][l]{Premier tri}

  \vspace*{1em}

  \begin{tabular}[c]{|l|l|}\hline
    encodage & 8859-1\\ \hline
    encodage & UTF-8\\ \hline
    referer & "http://www.irit.fr/"\\ \hline
  \end{tabular}

  \vspace*{1em}
  $\downarrow$ \makebox[0pt][l]{Fusion des {\'e}l{\'e}ments de m{\^e}me clef}

  \vspace*{1em}

  \begin{tabular}[c]{|l|l|}\hline
    encodage & 8859-1;UTF-8\\ \hline
    \multicolumn 2{|c|} {\texttt{NULL}} \\ \hline
    referer & "http://www.irit.fr/"\\ \hline
  \end{tabular}

  \vspace*{1em}

  $\downarrow$ \makebox[0pt][l]{Second tri}

  \vspace*{1em}

  \begin{tabular}[c]{|l|l|}\hline
    encodage & 8859-1;UTF-8\\ \hline
    referer & "http://www.irit.fr/"\\ \hline
    \multicolumn 2{|c|} {\texttt{NULL}} \\ \hline
  \end{tabular}
\end{center}

\question (3 pts) {\'E}crire une fonction \texttt{normalisation} qui prend
en entr{\'e}e une valeur de type \texttt{tab\_bc}, et impl{\'e}mente
l'algorithme ci-dessus.

