


\exo{Jeu de la vie}

Le \emph{jeu de la vie} est un jeu informatique qui se joue dans un
tableau {\`a} deux dimensions $n\times m$. Chaque case de ce tableau,
appel{\'e}e \emph{cellule}, est soit vivante, soit morte. Chaque cellule a
8 \emph{voisines} (dans les 8 directions nord, nord-ouest, ouest,
sud-ouest, sud, sud-est, est, nord-est). Les cellules au-del{\`a} des
bords sont toutes consid{\'e}r{\'e}es comme mortes. Un \emph{{\'e}tat} est un
tableau dans lequel chaque cellule est soit vivante, soit morte.

On passe de l'{\'e}tat courant {\`a} l'{\'e}tat suivant en examinant, pour
chaque cellule en dehors des bords, le nombre $k$ de voisines en
vie\,:
\begin{itemize}
\item s'il y en a 3, et que la cellule est morte, elle devient vivante\,;
\item s'il y en a 2 ou 3, et que la cellule est vivante, elle le reste\,;
\item Dans tous les autres cas, la cellule meurt.
\end{itemize}


\question (1 pt) D{\'e}clarer le type pour repr{\'e}sentant l'{\'e}tat des
cellules.

\question (1 pt) D{\'e}clarer un type pour repr{\'e}senter les tableaux dans
lesquels se jouent le jeu de la vie.

\question (1 pt) {\'E}crire un fonction prenant en entr{\'e}e deux entiers $n$
et $m$, et rendant un tableau pour le jeu de la vie dans lequel toutes
les cellules sont mortes.

\question (1 pt) {\'E}crire une fonction donnant le nombre de voisines
vivantes d'une cellule dans un tableau.


\question (3 pts) {\'E}crire une fonction prenant en entr{\'e}e un tableau
(ainsi qu'{\'e}ventuellement d'autres arguments) repr{\'e}sentant un {\'e}tat
courant, et rendant un tableau contenant l'{\'e}tat suivant.

\paragraph{Exemple.} Dans la suite d'{\'e}tats suivante, une cellule morte
est vide, une vivante est repr{\'e}sent{\'e}e par un V.
\def\morte{\phantom{V}} \def\vivante{V}
\begin{center}
\begin{tabular}[c]{|c|c|c|c|c|}
    \hline 
    \morte{}&\morte{}&\morte{}&\morte{}&\morte{}\\
    \hline 
    \morte{}&\vivante{}&\morte{}&\morte{}&\morte{}\\
    \hline 
    \morte{}&\vivante{}&\vivante{}&\vivante{}&\morte{}\\
    \hline 
    \morte{}&\morte{}&\morte{}&\vivante{}&\morte{}\\
    \hline 
    \morte{}&\morte{}&\morte{}&\morte{}&\morte{}\\
    \hline 
  \end{tabular} $\Rightarrow$ \begin{tabular}{|c|c|c|c|c|}
    \hline 
    \morte{}&\morte{}&\morte{}&\morte{}&\morte{}\\
    \hline 
    \morte{}&\vivante{}&\morte{}&\morte{}&\morte{}\\
    \hline 
    \morte{}&\vivante{}&\morte{}&\vivante{}&\morte{}\\
    \hline 
    \morte{}&\morte{}&\morte{}&\vivante{}&\morte{}\\
    \hline 
    \morte{}&\morte{}&\morte{}&\morte{}&\morte{}\\
    \hline 
  \end{tabular}
$\Rightarrow$
\begin{tabular}{|c|c|c|c|c|}
    \hline 
    \morte{}&\morte{}&\morte{}&\morte{}&\morte{}\\
    \hline 
    \morte{}&\morte{}&\vivante{}&\morte{}&\morte{}\\
    \hline 
    \morte{}&\morte{}&\morte{}&\morte{}&\morte{}\\
    \hline 
    \morte{}&\morte{}&\vivante{}&\morte{}&\morte{}\\
    \hline 
    \morte{}&\morte{}&\morte{}&\morte{}&\morte{}\\
    \hline 
  \end{tabular}
$\Rightarrow$
\begin{tabular}{|c|c|c|c|c|}
    \hline 
    \morte{}&\morte{}&\morte{}&\morte{}&\morte{}\\
    \hline 
    \morte{}&\morte{}&\morte{}&\morte{}&\morte{}\\
    \hline 
    \morte{}&\morte{}&\morte{}&\morte{}&\morte{}\\
    \hline 
    \morte{}&\morte{}&\morte{}&\morte{}&\morte{}\\
    \hline 
    \morte{}&\morte{}&\morte{}&\morte{}&\morte{}\\
    \hline 
  \end{tabular}

\end{center}

