\exo{Opérations sur les tableaux (7 pts)}

\utilisation{partiel}{pcp}{2016}

Dans le reste de cet exercice, on considère des tableaux de \cfloat. 

\question (1 pt) Écrire une fonction \cfun{val}(\cvar t,\cvar n, \cvar
i) qui rend la i-ème valeur du tableau \texttt{t} si $0\le i< n$,
$t[0]$ si $i<0$, et $t[n-1]$ si $i\ge n$.

\question (2 pts) On cherche à lisser un tableau de données en
rempla\c cant chaque élément par la moyenne arithmétique des $k$
éléments précédents du tableau (en considérant que $t[i] == t[0]$ pour
tout $i\le 0$). Écrire une fonction qui prend en entrée un tableau de
\cfloat, sa taille \cvar n, et un paramètre de lissage \cvar k, et qui
rend un nouveau tableau $T$ contenant dans chaque case $i$ la valeur
$T[i]=\frac 1k \times\Sigma_{j=i-k+1}^i t[j]$.

Pour ne pas donner trop de poids aux anciennes valeurs, on peut aussi
faire un \emph{lissage exponentiel}, et définir le tableau $T$ lissé en
fonction du non-lissé par:
$$
T[i]=\Sigma_{j=0}^{+\infty} t[i-j] \times \frac 1{2^{j+1}}
$$ 
en reprenant la convention $t[i] == t[0]$ pour tout $i\le 0$.  On
rappelle que $\Sigma_{j=0}^{+\infty} \frac 1{2^{j+1}} = 1$. Les deux
premières questions sont mathématiques et 

\question (1 pt) Que vaut $T[0]$ ?

\question (1 pt) Montrez que \(T[i+1] = \frac 12 \cdot ( T[i] + t[i] )\)

\question (2 pts) Écrire une fonction prenant en entrée un tableau $t$
de taille $n$ et renvoyant le tableau $T$ contenant son lissage
exponentiel.
