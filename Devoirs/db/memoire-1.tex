\exo{Test de compréhension (4 points)}

\begin{center}
  \fbox{~\parbox{0.9\textwidth}{%
      Comme cela n'a pas été beaucoup vu en TP, on rappelle que si \(x\) est une valeur, alors:
      \begin{center}
        \texttt{( t ) x}
      \end{center}
      est la traduction de cette valeur dans le type \texttt{t}.
    }~}
\end{center}

On considère la fonction \texttt{main} suivante:

\begin{lstlisting}[language=C]
int main ( int argc , char * argv[] )
{
  int t[3] = {1,2,3} ;
  int * p1 = ( int * ) ( t + 1 ) ;
  int * p2 = ( int * ) ( & t + 1 ) - 1 ;
  * ( t + 2 ) = p1 [ 1 ] - t [ 0 ] ;
  p2[-2] = p1[1] ;
  printf ( "%d %d %d\n" , t[0] , t[1] , t[2] ) ;
  return 0 ;
}
\end{lstlisting}

\question (2 points) Expliquez, en vous aidant si possible de schémas,
quelles sont les valeurs, en tant qu'entier, de \(p_1\) et \(p_2\) en
fonction de la valeur de \(t\) en tant qu'entier. Pour simplifier, on
suppose que \texttt{sizeof ( int ) = 4}.
\begin{solution}
  Il est nécessaire (et suffisant) de faire un schéma de la mémoire pour s'y retrouver !
  Au début, on a:
  \begin{center}
    \begin{tabular}{|c|c|c|c|c|}
      \multicolumn 1l{100}&\multicolumn 1l{104}&\multicolumn 1l{108}&\multicolumn 1l{112}&\multicolumn 1l{116}\\
      \hline
      1&2&3&& \\
      \hline
      \multicolumn 1c{t[0]}&\multicolumn 1c{t[1]}&\multicolumn 1l{t[2]}&\multicolumn 1l{p1}&\multicolumn 1l{p2}\\
      \multicolumn 3c{\upbracefill}&\multicolumn 2c{}\\
      \multicolumn 3c{t}&\multicolumn 2c{}\\
    \end{tabular}
  \end{center}
  Pour les additions et soustractions sur les adresses en tant
  qu'entier, si \(x\) est l'adresse d'une valeur de type \texttt{t}, alors
  en tant qu'entier, \(x+1\) vaut \texttt{x + sizeof ( t )}.
  \begin{enumerate}
  \item \texttt{t} est l'adresse de \texttt{t[0]}, une valeur de type
    \texttt{int}, donc \texttt{t+1} vaut en tant qu'entier \texttt{t +
      sizeof ( int )}, soit \texttt{104} sur le schéma (c'est défini
    ainsi pour que \texttt{t+1} soit l'adresse de \texttt{t[1]});
  \item \texttt{\& t} est l'adresse de \texttt{t}, une
    valeur qui est un tableau de 3 entiers, donc
    \texttt{\& t+1} vaut en tant qu'entier \texttt{t + 3 *
      sizeof ( int )}, soit \texttt{112} sur le schéma. On traduit
    ensuite cette valeur en tant qu'adresse d'un entier, et on enlève 1, donc en tant qu'entier on obtient l'adresse \texttt{112 - sizeof ( int )=108}.
  \end{enumerate}
  Donc après l'initialisation, on a:
  \begin{center}
    \begin{tabular}{|c|c|c|c|c|}
      \multicolumn 1l{100}&\multicolumn 1l{104}&\multicolumn 1l{108}&\multicolumn 1l{112}&\multicolumn 1l{116}\\
      \hline
      1&2&3&104&108 \\
      \hline
      \multicolumn 1c{t[0]}&\multicolumn 1c{t[1]}&\multicolumn 1l{t[2]}&\multicolumn 1l{p1}&\multicolumn 1l{p2}\\
      \multicolumn 3c{\upbracefill}&\multicolumn 2c{}\\
      \multicolumn 3c{t}&\multicolumn 2c{}\\
    \end{tabular}
  \end{center}
  On passe ensuite à l'évaluation des 2 instructions:
  \begin{itemize}
  \item \texttt{t[2] = * ( p1 + 1 ) - t[0]}. D'après ce qui précède,
    \texttt{p1+1} est l'adresse de \texttt{t[2]}, donc on pourrait
    écrire de manière plus simple: \texttt{t[2] = t[2] - t[0]=2};
  \item \texttt{p2[-2] = t[2]}. D'après ce qui précède, \texttt{p2-2}
    est l'adresse d'une case contenant un entier 2 cases contenant des
    entiers avant \texttt{t[2]}, donc on pourrait écrire de manière
    plus simple: \texttt{t[0] = t[2] =2};
  \end{itemize}
  Donc avant le \texttt{printf}, le contenu de la mémoire est:
  \begin{center}
    \begin{tabular}{|c|c|c|c|c|}
      \multicolumn 1l{100}&\multicolumn 1l{104}&\multicolumn 1l{108}&\multicolumn 1l{112}&\multicolumn 1l{116}\\
      \hline
      2&2&2&104&108 \\
      \hline
      \multicolumn 1c{t[0]}&\multicolumn 1c{t[1]}&\multicolumn 1l{t[2]}&\multicolumn 1l{p1}&\multicolumn 1l{p2}\\
      \multicolumn 3c{\upbracefill}&\multicolumn 2c{}\\
      \multicolumn 3c{t}&\multicolumn 2c{}\\
    \end{tabular}
  \end{center}  
  Le programme affiche \texttt{2 2 2}.

   
\end{solution}
