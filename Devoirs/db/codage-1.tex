\exo{Codage (4 points)}


\question Traduire les entiers 13 et -13 sur 5 bits en complément à deux.

\begin{solution}
  \begin{enumerate}
  \item On a \(2^5=32\), donc on peut coder les entiers de \(-16\) à
    \(+15\), et donc \(-13\) et \(13\);
  \item En complément à deux sur \(n\) bits, les entiers sont codés
    modulo \(2^n\), donc ici modulo \(32\)
  \item Comme entiers positifs, il faut donc trouver le codage de
    \(13\) et \(-13+32=19\).
  \item \(13=8+4+1=(1101)_2=01101\) sur 5 bits;
  \item \(19=16+2+1=(10011)_2=10011\) sur 5bits.
  \end{enumerate}

\end{solution}


\question Traduire si possible les entiers \(-8\), \(-7\), \(7\) et \(8\) sur 4
bits en complément à deux.

\begin{solution}
  \begin{enumerate}
  \item On a \(2^4=16\), donc on peut coder les entiers de \(-8\) à
    \(+7\), et donc seulement \(-8\), \(-7\) et \(7\);
  \item En complément à deux sur \(n\) bits, les entiers sont codés
    modulo \(2^n\), donc ici modulo \(16\)
  \item Comme entiers positifs, il faut donc trouver le codage de
    \(7\) et \(-8+16=8\) et \(-7+16 = 9\).
  \item \(7=4+2+1=(111)_2=0111\) sur 4 bits;
  \item \(8=8=(1000)_2=1000\) sur 4 bits;
  \item \(8=8+1=(1001)_2=1001\) sur 4 bits;
  \end{enumerate}

\end{solution}