\exo{Points dans le plan}

Le but de cet exercice est d'impl{\'e}menter des fonctions permettant de
travailler avec des points dans un plan (un espace de dimension 2).
Cette premi{\`e}re s{\'e}rie de questions porte sur les op{\'e}rations de base sur
les points.

\question (1 pt) D{\'e}clarez une structure \ctype{Point\_s}, ainsi que le
type \ctype{Point} des pointeurs vers ces structures.


\question (1 pt) {\'E}crivez une fonction \cfun{point} cr{\'e}ant une nouvelle
valeur de type \ctype{Point} {\`a} partir de ses arguments.

\question (0,5 pt) {\'E}crivez une fonction \cfun{distance} calculant la
distance habituelle entre deux points. Pr{\'e}cisez la biblioth{\`e}que {\`a}
inclure, et les options de compilation n{\'e}cessaires.


\question (1 pt) {\'E}crivez deux fonction \cfun{ord\_compare} et
\cfun{abs\_compare} comparant deux points suivant respectivement leur
ordonn{\'e}e et leur abscisse. Ces fonctions rendent un entier n{\'e}gatif,
nul, ou positif si le premier point a une ordonn{\'e}e (resp. une
abscisse) inf{\'e}rieure, {\'e}gale, ou sup{\'e}rieure {\`a} celle du second.


On va travailler maintenant sur les tableaux de points.

\question (0,5 pt) D{\'e}clarer un type \ctype{Points} pour les tableaux de points.


\question (3 pts) {\'E}crire deux fonctions \cfun{lire\_points} et
\cfun{ecrire\_points} permettant respectivement de lire et d'{\'e}crire un
tableau de points dans un fichier (repr{\'e}sent{\'e} par une valeur de type
\ctype{FILE *} pass{\'e}e en argument). Il est demand{\'e} que la fonction de
lecture puisse lire un fichier {\'e}crit par la fonction d'{\'e}criture.

\question (4 pts) {\'E}crire deux fonctions permettant de trier un tableau
de points en fonction respectivement de leur abscisse et de leur
ordonn{\'e}e. Il est conseill{\'e}, mais non requis, d'{\'e}crire des fonctions
interm{\'e}diaires.
