
\def\modulename{pcp}
\RequirePackage[utf8]{course}
\pcp
\newif\ifsolution
\solutiontrue
\usepackage{listings}

\lstset{language=C}

\usepackage{caption}
\usepackage{subcaption}
\usepackage{units}
\usepackage{Cdefs}
\usepackage{tikz}
\usetikzlibrary{arrows}

\tikzset{
  treenode/.style = {align=center, inner sep=0pt, text centered,
    font=\sffamily},
  arn_n/.style = {treenode, circle, white, font=\sffamily\bfseries, draw=black,
    fill=black, text width=1.5em},% arbre rouge noir, noeud noir
  arn_r/.style = {treenode, circle, red, draw=red, 
    text width=1.5em, very thick},% arbre rouge noir, noeud rouge
  arn_x/.style = {treenode, rectangle, draw=black,
    minimum width=0.5em, minimum height=0.5em}% arbre rouge noir, nil
}

\def\consigne#1{\null\begin{center}\parbox{13cm}{\large #1}\end{center}\relax}

\def\Question#1#2{\question{}\textbf{(#2)} #1}
\ifsolution
\includecomment{solution}
\else
\excludecomment{solution}
\fi

\begin{document}
\controleterminal

\consigne{{\bf Dur{\'e}e: 2h. Aucun document n'est autorisé.}}

\begin{quotation}
  \em 
  \begin{itemize}
  \item Il n'est pas nécessaire de mettre les fichiers \texttt{.h} à inclure;
  \item Dans les énoncés, \texttt{tableau} signifie juste l'adresse du
    début d'une partie de la mémoire contenant un tableau.
  \end{itemize}
\end{quotation}


\vspace*{1em}

\exo{Conversions (4 pts)}

\question (2 pts) Convertir \(44\) en base \(2\) et en base \(16\).

\question (2 pts) Convertir exactement \(44, 125\) en base \(2\) et
en base \(16\).

\exo{Élections (10 points)}

On veut pouvoir modéliser les résultats des élections pour un
parlement. Pour ces élections, il y a plusieurs états, plusieurs
partis, et chaque état désigne un nombre de députés pour chaque
parti. Le résultat d'une telle élection est modélisé par une structure
contenant:
\begin{itemize}
\item un nombre \(n\) d'états;
\item un nombre \(p\) de partis;
\item un tableau \(t\) à deux dimensions de taille \(n\times p\)
  contenant des entiers, qui contient le nombre de députés par état et
  par parti.
  \begin{center}
    \em Par exemple \(t[2][1]\) est le nombre de députés dans l'état
    d'indice \(2\) du parti d'indice \(1\).
  \end{center}
\end{itemize}

\question (1 pt) Faire un schéma décrivant le contenu de la mémoire 
lorsqu'il y a 3 états et 2 partis.

\question (1 pt) Déclarez une structure \texttt{Scrutin\_s} qui peut
représenter le résultat d'une élection, et le type \texttt{Scrutin} des adresses de
ces structures.

\question (0,5 pts) Pour une conception modulaire, indiquez dans quels
fichiers (\texttt{.h} ou \texttt{.c}) vous mettriez les déclarations
de la question précédente.

\question (1 pt) Écrivez une fonction \texttt{nb\_etats} qui prend
en argument une valeur de type \texttt{Scrutin} et qui renvoie le
nombre d'états.

\vspace*{1ex} {\bf Dans la suite, utilisez toujours la fonction
  \texttt{nb\_etats} au lieu d'utiliser directement la valeur stockée
  dans la structure.}

\question (1,5 pts) Écrivez une fonction \texttt{nouveau\_scrutin} qui
prend en argument le nombre \(n\) d'états et le nombre \(p\) de
partis, et qui renvoie une valeur de type \texttt{Scrutin}.

\question (2 pts) Écrivez une fonction \texttt{affiche\_scrutin} qui
prend en argument une valeur de type \texttt{Scrutin} et qui l'affiche.

\question (1,5 pts) Écrivez une fonction \texttt{entree\_resultat} qui
prend en argument une valeur de type \texttt{Scrutin}, un entier \(k\)
et un tableau d'entiers, et qui copie le contenu de ce tableau dans la
ligne correspondant à l'état d'indice \(k\). Cette fonction renvoie
\(0\) si \(k\) est l'indice d'un état, et renvoie \(1\) sinon.

\question (1,5 pts) Écrivez une fonction prenant en argument une
valeur de type \texttt{Scrutin}, un numéro \(l\) de parti, et
l'adresse \(a\) d'un entier.  Cette fonction renvoie \(1\) si l'entier
n'est pas le numéro d'un parti. Sinon, elle stocke à l'adresse \(a\)
le nombre total d'élus de ce parti et renvoie \(0\).


\exo{Union (6 pts)}

\textit{Cet exercice est la suite du précédent, mais est indépendant.}

On veut maintenant modéliser les types \texttt{Etat} et \texttt{Union}
de la manière suivante:
\begin{itemize}
\item un Etat est l'adresse d'une structure contenant un nom (une
  chaîne de caractères) et le nombre de députés de cet état;
\item une \texttt{Union}\footnote{la majuscule est importante pour ne
    pas la confondre avec le type \texttt{union}} est l'adresse d'une
  structure contenant un nombre \(n\) états, et un tableau de \(n\)
  valeurs de type \texttt{Etat}.
\end{itemize}

\question (1 pt) Faire un schéma de la mémoire pour une Union ayant 2
états, "sud" et "ouest", avec 10 députés chacun.

\question (1 pt) Quel est le type d'une chaîne de caractères ? D'un
tableau contenant des chaînes de caractères ?

\question (1 pt) Écrivez une fonction qui en argument un nom et un
nombre de député, qui crée et met à jour une structure correspondante,
et qui renvoie l'adresse de la structure créée.

\question (1 pt) Écrivez une fonction qui en argument un nombre
d'états, un tableau de noms d'états, un tableau de nombres de député,
qui crée et met à jour une structure modélisant l'union de ces états,
et renvoie son adresse.

\question (2 pts) Expliquez comment vous pouvez modifier vos réponses
à l'exercice précédent pour qu'une structure \texttt{Scrutin\_s}
contienne l'adresse d'une Union au lieu du nombre \(n\) d'états.





\end{document}

