\titre{\ttableaux}




\exo{Crible d'Érathostène Simple}
% 1
\prerequis \tio 6

L'algorithme du crible d'Érathostène est utilisé pour calculer en une
fois tous les nombres premiers jusqu'à une certaine limite. On veut
calculer un tableau \texttt{t} d'entiers tel que \texttt{t[i] = 1} si,
et seulement si, \texttt{i} est un nombre premier. Tout le travail se
fait dans le fichier \texttt{exercice\arabic{exercicenum}.c} à partir
du programme de base.

\question On commence par fixer la taille maximale qu'on accepte pour
tester des nombres premiers. Ajouter en haut du fichier, après
l'inclusion de \texttt{stdio.h}:
\begin{lstlisting}[language=C]
#define TAILLE 100000
\end{lstlisting}

\question On va écrire une fonction \texttt{crible} calculant le crible. Pour celà:
\begin{itemize}
\item il faut déclarer dans la fonction \texttt{main} un tableau
  d'entiers \emph{t} de \texttt{TAILLE} cases;
\item appeler la fonction \texttt{crible} avec l'adresse de ce tableau
  comme paramètre.
\end{itemize}
Commencez par écrire une fonction \texttt{crible} qui ne fait rien,
mais est appelée avec les bons arguments, et vérifiez que le programme
compile correctement.

\begin{solutioncachee}
  \begin{lstlisting}[language=C]

#include <stdio.h>

#define TAILLE 100000

void crible ( int * t )
{
}


int
main ( int argc , char * argv[] )
{
    int n ;
    int t[TAILLE] ;
    if ( crible ( t , TAILLE ) != 0 )
    {
        printf ( "Il y a eu une erreur lors du calcul.\n" ) ;
        return 1 ;
    }
    while ( 1 )
    {
        printf ( "Donnez un entier strictement inférieur à %d:\n" , TAILLE ) ;
        scanf ( "%d" , & n ) ;
        if ( n < 0 )
            break ;
        if ( n >= TAILLE )
            continue ;
        if ( t[n] == 1 )
            printf ( "%d est un nombre premier.\n" , n ) ;
        else
            printf ( "%d n'est pas un nombre premier.\n" , n ) ;
    }
    return 0 ;
} 
  \end{lstlisting}
\end{solutioncachee}

\question Lorsqu'on programme en C, il est recommandé de suivre la
recommandation suivante:
\begin{itemize}
\item si une fonction n'a pas eu d'erreur, elle renvoie l'entier 0;
\item sinon, elle renvoie un entier (le code d'erreur) qui indique
  quelle a été l'erreur.
\end{itemize}
Modifiez le programme pour qu'en cas d'erreur de la fonction
\texttt{crible}, la fonction \texttt{main} s'arrête immédiatement en
renvoyant un entier \(> 0\).
\begin{solutioncachee}
  \begin{lstlisting}[language=C]
    if ( crible ( t  ) != 0 )
    {
        printf ( "Il y a eu une erreur lors du calcul.\n" ) ;
        return 1 ;
    }
  \end{lstlisting}
\end{solutioncachee}

\question Dans la fonction \texttt{crible}, commencez par mettre les
deux premières cases à \(0\) (car \(0\) et \(1\) ne sont pas premiers)
et toutes les autres à \(1\). Compilez et vérifiez que le programme
s'exécute correctement, même s'il ne fait rien pour l'instant.

\question L'algorithme du crible d'Érathostène contient 2 boucles imbriquées:
\begin{itemize}
\item La première parcourt toutes les cases du tableau. Pour chaque case \(i\):
  \begin{itemize}
  \item Si \(t[i]=0\), passer à la case suivante;
  \item Si \(t[i]=1\), alors \(i\) est un nombre premier, et:
    \begin{itemize}
    \item Pour tous les \(j\ge 2\) tels que \(i*j\le \text{\tt TAILLE}\), mettre
      la case \(i*j\) à 0 (car \(i*j\) est un multiple de \(i\)).
    \end{itemize}
  \end{itemize}
\end{itemize}

\question Modifiez la boucle externe pour ne pas parcourir toutes les
cases jusqu'à \texttt{TAILLE} et vous arrêter avant. 

\begin{solutioncachee}
  \begin{lstlisting}[language=C]
int crible ( int * t )
{
    int i , j ;
    if ( TAILLE < 2 )
        return 1 ;
    t[0] = t[1] = 0 ;
    for ( i = 2 ; i < TAILLE ; i++ )
        t[i] = 1 ;
    for ( i = 2 ; i * i < TAILLE ; i++ )
        if ( t[i] == 1 )
            for ( j = 2 ; i * j < n ; j++ )
                t[i*j] = 0
    return  0 ;
}    
  \end{lstlisting}
\end{solutioncachee}

\question Dans la fonction \texttt{main}, faire une boucle infinie
(dont la condition est toujours vraie) demandant à l'utilisateur un entier
et faisant:
\begin{itemize}
\item Si l'entier est strictement négatif, on sort de la boucle avec
  \texttt{break} (c'est le seul moyen de sortir de cette boucle
  infinie);
\item Si l'entier est plus grand ou égal à la taille, on ne peut pas
  faire de calcul, donc on passe à l'itération suivante avec \texttt{continue};
\item Sinon, il faut afficher si l'entier \(i\) qui a été lu est
  premier en regardant la valeur de \(t[i]\).

\end{itemize}

\begin{solutioncachee}
  \begin{lstlisting}[language=C]
int
main ( int argc , char * argv[] )
{
    int n ;
    int t[TAILLE] ;
    if ( crible ( t , TAILLE ) != 0 )
    {
        printf ( "Il y a eu une erreur lors du calcul.\n" ) ;
        return 1 ;
    }
    while ( 1 )
    {
        printf ( "Donnez un entier strictement inférieur à %d:\n" , TAILLE ) ;
        scanf ( "%d" , & n ) ;
        if ( n < 0 )
            break ;
        if ( n >= TAILLE )
            continue ;
        if ( t[n] == 1 )
            printf ( "%d est un nombre premier.\n" , n ) ;
        else
            printf ( "%d n'est pas un nombre premier.\n" , n ) ;
    }
    return 0 ;
}    
  \end{lstlisting}
\end{solutioncachee}

\exo{Crible d'Érathostène utile}
% 2

Commencez par copier dans \texttt{exercice\arabic{exercicenum}.c} le
programme de l'exercice précédent.

\question Modifiez la fonction \texttt{crible} pour que la case \(t[i]\) contienne:
\begin{itemize}
\item \(i\) si \(i\) est un nombre premier;
\item \(j\) qui est le plus grand diviseur premier de \(i\) sinon.
\end{itemize}
Pour \(i=1\), on prend \(t[1]=1\).


\begin{solutioncachee}
  \begin{lstlisting}[language=C]
int crible ( int * t )
{
    int i , j ;
    t[0] = 0 ;
    t[1] = 1 ;
    for ( i = 2 ; i < TAILLE ; i++ )
        t[i] = i ;
    for ( i = 2 ; i * i < TAILLE ; i++ )
        if ( t[i] == i )
            for ( j = 2 ; i * j < TAILLE ; j++ )
                t[i*j] = i ;
    return  0 ;
}
  \end{lstlisting}
\end{solutioncachee}

\question Écrivez une fonction qui prend en entrée le crible et un
entier, et affiche la décomposition en facteurs premiers de cet
entier.


\begin{solutioncachee}
  \begin{lstlisting}[language=C]
int
decomposition ( int * t , int i )
{
    int j ;
    if ( ( i < 1 ) || ( i >= TAILLE ) )
    {
        printf ( "Pas de décomposition possible pour %d.\n" , i ) ;
        return 1 ;
    }
    for ( j = i ; t[j] != j ; j = j / t[j] )
        printf ( "%d " , t[j] ) ;
    printf ( "%d\n" , t[j] ) ;
    return 0 ;
}
  \end{lstlisting}
\end{solutioncachee}

\question Modifiez la fonction \texttt{main} pour qu'elle affiche la
décomposition en facteurs premiers des entiers que donne
l'utilisateur.

\begin{solutioncachee}
  \begin{lstlisting}[language=C]
int
main ( int argc , char * argv[] )
{
    int n ;
    int t[TAILLE] ;
    if ( crible ( t ) != 0 )
    {
        printf ( "Il y a eu une erreur lors du calcul.\n" ) ;
        return 1 ;
    }
    while ( 1 )
    {
        printf ( "Donnez un entier strictement inférieur à %d:\n" , TAILLE ) ;
        scanf ( "%d" , & n ) ;
        if ( n < 0 )
            break ;
        if ( n >= TAILLE )
            continue ;
        printf ( "La décomposition de %d en facteurs premiers est:\n\t" , n ) ;
        decomposition ( t , n ) ;
    }
    return 0 ;
}
  \end{lstlisting}
\end{solutioncachee}

\exo{Recherche de majorité}

Le but de cet exercice est de chercher dans un tableau d'entiers
aléatoires si une des valeurs apparaît dans plus de la moitié des
cases.

\begin{fminipage}{0.9\textwidth}
  \textbf{Obtenir un entier aléatoire.} Pour obtenir un entier
  suffisament aléatoire, on va utiliser un \emph{générateur de nombres
    pseudo-aléatoires}. Ce générateur doit être \emph{initialisé} par
  un entier différent à chaque lancement du programme, sinon ce seront
  toujours les mêmes nombres qui seront choisis. Pour celà, on
  initialise le GNPA avec l'heure actuelle:
  \begin{lstlisting}[language=C]
    srand ( time ( NULL ) ) ;
  \end{lstlisting}
  Il est nécessaire d'inclure les bibliothèques \texttt{time.h} et
  \texttt{stdlib.h}. Ensuite, chaque fois qu'on a besoin d'un entier
  aléatoire, on appelle la fonction \texttt{rand}.
\end{fminipage}

On fixe une taille de tableau à 10 éléments.


\question Écrire un programme qui demande à l'utilisateur un entier
positif \(n\), et initialise un tableau \texttt{t} avec des entiers
entre \(0\) et \(n-1\).

\begin{solutioncachee}
  \begin{lstlisting}[language=C]
#include <stdio.h>
#include <stdlib.h>
#include <time.h>

#define TAILLE 10

void initialise ( int * t , int modulo )
{
    int i ;
    for ( i = 0 ; i < TAILLE ; i++ )
        t[i] = rand ( ) % modulo ;
}
void lit_entier_positif ( int * n )
{
    int lu , c ;
    do
    {
        printf ( "Donnez un entier positif:\n" ) ;
        lu = scanf ( "%d" , n ) ;
        do
        {
            c = getchar ( ) ;
        } while ( c != '\n') ;
    } while ( ( lu != 1 ) || ( *n <= 0 ) ) ;
    
}
int 
main ( int argc , char * argv[] )
{
    int n ;
    int t[TAILLE] ;
    srand ( time ( NULL ) ) ;
    printf ( "Choisissez le nombre de valeurs différentes possibles:\n" ) ;
    lit_entier_positif ( & n ) ;
    initialise ( t , n ) ;
    return 0 ;
}

  \end{lstlisting}
\end{solutioncachee}

La recherche d'un candidat majoritaire se fait en 2 phases:
\begin{enumerate}
\item d'abord, on parcourt une fois toutes les cases du tableau pour trouver
  un candidat possible. Pour celà:
  \begin{itemize}
  \item si la majorité actuelle est de 0, l'élément courant devient candidat avec une majorité de 1;
  \item si la majorité actuelle est strictement plus grande que 0:
    \begin{itemize}
    \item si l'élément courant est égal au candidat courant, on augmente la majorité de 1 ;
    \item sinon, on diminue la majorité de 1
    \end{itemize}
  \end{itemize}
  Cet algorithme garantit que \textbf{s'il existe un élément majoritaire}, alors le candidat est cet élément;
\item Ensuite, on parcourt le tableau une deuxième fois, et on compte
  le nombre de fois où le candidat apparaît.
  \begin{itemize}
  \item Si ce nombre est strictement supérieur à \texttt{TAILLE / 2}
    alors on a trouvé l'élément majoritaire;
  \item Sinon, aucune valeur n'est majoritaire.
  \end{itemize}
\end{enumerate}

\question Écrire la fonction \texttt{trouve\_candidat} qui trouve
le candidat qui peut être majoritaire: 
\begin{lstlisting}[language=C]
void trouve_candidat ( int * t , int * candidat ) ;  
\end{lstlisting}
avec \texttt{t} l'adresse de la première case d'un tableau de
\texttt{TAILLE} entiers, et \texttt{candidat} l'adresse de la variable
dans laquelle il faut mettre la valeur trouvée.

\begin{solutioncachee}
  \begin{lstlisting}[language=C]
void trouve_candidat ( int * t , int * candidat )
{
    int i ;
    int majorite ;
    for ( i = 0 , majorite = 0 ; i < TAILLE ; i++ )
        if ( majorite == 0 )
        {
            majorite = 1 ;
            *candidat = t[i] ;
        }
        else
        {
            if ( t[i] == *candidat )
                majorite += 1 ;
            else
                majorite -= 1 ;
        }
}    
  \end{lstlisting}
\end{solutioncachee}

\question Écrire la fonction \texttt{verifie\_candidat} qui vérifie
que le candidat trouvé est majoritaire.
\begin{lstlisting}[language=C]
int verifie_candidat ( int * t , int candidat ) ;  
\end{lstlisting}

\begin{solutioncachee}
  \begin{lstlisting}[language=C]
int verifie_candidat ( int * t , int candidat )
{
    int i ;
    int occurrences ;
    for ( i = 0 , occurrences = 0 ; i < TAILLE ; i++ )
        occurrences += ( t[i] == candidat ) ;
    return ( occurrences > ( TAILLE / 2 ) ) ;
}    
  \end{lstlisting}
\end{solutioncachee}

\question Afin de vérifier le résultat, écrire une fonction qui
affiche le contenu du tableau d'entiers.

\begin{solutioncachee}
  \begin{lstlisting}[language=C]
int
affiche_tableau ( int * t )
{
    int i ;
    for ( i = 0 ; i < TAILLE ; i++ )
        printf ( "%d " , t[i] ) ;
    printf ( "\n" ) ;
    return 0 ;
}
  \end{lstlisting}
\end{solutioncachee}

\question Complétez la fonction \texttt{main} et vérifiez que votre programme marche. Il est conseillé de prendre de choisir \(n=2\) ou \(n=3\) pour avoir de bonnes chances d'avoir un élément majoritaire.

\begin{solutioncachee}
  \begin{lstlisting}[language=C]
int 
main ( int argc , char * argv[] )
{
    int n ;
    int t[TAILLE] ;
    int candidat ;
    srand ( time ( NULL ) ) ;
    printf ( "Choisissez le nombre de valeurs différentes possibles:\n" ) ;
    lit_entier_positif ( & n ) ;
    initialise ( t , n ) ;
    affiche_tableau( t ) ;
    trouve_candidat ( t , & candidat ) ;
    if ( verifie_candidat ( t , candidat ) )
    {
        printf ( "L'élément %d est majoritaire.\n" , candidat ) ;
    }
    else
    {
        printf ( "Il n'y a pas d'éléments majoritaires dans le tableau.\n") ;
    }
    return 0 ;
}
  \end{lstlisting}
\end{solutioncachee}

\exo{Demi-additionneur et additionneur}
% 4 

On va voir dans cet exercice comment additionner deux entiers
ayant un nombre arbitraire de chiffres. Le principe est simple:
\begin{enumerate}
\item Chaque entier en entrée est codé par un tableau de chiffres;
\item On calcule la solution chiffre par chiffre, en n'oubliant pas 
  la retenue.
\end{enumerate}

Pour le 1. on pourrait choisir des chiffres dans n'importe quelle base
(au fond, c'est l'ordinateur qui fait les additions), mais pour avoir
des exemples faciles à construire, on va utiliser la base 256, donc:
\begin{itemize}
\item Chaque chiffre est un \emph{unsigned char} entre \(0\) et \(255\);
\item Un entier est un tableau de \emph{unsigned char}.
\end{itemize}
Pour les exemples, on va fixer la taille de ces tableaux à 10, mais
n'importe quelle valeur est possible. Pour pouvoir parcourir les
tableaux à partir de l'indice 0, on suppose en plus que les chiffres
sont écrits dans l'ordre inversé (le chiffre des unités dans la case
0, puis celui des ``dizaines'', etc.)

\question Dans la fonction \emph{main}, définissez 3 tableaux de type
\emph{unsigned char}, et initialisez les 2 premiers. Vérifiez que le
code est correct en compilant et en exécutant le programme (il ne fait
rien).
\paragraph{Initialisation:} 
\begin{lstlisting}[language=C]
unsigned char
    t1[10] = { 56 , 125 , 234 , 12 , 124 , 0 } ,
    t2[10] = { 34 , 131 , 20 , 244 , 200 , 0 } ,
    t3[10] ;
\end{lstlisting}
\begin{solutioncachee}
  \begin{lstlisting}language=C]
#include <stdio.h>

int
main ( int argc , unsigned char * argv[] )
{
  unsigned char
    t1[10] = { 56 , 125 , 234 , 12 , 124 , 0 } ,
    t2[10] = { 34 , 131 , 20 , 244 , 200 , 0 } ,
      t3[10] ;
  return 0 ;
}
  \end{lstlisting}
\end{solutioncachee}

\question Écrire une fonction qui prend en entrée l'adresse d'un tableau d'\emph{unsigned char} et son nombre de cases, et qui affiche le contenu du tableau. Modifiez la fonction 
\emph{main} pour afficher les 3 tableaux.
\begin{solutioncachee}
  \begin{lstlisting}[language=C]
#include <stdio.h>

void print_tableau_c ( unsigned char * t , int n )
{
  int i ;
  printf ( "[ %3d " , t[0] ) ;
  for ( i = 1 ; i < n ; i ++ )
    printf ( ", %3d " , t[i] ) ;
  printf ( "]\n" ) ;
}

int
main ( int argc , unsigned char * argv[] )
{
  unsigned char
    t1[10] = { 56 , 125 , 234 , 12 , 124 , 0 } ,
    t2[10] = { 34 , 131 , 20 , 244 , 200 , 0 } ,
      t3[10] ;
      print_tableau_c ( t1 , 3 ) ;
      print_tableau_c ( t2 , 3 ) ;
      print_tableau_c ( t3 , 3 ) ;
  return 0 ;
}    
  \end{lstlisting}
\end{solutioncachee}

\question Écrire une fonction \emph{additionneur} qui prend en entrée:
\begin{itemize}
\item l'adresse de deux \emph{unsigned char}, \texttt{resultat} et
  \texttt{retenue}, à laquelle elle mettra les valeurs de la somme et
  de la retenue de la somme suivante;
\item 2 \emph{unsigned char} dont elle doit faire la somme avec la valeur, lors
  de l'appel de la fonction, contenue à l'adresse \texttt{retenue}.
\end{itemize}
Cette fonction ne renvoie pas de résultat.

\begin{solutioncachee}
  \begin{lstlisting}[language=C]
void additionneur (
		       unsigned char * retenue ,
		       unsigned char * resultat ,
		       unsigned char c1 ,
		       unsigned char c2 )
{
  int tmp ;
  tmp = c1 + c2 + * retenue ;
  *retenue = tmp / 256 ;
  *resultat = tmp % 256 ;
}    
  \end{lstlisting}
\end{solutioncachee}

\question Écrire une fonction \emph{addition\_tableau} qui prend en
entrée les adresses de 3 tableaux d'\emph{unsigned char} et leur
nombre de cases, et qui écrit dans le premier le résultat de
l'addition des 2 autres. Cette fonction renvoie 1 si la retenue finale
est 1 (c'est un cas d'erreur), et 0 sinon. Appliquez cette fonction
sur les tableaux de la fonction \emph{main}, et affichez le résultat.

\begin{solutioncachee}
  \begin{lstlisting}[language=C]
int addition_tableau (
		unsigned char * res ,
		unsigned char * t1 ,
		unsigned char * t2 ,
		int n )
{
  int i ;
  unsigned char retenue = 0 ;
  for ( i = 0 ; i < n ; i++ )
    additionneur ( & retenue , res + i , t1[i], t2[i] ) ;
  return retenue ;
}
  
  \end{lstlisting}
\end{solutioncachee}


\question \textbf{Attention, ce qui suit peut être perturbant !}\\
De la même manière que précédement, créez et initialisez 3 tableaux
d'entiers dans la fonction main, écrivez une fonction d'affichage des
tableaux d'entiers, et \emph{appelez la fonction addition\_tableau}
sur ces tableaux (attention au nombre de cases, cette fonction a
besoin du nombre de cases de type \emph{unsigned char}. Affichez le
résultat. Que remarquez-vous ? Est-ce normal ?

\begin{solutioncachee}
  Pour la solution complète, on utilise des demi-additionneurs, qui
  sont plus proches des circuits utilisés dans les processeurs (notez
  le \emph{ou logique} pour la retenue finale). De plus, on teste que
  toutes adresses sont possibles (la seule réellement impossible étant 
  l'adresse \texttt{NULL})
  \begin{lstlisting}[language=C]
#include <stdio.h>


int demi_additionneur (
		       unsigned char * retenue ,
		       unsigned char * resultat ,
		       unsigned char c1 ,
		       unsigned char c2 )
{
  int tmp ;
  if ( retenue == NULL )
    return 1 ;
  if ( resultat == NULL )
    return 1 ;
  * resultat = ( c1 + c2 ) % 256 ;
  * retenue = ( c1 + c2 ) / 256 ;
  return 0 ;
}

int additionneur (
		       unsigned char * retenue ,
		       unsigned char * resultat ,
		       unsigned char c1 ,
		       unsigned char c2 )
{
  unsigned char retenue1 , retenue2 ;
  if ( retenue == NULL )
    return 1 ;
  if ( resultat == NULL )
    return 1 ;
  if ( demi_additionneur ( & retenue1 , resultat , c1 , c2 ) )
    return 1 ;
  if ( demi_additionneur ( & retenue2 , resultat , * resultat , * retenue ) )
    return 1 ;
  * retenue = retenue1 || retenue2 ;
  return 0 ;
}



int addition_c (
		unsigned char * res ,
		unsigned char * t1 ,
		unsigned char * t2 ,
		int n )
{
  int i ;
  unsigned char retenue = 0 ;
  for ( i = 0 ; i < n ; i++ )
    if ( additionneur ( & retenue , res + i , t1[i], t2[i] ) )
      return 1 ;
  if ( retenue )
    return 2 ;
  return 0 ;
}

void print_tableau_c ( unsigned char * t , int n )
{
  int i ;
  printf ( "[ %3d " , t[0] ) ;
  for ( i = 1 ; i < n ; i ++ )
    printf ( ", %3d " , t[i] ) ;
  printf ( "]\n" ) ;
}
void print_tableau_i ( unsigned int * t , int n )
{
  int i ;
  printf ( "[ %3d " , t[0] ) ;
  for ( i = 1 ; i < n ; i ++ )
    printf ( ", %3d " , t[i] ) ;
  printf ( "]\n" ) ;
}

int calcule ( char * t3 , char * t1 , char * t2 , int n )
{
  int i ;
  print_tableau_c ( t1 , n ) ;
  print_tableau_c ( t2 , n ) ;
  switch ( addition_c ( t3 , t1 , t2 , n ) )
    {
    case 1:
      printf ( "Erreur lors de l'addition.\n" ) ;
      return 1 ;
    case 2:
      printf ( "Il reste une retenue (dépassement).\n" ) ;
      return 2 ;
    default:
      print_tableau_c ( t1 , n ) ;
      print_tableau_c ( t2 , n ) ;
      print_tableau_c ( t3 , n ) ;
    }
}

int
main ( int argc , unsigned char * argv[] )
{
  unsigned int
    i1[3] = { 12345 , 12345 , 0 } ,
    i2[3] = { 66666 , 44444 , 0 } ,
      i3[3] ,
	i ;
  unsigned char
    t1[10] = { 56 , 125 , 234 , 12 , 124 , 0 } ,
    t2[10] = { 34 , 131 , 20 , 244 , 200 , 0 } ,
      t3[10] ;
      calcule ( t3 , t1 , t2 , 10 ) ;
      calcule ( ( unsigned char * ) i3 , ( unsigned char * ) i1 , ( unsigned char * ) i2 , 3 * sizeof ( int ) ) ;
      print_tableau_i ( i1 , 3 ) ;
      print_tableau_i ( i2 , 3 ) ;
      print_tableau_i ( i3 , 3 ) ;
  return 0 ;
}    
  \end{lstlisting}
  Le résultat attendu est que l'addition est correcte pour les
  tableaux d'entiers. Quand on y pense; c'est très perturbant, car
  normalement, dans chaque case d'entier, c'est comme-ci on faisait
  l'addition de la gauche vers la droite (on avait inversé l'ordre des
  chiffres dans le codage en tableau).

  La raison est que les processeurs Intel ont évolué à partir
  d'ordinateurs (les processeurs 8086) qui ne faisaient d'additions
  que \emph{char} par \emph{char}. Par conservatisme, même maintenant,
  les entiers continuent d'être stockés dans l'ordre inverse, appelé
  en anglais \emph{Big Endian}: le dernier char dans la représentation
  est celui qui contient les chiffres les plus importants. À
  l'inverse, les tablettes et téléphones portables, qui utilisent des
  processeurs de type ARM, stockent les entiers dans l'ordre naturel
  (\emph{Little Endian}, les chiffres les moins importants sont à la
  fin de la représentation). 
\end{solutioncachee}

\exo{Tri lent}
% 5

Trier un tableau signifie ordonner ses éléments du plus petit au plus
grand. On va voir dans cet exercice et les suivants plusieurs algorithmes de
tri, en commençant par un algorithme simple:
On trie un tableau de \(n\) entiers en:
\begin{itemize}
\item Trouvant l'indice \(0\le i_{\text{max}}<n\) tel que la valeur
  \(t[i_{\text{max}}]\) soit maximale dans le tableau;
\item On échange cette valeur avec celle contenue dans la case d'indice \(n-1\);
\item On trie les \(n-1\) valeurs restantes.
\end{itemize}

\question Écrivez un programme initial contenant:
\begin{itemize}
\item une fonction qui affiche un tableau d'entiers (\textit{cf.} exercice précédent);
\item une fonction \emph{tri} qui ne fait rien pour l'instant, mais qui triera un tableau d'entiers de \(n\) cases
\item Dans la fonction \emph{main}:
  \begin{itemize}
  \item un tableau d'entiers initialisé (et avec des valeurs dans le
    désordre);
  \item un appel à la fonction d'affichage;
  \item un appel à la fonction de tri;
  \item un second appel à la fonction d'affichage.
  \end{itemize}
\end{itemize}

\begin{solutioncachee}
  \begin{lstlisting}[language=C]
#include <stdio.h>

void affichage ( int * t , int n )
{
  int i ;
  printf ( "[ %d " , t[0] ) ;
  for ( i = 1 ; i < n ; i++ )
    printf ( ", %d " , t[i] ) ;
  printf ( "]\n'' ) ;
}

void tri ( int * t , int n )
{
}

int main ( int argc , char * argv[] )
{
  int t[10] = {5,8,1,4,9,2,2,3,6,7} ;
  affiche ( t , 10 ) ;
  tri ( t , 10 ) ;
  affiche ( t , 10 ) ;
  return 0 ;
}
  \end{lstlisting}
\end{solutioncachee}

\question Écrivez une fonction \emph{echange} qui échange les valeurs
de deux cases de type \texttt{int} dont l'adresse est donnée en paramètre.

\begin{solutioncachee}
  \begin{lstlisting}[language=C]
void echange ( int * a , int * b )
{
  int c ;
  c = *a ;
  *a = *b ;
  *b = c ;
}
  \end{lstlisting}
\end{solutioncachee}

\question Écrivez une fonction qui calcule l'indice d'un élément
maximal d'un tableau de \(n\) entiers.

\begin{solutioncachee}
  \begin{lstlisting}[language=C]
void indice_max ( int * res , int * t , int n )
{
  int i ;
  *res = 0 ;
  for ( i = 1 ; i < n ; i++ )
    if ( t[*res] < t[i] )
      *res = i ;
}
  \end{lstlisting}
\end{solutioncachee}

\question Complétez la fonction \emph{tri} pour qu'elle trie
effectivement un tableau de \(n\) entiers.

\begin{solutioncachee}
  \begin{lstlisting}[language=C]
void tri ( int * t , int n )
{
  int i , i_max ;
  for ( i = n - 1 ; i > 0 ; i-- )
  {
    index_max ( & i_max , t , i ) ;
    echange ( t + i , t + i_max ) ;
  }
}
  \end{lstlisting}
\end{solutioncachee}

\question Écrivez une deuxième version qui calcule le minimum du
tableau, et l'échange avec le premier élément.

\begin{solutioncachee}
  \begin{lstlisting}[language=C]
void indice_min ( int * res , int * t , int n )
{
  int i ;
  *res = 0 ;
  for ( i = 1 ; i < n ; i++ )
    if ( t[*res] > t[i] )
      *res = i ;
}

void tri2 ( int * t , int n )
{
  int i_min ;
  for (  ; n > 0 ; t++ , n-- )
  {
    index_min ( & i_min , t , n ) ;
    echange ( t , t + i_min ) ;
  }
}
  \end{lstlisting}
\end{solutioncachee}

\exo{Tri rapide}

Le tri rapide est un tri en plusieurs étapes. Sur un tableau \(t\) de
\(n\) entiers:
\begin{enumerate}
\item une première fonction \texttt{choix\_pivot} choisit un indice
  pivot \(p\) et l'échange avec l'élément d'indice \(0\);
\item une fonction, qu'on appellera \texttt{drapeau\_hollandais}
  parcourt le tableau pour que, par échanges successifs, tous les
  éléments plus petits que le pivot soient placés avant dans le
  tableau, et tous les éléments plus grand soient placés après;
\item La procédure de tri est ensuite appelée récursivement sur les
  zones bleu (éléments strictement plus petits que le pivot) et rouge
  (éléments strictement plus grand que le pivot.
\end{enumerate}

\question Comme dans l'exercice précédent, initialisez et affichez un
tableau, et écrivez une fonction de tri qui ne fait rien.

\question On va aussi écrire une \texttt{choix\_pivot} qui, elle
aussi, ne fait rien. Les variantes de l'algorithme de tri rapide sont
basées sur des fonctions de choix plus élaborées, mais on codera
uniquement la version de base.

\question \textbf{Procédure de drapeau hollandais.} Dans cette
procédure, un tableau est divisé en 4 zones:
\begin{itemize}
\item Une zone bleu, qui commence toujours à l'indice 0, mais peut
  être vide, et qui contient les éléments strictement plus petits que
  le pivot;
\item Une zone blanche, qui commence juste après la zone bleu, et qui
  contient les éléments égaux au pivot. Au début, cette zone blanche
  contient juste le premier élément du tableau (le pivot);
\item Une zone rouge, qui part de la fin du tableau, et contient tous
  les éléments strictement plus grands que le pivot. Au départ, cette
  zone est vide;
\item une zone grise, qui contient au départ tous les éléments du
  tableau sauf le premier, et qui correspond aux éléments qui n'ont
  pas encore été comparés au pivot. Elle est entre la zone blanche et
  la zone grise;
\end{itemize}
Ces zones sont délimitées par 3 indices, \texttt{premier\_blanc},
\texttt{dernier\_blanc}, et \texttt{dernier\_gris}.  Ces indices
évoluent quand on compare le dernier élément blanc avec l'élément
suivant (\textit{cf.} schéma ou S2). À chaque itération la zone grise
doit diminuer d'un élément, et le parcourt du tableau est terminé
quand la zone grise est vide. La procédure doit alors indiquer le
premier et le dernier indice de la zone blanche.

\question La procédure de tri rapide récupère ensuite les indices de
début et de fin de la zone blanche (les éléments entre ces deux
indices sont correctement placés), et est appelée récursivement sur
les zones bleu et rouge si elles ont strictement plus qu'un élément.





% tri fusion et tri avec qsort dans le thème pointeurs