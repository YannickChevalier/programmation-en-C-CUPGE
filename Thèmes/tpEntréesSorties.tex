\titre{\tio}


\begin{center}
  \parbox{0.9\textwidth}{%
    \sl Dans ce thème, nous verrons comment un programme écrit en C
    interagit avec son environnement, que ce soit un terminal (la
    fenêtre dans laquelle il est exécuté) ou des fichiers.
  }
\end{center}

\exo{Premier programme}
% 1
On va travailler sur les types, donc dans le répertoire \url{~/workspace/programmation-en-C-CUPGE/src/Entrées-Sorties/}.

\question Vérifiez que vous êtes bien sur la branche
\texttt{master}, et si ce n'est pas le cas, allez sur cette branche.

\begin{solutioncachee}
  \begin{lstlisting}[language=bash]
    git branch
    git checkout etudiant
  \end{lstlisting}
\end{solutioncachee}

\question Créez le fichier \expandafter\url{~/workspace/programmation-en-C-CUPGE/src/Entrées-Sorties/exercice\arabic{exercicenum}.c} et
ouvrez-le.

\begin{fminipage}{\textwidth}
  Il ne faut jamais faire du copier-coller d'un fichier pdf vers un
  fichier texte: même si les caractères affichés se ressemblent, ils
  sont différents.
\end{fminipage}


\question Tapez le code suivant dans le fichier:
\begin{lstlisting}[language=C]
#include <stdio.h>

int
main ( int argc , char * argv[] )
{
  return 0 ;
}
\end{lstlisting}

\question Sauvegardez le fichier (avec Ctrl-S). 

\begin{center}
  \sl On va maintenant compiler et exécuter ce programme.
\end{center}

\question Changer de répertoire pour aller dans \url{~/workspace/src/Entrées-Sorties}.
\begin{solutioncachee}
  \begin{lstlisting}[language=bash]
    cd ~/workspace/src/Entrées-Sorties
  \end{lstlisting}
\end{solutioncachee}

\question Compiler le programme avec la commande:
\begin{solutioncachee}
  \begin{lstlisting}[language=bash]
    gcc -o exercice%\arabic{exercicenum}% exercice%\arabic{exercicenum}%.c
  \end{lstlisting}
\end{solutioncachee}

\question Exécutez le programme. Que se passe-t'il ?
\begin{solutioncachee}
  \begin{lstlisting}[language=bash]
    ./exercice%\arabic{exercicenum}%
  \end{lstlisting}
  Il ne se passe rien.
\end{solutioncachee}

\question Ajoutez la ligne suivante juste au dessus de \texttt{return 0}:
\begin{lstlisting}[language=C]
  printf ( "Bonjour." ) ;
\end{lstlisting}
et compilez et exécutez le programme.

\begin{center}
  \fbox{~\parbox{0.9\textwidth}{\texttt{printf} est une \emph{fonction} qui permet à un programme d'envoyer des données au terminal dans lequel il est exécuté.}~}
\end{center}

\question Ajoutez \verb+\n+ après Bonjour,   compilez et exécutez le programme.

\begin{center}
  \fbox{~\parbox{0.9\textwidth}{\texttt{\textbackslash n} est un caractère qui demande au terminal de passer à la ligne suivante.}~}
\end{center}

\exo{Affichage d'un entier.}
% 2

Le 'f' de \texttt{printf} signifie \emph{mettre en
  \texttt{f}orme}. Cette fonction est un peu spéciale car
contrairement aux fonctions normales en C, elle peut prendre un nombre
variable d'arguments, mais il en faut au moins 1. 

Le premier argument est une \emph{chaîne de formattage} qui sert à
mettre en forme les données qui sont contenues dans les autres
arguments, dans l'ordre de lecture. Pour celà, la chaîne de formattage
contient des \emph{directives d'affichage}. La première directive sera
utilisée pour afficher la première donnée après la chaîne de
formattage, la seconde directive sera utilisée pour afficher la
seconde donnée après la chaîne de formattage, etc.

\paragraph{Directives d'affichage.} Une directive commence toujours
par \%. Il y en a des dizaines, donc on va juste en regarder quelques unes:
\begin{itemize}
\item \texttt{\% d}: permet d'afficher un entier en base \texttt{d}ix;
\item \texttt{\% c}: permet d'afficher une lettre (un caractère) à partir d'un entier;
\item \texttt{\% f}: permet d'afficher un nombre à virgule;
\item \texttt{\% s}: permet d'afficher une chaîne de caractères;
\end{itemize}
Pour l'exercice 2, on commence par copier le contenu du programme
\texttt{exercice1.c} dans un nouveau fichier \texttt{exercice\arabic{exercicenum}.c}.


\question Modifiez le premier programme pour afficher sur 10 lignes
chaque lettre de 0 à 9 suivi de sa valeur en tant qu'entier.
\begin{solutioncachee}
  \begin{lstlisting}[language=C]
    #include <stdio.h>

    int main ( int argc , char * argv[] ) 
    {
      print ( "%c -> %d \n" , '0' , '0' ) ;
      print ( "%c -> %d \n" , '1' , '1' ) ;
      print ( "%c -> %d \n" , '2' , '2' ) ;
      print ( "%c -> %d \n" , '3' , '3' ) ;
      print ( "%c -> %d \n" , '4' , '4' ) ;
      print ( "%c -> %d \n" , '5' , '5' ) ;
      print ( "%c -> %d \n" , '6' , '6' ) ;
      print ( "%c -> %d \n" , '7' , '7' ) ;
      print ( "%c -> %d \n" , '8' , '8' ) ;
      print ( "%c -> %d \n" , '9' , '9' ) ;
      return 0 ; 
    }
  \end{lstlisting}
\end{solutioncachee}
Que remarque-t'on ?

\begin{solutioncachee}
  les codes se suivent dans le même ordre que les chiffres.
\end{solutioncachee}

\question Faites de même pour les lettres \texttt{'a'} et \texttt{'z'}, et affichez leur
différence \texttt{'z'} - \texttt{'a'}. Idem pour les majuscules.

\exo{Classes de caractères}
\prerequis\ttype 1
%3
Comme les caractères de même type (lettres minuscules, chiffres, etc.)
se suivent, on va écrire un programme qui va lire un caractère, et va
afficher son type (lorsqu'on le connaît).

\begin{solutioncachee}
  \begin{lstlisting}[language=C]
#include <stdio.h>

int 
main ( int argc , char * arv[] ) 
{ 
  int x ; 
  x = getchar ( ) ;
  if ( ( '0' <= x ) && ( x <= '9' ) ) 
  { 
    printf ( "%c est un chiffre.\n" , x ) ;
  } 
  if ( ( 'a' <= x ) && ( x <= 'z' ) ) 
  { 
    printf ( "%c est une lettre minuscule.\n" , x ) ;
  } 
  if ( ( 'A' <= x ) && ( x <= 'Z' ) ) 
  { 
    printf ( "%c est une lettre majuscule.\n" , x ) ;
  } 
  return 0 ; 
}
  \end{lstlisting}
\end{solutioncachee}


\exo{Chaînes de caractères}
% 4
\prerequis \ttype 3
Contrairement à Python, C fait la différence entre un caractère seul
et une suite de caractères.  Les valeurs de type caractère
(\texttt{char}) s'écrivent entre des apostrophes, comme \texttt{'a'},
\texttt{'3'}, \texttt{'\textbackslash n'}. Les \emph{chaînes de
  caractères} sont juste des tableaux de caractères. Pour indiquer la
fin de la chaîne, on met dans le tableau l'entier \(0\), qu'on peut
aussi écrire \texttt{'\textbackslash 0'}. Il s'agit d'une convention
qui est utilisée pour pouvoir traîté plus facilement les chaînes de
caractères:
\begin{itemize}
\item Il n'y a pas de type spécifique: une chaîne de caractères est
  définie par l'adresse du premier caractère, donc on utilise \texttt{char *};
\item Pour afficher une chaîne de caractères avec \texttt{printf}, on
  utilise la directive d'affichage \texttt{\%s}:
  \begin{exemple}
    Les deux morceaux suivants de programme font la même chose:
    \begin{lstlisting}[language=C]
      printf ( "Bonjour.\n" ) ;
    \end{lstlisting}
    \begin{lstlisting}[language=C]
      char * hello = "Bonjour" ;
      printf ( "%s.\n" , hello ) ;
    \end{lstlisting}
  \end{exemple}
\end{itemize}

\question Écrivez dans le fichier
\texttt{exercice\arabic{exercicenum}.c} un programme qui affiche les 3
premiers éléments du tableau \texttt{argv}. Compilez et exécutez ce
programme avec \emph{et sans} arguments. Que remarque-t'on ?

\begin{solutioncachee}
  On utilise le programme suivant:
  \begin{lstlisting}[language=C]
#include <stdio.h>

int main ( int argc , char * argv[] )
{
    char * hello = "Bonjour" ;
    printf ( "%s.\n" , hello ) ;
    printf ( "Bonjour.\n") ;
    printf ( "argument 0: %s\n" , argv[0] ) ;
    printf ( "argument 1: %s\n" , argv[1] ) ;
    printf ( "argument 2: %s\n" , argv[2] ) ;
    printf ( "argument 3: %s\n" , argv[3] ) ;
    printf ( "argument 4: %s\n" , argv[4] ) ;
    printf ( "argument 5: %s\n" , argv[5] ) ;
    printf ( "argument 6: %s\n" , argv[6] ) ;
    return 0 ;
}
  \end{lstlisting}
  Une fois que les arguments sont terminés, il y a une case qui vaut
  \texttt{(null)} (c'est l'adresse 0), suivie de cases qui ont une
  valeur de la forme \texttt{NOM=chaine}
\end{solutioncachee}

\question Dans le terminal, tapez la commande:
\begin{lstlisting}[language=bash]
  echo $LANGUAGE_PID_FILE
\end{lstlisting}
%$
Que remarquez vous ?

\begin{solutioncachee}
  Plein de choses !
  \begin{itemize}
  \item La mémoire du programme contient aussi les valeurs des
    \emph{variables d'environnement} qui sont définies dans le
    terminal où le programme a été exécuté.
  \item En fait, la mémoire contient deux tableaux l'un à côté de l'autre:
    \begin{itemize}
    \item Dans le premier, il y a l'adresse des arguments du programme;
    \item Dans le second, il y a la valeur des variables d'environnement;
    \item La fin de ces tableaux est marquée avec l'adresse 0 (comme
      la fin d'une chaîne de caractères est marquée par l'entier 0);
    \item \textbf{Très mauvaise nouvelle}: en C, il n'y a aucun
      mécanisme qui oblige à rester dans un tableau. Quand on parcourt
      les cases tableau, si on ne fait pas attention, on fini par
      tomber sur des cases dans la mémoire qui peuvent être définies
      (ou pas), et qui contiennent d'autres données au bon vouloir du
      compilateur. 
    \end{itemize}
  \end{itemize}

\end{solutioncachee}

\question On va maintenant tenter de voir ce qu'il y a plus
loin. Affichez la case 80 du tableau \texttt{argv}.

\begin{solutioncachee}
  Lors des tests, celà suffit pour avoir un \emph{segmentation
    fault}. Donc si accède à une case qui n'est pas dans le tableau, on peut
  avoir soit:
  \begin{itemize}
  \item un résultat qui vient d'ailleurs (de gcc, par exemple);
  \item une erreur qui arrête le programme. \emph{Segmentation fault}
    signifie qu'on veut accéder à une adresse dans la mémoire qui
    n'est pas définie.
  \end{itemize}
\end{solutioncachee}