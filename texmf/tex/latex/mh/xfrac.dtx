% \iffalse meta-comment
%
% Copyright (C) 2004, 2008-2010 by Morten Hoegholm <mh.ctan@gmail.com>
%
% This work may be distributed and/or modified under the
% conditions of the LaTeX Project Public License, either
% version 1.3c of this license or (at your option) any later
% version. The latest version of this license is in
%    http://www.latex-project.org/lppl.txt
%
% This work has the LPPL maintenance status "maintained".
%
% This Current Maintainer of this work is Morten Hoegholm.
%
% This work consists of the main source file xfrac.dtx
% and the derived files
%    xfrac.sty, xfrac.pdf, xfrac.ins, xfrac.drv.
%
% Distribution:
%    CTAN:macros/latex/contrib/mh/xfrac.dtx
%    CTAN:macros/latex/contrib/mh/xfrac.pdf
%
% Unpacking:
%    (a) If xfrac.ins is present:
%           tex xfrac.ins
%    (b) Without xfrac.ins:
%           tex xfrac.dtx
%    (c) If you insist on using LaTeX
%           latex \let\install=y% \iffalse meta-comment
%
% Copyright (C) 2004, 2008-2010 by Morten Hoegholm <mh.ctan@gmail.com>
%
% This work may be distributed and/or modified under the
% conditions of the LaTeX Project Public License, either
% version 1.3c of this license or (at your option) any later
% version. The latest version of this license is in
%    http://www.latex-project.org/lppl.txt
%
% This work has the LPPL maintenance status "maintained".
%
% This Current Maintainer of this work is Morten Hoegholm.
%
% This work consists of the main source file xfrac.dtx
% and the derived files
%    xfrac.sty, xfrac.pdf, xfrac.ins, xfrac.drv.
%
% Distribution:
%    CTAN:macros/latex/contrib/mh/xfrac.dtx
%    CTAN:macros/latex/contrib/mh/xfrac.pdf
%
% Unpacking:
%    (a) If xfrac.ins is present:
%           tex xfrac.ins
%    (b) Without xfrac.ins:
%           tex xfrac.dtx
%    (c) If you insist on using LaTeX
%           latex \let\install=y% \iffalse meta-comment
%
% Copyright (C) 2004, 2008-2010 by Morten Hoegholm <mh.ctan@gmail.com>
%
% This work may be distributed and/or modified under the
% conditions of the LaTeX Project Public License, either
% version 1.3c of this license or (at your option) any later
% version. The latest version of this license is in
%    http://www.latex-project.org/lppl.txt
%
% This work has the LPPL maintenance status "maintained".
%
% This Current Maintainer of this work is Morten Hoegholm.
%
% This work consists of the main source file xfrac.dtx
% and the derived files
%    xfrac.sty, xfrac.pdf, xfrac.ins, xfrac.drv.
%
% Distribution:
%    CTAN:macros/latex/contrib/mh/xfrac.dtx
%    CTAN:macros/latex/contrib/mh/xfrac.pdf
%
% Unpacking:
%    (a) If xfrac.ins is present:
%           tex xfrac.ins
%    (b) Without xfrac.ins:
%           tex xfrac.dtx
%    (c) If you insist on using LaTeX
%           latex \let\install=y% \iffalse meta-comment
%
% Copyright (C) 2004, 2008-2010 by Morten Hoegholm <mh.ctan@gmail.com>
%
% This work may be distributed and/or modified under the
% conditions of the LaTeX Project Public License, either
% version 1.3c of this license or (at your option) any later
% version. The latest version of this license is in
%    http://www.latex-project.org/lppl.txt
%
% This work has the LPPL maintenance status "maintained".
%
% This Current Maintainer of this work is Morten Hoegholm.
%
% This work consists of the main source file xfrac.dtx
% and the derived files
%    xfrac.sty, xfrac.pdf, xfrac.ins, xfrac.drv.
%
% Distribution:
%    CTAN:macros/latex/contrib/mh/xfrac.dtx
%    CTAN:macros/latex/contrib/mh/xfrac.pdf
%
% Unpacking:
%    (a) If xfrac.ins is present:
%           tex xfrac.ins
%    (b) Without xfrac.ins:
%           tex xfrac.dtx
%    (c) If you insist on using LaTeX
%           latex \let\install=y\input{xfrac.dtx}
%        (quote the arguments according to the demands of your shell)
%
% Documentation:
%    (a) If xfrac.drv is present:
%           latex xfrac.drv
%    (b) Without xfrac.drv:
%           latex xfrac.dtx; ...
%    The class ltxdoc loads the configuration file ltxdoc.cfg
%    if available. Here you can specify further options, e.g.
%    use A4 as paper format:
%       \PassOptionsToClass{a4paper}{article}
%
%    Programm calls to get the documentation (example):
%       pdflatex xfrac.dtx
%       makeindex -s gind.ist xfrac.idx
%       pdflatex xfrac.dtx
%       makeindex -s gind.ist xfrac.idx
%       pdflatex xfrac.dtx
%
% Installation:
%    TDS:tex/latex/mh/xfrac.sty
%    TDS:doc/latex/mh/xfrac.pdf
%    TDS:source/latex/mh/xfrac.dtx
%
%<*ignore>
\begingroup
  \def\x{LaTeX2e}
\expandafter\endgroup
\ifcase 0\ifx\install y1\fi\expandafter
         \ifx\csname processbatchFile\endcsname\relax\else1\fi
         \ifx\fmtname\x\else 1\fi\relax
\else\csname fi\endcsname
%</ignore>
%<*install>
\input docstrip.tex
\Msg{************************************************************************}
\Msg{* Installation}
\Msg{* Package: xfrac 2009/08/10 v0.3 textstyle fractions (MH)}
\Msg{************************************************************************}

\keepsilent
\askforoverwritefalse

\preamble

This is a generated file.

Copyright (C) 2004, 2008-2010 by Morten Hoegholm <mh.ctan@gmail.com>

This work may be distributed and/or modified under the
conditions of the LaTeX Project Public License, either
version 1.3c of this license or (at your option) any later
version. The latest version of this license is in
   http://www.latex-project.org/lppl.txt

This work has the LPPL maintenance status "maintained".

This Current Maintainer of this work is Morten Hoegholm.

This work consists of the main source file xfrac.dtx
and the derived files
   xfrac.sty, xfrac.pdf, xfrac.ins, xfrac.drv.

\endpreamble

\generate{%
  \file{xfrac.ins}{\from{xfrac.dtx}{install}}%
  \file{xfrac.drv}{\from{xfrac.dtx}{driver}}%
  \usedir{tex/latex/mh}%
  \file{xfrac.sty}{\from{xfrac.dtx}{package}}%
}

\obeyspaces
\Msg{************************************************************************}
\Msg{*}
\Msg{* To finish the installation you have to move the following}
\Msg{* file into a directory searched by TeX:}
\Msg{*}
\Msg{*     xfrac.sty}
\Msg{*}
\Msg{* To produce the documentation run the file `xfrac.drv'}
\Msg{* through LaTeX.}
\Msg{*}
\Msg{* Happy TeXing!}
\Msg{*}
\Msg{************************************************************************}

\endbatchfile
%</install>
%<*ignore>
\fi
%</ignore>
%<*driver>
\NeedsTeXFormat{LaTeX2e}
\ProvidesFile{xfrac.drv}%
  [2010/02/02 v0.3 Text style fractions]
\RequirePackage{fix-cm}
\documentclass{ltxdoc}


\usepackage{xfrac}
\usepackage{nicefrac}
\usepackage[latin1]{inputenc}
\usepackage[T1]{fontenc}
\makeatletter
\newenvironment{TemplateInterfaceDescription}[1]
  {\subsection{The Template Type `#1'}%
   \begingroup
   \@beginparpenalty\@M
   \description
   \def\TemplateArgument##1##2{\item[Arg: ##1]##2\par}%
   \def\TemplateSemantics{\enddescription\endgroup
       \subsubsection*{Semantics:}}%
  }
  {\par\bigskip}

\newenvironment{TemplateDescription}[2]
  {\subsection{The Template `#2' (type #1)}%
   \subsubsection*{Attributes:}%
   \begingroup
   \@beginparpenalty\@M
   \description
   \def\TemplateKey##1##2##3##4{\item[##1 (##2)]##3%
     \ifx\TemplateKey##4\TemplateKey\else
%         \hskip0ptplus3em\penalty-500\hskip 0pt plus 1filll Default:~##4%
         \hfill\penalty500\relax\qquad \hbox{}\hfill Default:~##4%
         \nobreak\hskip-\parfillskip\hskip0pt\relax
     \fi
     \par}%
   \def\TemplateSemantics{\enddescription\endgroup
       \subsubsection*{Semantics \& Comments:}}%
  }
  {\par\bigskip}

\renewcommand*\MakePrivateLetters{\makeatletter\catcode`\_=11\relax}
\makeatother

\providecommand*\eTeX{\ensuremath{\varepsilon}-\kern-.125em\TeX}
\providecommand*\LaTeXplus{\LaTeXe$*$}
\providecommand*\key[1]{\textbf{#1}}
\providecommand*\pkg[1]{\textsf{#1}}
\newcommand*\switch[2]{{\fontfamily{#1}\selectfont #2}}

\usepackage{lmodern}
\begin{document}
  \catcode`\_=12
  \DocInput{xfrac.dtx}
\end{document}
%</driver>
% \fi
%
%  \GetFileInfo{xfrac.drv}
%  \CheckSum{249}
%  
%  \changes{v0.3}{2010/02/02}{Update to new \pkg{xtemplate} system}
%  \changes{v0.3}{2010/02/02}{Include custom values for
%    \pkg{lmodern}}
%
%  \changes{v0.2a}{2009/08/10}{Update for new version of expl3}
%
%  \changes{v0.2}{2008/08/04}{Require expl3 and get rid of .ins file}
%  \changes{v0.2}{2008/08/04}{Move to macros/latex/contrib/mh on CTAN}
%
%  \changes{v0.11}{2004/05/08}{Consistent naming}
%
%  \changes{v0.10a}{2004/04/06}{Added a dependency of the latest
%  \LaTeX{} release}
%  \changes{v0.10a}{2004/04/06}{First version on \textsc{ctan}}
%
%  \changes{v0.10}{2004/04/05}{First publicly available prototype
%  implementation}
%
%  \title{The \pkg{xfrac} package\thanks{This file has version number
%  \fileversion, last revised \filedate.}}
%
%  \author{Morten H\o gholm}
%  \date{\filedate}
%
%  \maketitle
%
%  \begin{abstract}
%  This package uses a template interface to produce nicely looking
%  \emph{split level} fractions like \nicefrac{7}{9}\ldots{}
%  ehrm\ldots{} I mean \sfrac{7}{9}.
%  \end{abstract}
%
%  \tableofcontents
%
%  \section{User Interface}
%
%  The \pkg{xfrac} package defines a document command \cs{sfrac}
%  with the following syntax:
%  \begin{quote}
%  \cs{sfrac}\oarg{instance}\marg{num}\oarg{sep}\marg{denom}
%  \end{quote}
%  Let's show a few examples:
%  \begin{verbatim}
%  \sfrac{1}{2},  $\sfrac{1}{2}$,
%  $\mathbf{3\times\sfrac{1}{2}}$
%  \quad \fontfamily{ppl}\selectfont Palatino: \sfrac{1}{2}
%  \quad \fontfamily{ptm}\selectfont Times: \sfrac{1}{2}
%  \end{verbatim}
%  \begin{quote}
%  \sfrac{1}{2},  $\sfrac{1}{2}$, $\mathbf{3\times\sfrac{1}{2}}$
%  \quad \fontfamily{ppl}\selectfont Palatino: \sfrac{1}{2}
%  \quad \fontfamily{ptm}\selectfont Times: \sfrac{1}{2}
%  \end{quote}
%  You'll notice something interesting: Not only does the \cs{sfrac}
%  command work as it should in math mode, it also gets the job done
%  for other fonts as well.
%
%
%  \section{A Bit of History}
%
%  \subsection{The Past}
%
%  One of the first exercises in \emph{The \TeX Book} is to design a
%  macro for split level fractions. The solution presented is fairly
%  simple, using a \emph{virgule} (a slash) for separating the two
%  components. It looks okay because the text font and math font of
%  Computer Modern look almost identical.
%
%  The proper symbol to use instead of the virgule is a \emph{solidus}
%  which does not exist in Computer Modern. It is however available in
%  the European Computer Modern fonts, but I'll get back to that.
%
%  \subsection{The Present}
%
%
%  The most common way to produce split level fractions within \LaTeX\
%  is by means of the \pkg{nicefrac} package. Part of the reason it
%  has found widespread use is due to the strange design of the
%  built-in text fractions of the EC fonts, which look like this:
%  \textonehalf. The package is very simple to use but there are a few
%  issues:
% \begin{itemize}
%  \item It uses the virgule instead of the solidus.
%  \item Font size of numerator and denominator is bigger than in the
%    built-in symbol. Compare Palatino: \switch{ppl}{\nicefrac{1}{2}}
%    vs. \switch{ppl}{\textonehalf }.
%  \item It doesn't correct for fonts using text figures such as in the
%    \pkg{eco} package. Compare \switch{cmor}{\nicefrac{1}{2}} and
%    \switch{cmor}{\nicefrac{8}{9}}.
%  \item In math mode, it doesn't always pick up the correct math
%    alphabet.
% \end{itemize}
% In short: \pkg{nicefrac} doesn't attempt to be the answer to
% everything and so this is not a criticism of the package. It works
% quite well for Computer Modern which was pretty much what was widely
% available at the time it was developed. Users these days, however,
% have a choice of many fonts when they write their documents.
%
%
%
%
%
%  \subsection{The Future}
%
%  Fonts are wildly different; one macro that works fine for Computer
%  Modern obviously doesn't work well at all in Palatino. For one we
%  have to make the separator symbol configurable, and we need to
%  take care of several details as well: font scaling of the
%  numerator/denominator pair (ND), font selection of ND etc. If we
%  are to have a single package for this in the future\footnote{As
%  this is intended to be about the future, the \pkg{xfrac} package
%  requires the \eTeX{} extensions.} we have to define a totally
%  generic interface for the fraction commands and then adjust
%  parameters depending on the current font. What you see in this
%  prototype implementation of \pkg{xfrac} is just that.
%
%
%  \section{Advanced User Interface}
%
%
%  \subsection{Text mode}
%
%  The usual problem in text mode has a name: Computer Modern. The
%  solidi of all the Computer Modern fonts leave a lot to be desired,
%  although things are potentially looking better as the Latin Modern
%  fonts are becoming more stable and widespread. As long as the
%  default fonts are Computer Modern variants we must however work
%  around this. One idea that comes to mind is to see what happens
%  when you use a solidus from another font instead. Let's try with
%  Times:
%  \begin{quote}
%  \DeclareInstance{xfrac}{cmr2}{text}{
%    slash-symbol-font = ptm,
%  }
%  ``You take \sfrac[cmr2]{1}{2} cup of sugar, \ldots''
%  \end{quote}
%  That looks quite good actually, so it was probably very difficult
%  to obtain that result. Nope, it was extremely easy---if you happen
%  to know about \emph{instances}:
%  \begin{verbatim}
%  \DeclareInstance{xfrac}{cmr}{text}{
%    slash-symbol-font = ptm,
%  }
%  \end{verbatim}
%  So we define an instance with the name `cmr' from the template
%  `text' which in turn is of template type `xfrac'. You'll notice
%  the `cmr' is also the name of the font family for Computer Modern
%  Roman and the reasoning behind is that every font family should
%  have it's own settings, and if a document command is to work well
%  in that scheme, letting it use the name of the current font family
%  seems like a good idea. Thus the \cs{sfrac} command checks to see
%  whether an instance with same name as the current font family
%  exists and uses it if the test is true; otherwise the default
%  setting is used. Here we defined the instance to be used for the
%  font family `cmr' and just told it to use the Times font for
%  typesetting the slash symbol which turns out to be a solidus by
%  default.
%
%  The option \texttt{cm-recommended} which is loaded by default uses
%  the Times solidus for Computer Modern Roman and Computer Modern
%  Sans Serif and the Palatino solidus for Computer Modern Typewriter
%  Type. This looks quite good. Should you however not want this you
%  can use the option \texttt{cm-standard} which produces somewhat
%  acceptable results using Computer Modern exclusively.
%
%  So what about old style figures? If you use the \pkg{eco}
%  package you might define an instance similar to this (`cmor' is
%  the name of the roman font activated by \pkg{eco}):
%   \DeclareInstance{xfrac}{cmor}{text}{
%    slash-symbol-font = ptm,
%    numerator-font    = cmr,
%    denominator-font  = cmr,
%  }
%  \begin{verbatim}
%  \DeclareInstance{xfrac}{cmor}{text}{
%    slash-symbol-font = ptm,
%    numerator-font    = cmr,
%    denominator-font  = cmr,
%  }
%  \end{verbatim}
%  We also use regular Computer Modern Roman for typesetting ND, so
%  we end up with \switch{cmor}{\sfrac{1}{2}} and
%  \switch{cmor}{\sfrac{8}{9}} instead of
%  \switch{cmor}{\nicefrac{1}{2}} and \switch{cmor}{\nicefrac{8}{9}}.
%  Much better.
%
%  There are also situations where other tricks are useful. If you
%  don't have the inferior and superior figures available in a font,
%  or the font doesn't have a wider design for small font sizes, you
%  can cheat by manually scaling the ND-pair. I got nice results for
%  Adobe's Stempel Garamond (with small caps and old style figures)
%  with the following setup:
%  \begin{verbatim}
%  \DeclareInstance{xfrac}{pegj}{text}{
%    numerator-font   = pegx,
%    denominator-font = pegx,
%    scale-factor     = .9,
%    h-scale          = 1.1,
%  }
%  \end{verbatim}
%  We use the font family `pegx' (Stempel Garamond with real small
%  caps) for typesetting the ND-pair. Additionally the key
%  \key{scale-factor} specifies that the font size used for the
%  ND-pair should be $0.9$ of the height of the solidus, and the key
%  \key{h-scale} specifies that the ND-pair should be scaled an extra
%  10\% horizontally.
%
%  Should you be so fortunate the have a font with inferior and
%  superior figures like in the Monotype Janson example from Philipp
%  Lehman's excellent \emph{The Font Installation Guide}. In that
%  example Philipp defines the font families `mjn0' for the inferior
%  figures and `mjn1' for the superior. Thus to get the \cs{sfrac}
%  command to use them on the fly for the font family `mjnj' (Janson,
%  old style figures) we would say
%  \begin{verbatim}
%  \DeclareInstance{xfrac}{mjnj}{text}{
%    numerator-font      = mjn1,
%    denominator-font    = mjn0,
%    scaling             = false,
%    numerator-bot-sep   = 0pt,
%    denominator-bot-sep = 0pt,
%  }
%  \end{verbatim}
%  I think this example is a very clean way to do it. An alternative
%  approach could be to use the keys \key{numerator-format} and
%  \key{denominator-format} to process the arguments and let them
%  determine what to do.
%
%  As a side note Harald Harders was so kind to test it, and it
%  \emph{does} actually work---I hadn't tested it myself.
%
%  \subsection{Math Mode}
%
%  In math mode the choices are a lot fewer because first of all
%  \TeX{} comes with a built-in limitation of 16 math families.
%  Additionally we will not need a solidus for typesetting split
%  fractions in math, as tradition is to use a virgule instead. We
%  define the basic `mathdefault' instance to simply use the math
%  family in use when the instance is run. So if we're in normal math
%  like |$\sfrac{7}{9}$| we simply get family~$-1$. If we're inside a
%  \cs{mathbf} we're in family~$4$ (in the standard setup at least),
%  and so the fraction is typeset with the same math family. Simple,
%  isn't?
%
%  You can also declare instances for the math families, but I really
%  don't see why you would. If you do then name them according to the
%  scheme `mathfam\meta{N}', where \meta{N} is the family number, and
%  only do it if you \emph{really} know how to set up math fonts.
%  That is, if \cs{DeclareMathAlphabet} is unbeknownst to you, then
%  just don't go there.
%
%
%  Another example: If we want \cs{sfrac} to produce split fractions
%  without doing anything at all, we can choose the collection
%  `plainmath', which is defined as
%  \begin{verbatim}
%  \DeclareCollectionInstance{plainmath}{xfrac}{mathdefault}{math}{
%    denominator-bot-sep = 0pt,
%    numerator-bot-sep   = 0pt,
%    numerator-top-sep   = \c_max_dim,
%    scaling             = false,
%    slash-right-mkern   = 0mu,
%    slash-left-mkern    = 0mu,
%  }
%  \end{verbatim}
%  This creates an alternative version of the instance `mathdefault'
%  with settings as specified by the keys. In the default math setup
%  \key{numerator-top-sep} is set to 0~pt, and here we set
%  \key{numerator-bot-sep} to 0~pt as well, so in order to avoid
%  over-specification (and an error message) we must set
%  \key{numerator-top-sep} to \cs{c_max_dim}. We activate (obeying
%  normal scoping rules) it with:
%  \begin{verbatim}
%  \UseCollection{xfrac}{plainmath}
%  \end{verbatim}
%  Then |$\sfrac{8}{13}$| produces\begingroup
%    \UseCollection{xfrac}{plainmath}
%    $\sfrac{8}{13}$ and just typing |$8/13$| gives the same result:
%    $8/13$.
%  \endgroup
%
%
%  \section{The Template Interface}
%
%  \begin{TemplateInterfaceDescription}{xfrac}
%
%  \TemplateArgument{1}
%      {The numerator}
%
%  \TemplateArgument{2}
%       {The separator}
%
%   \TemplateArgument{3}
%      {The denominator}
%
%  \TemplateSemantics
%
%  Typesets arguments 1 and 3 separated by argument 2, which in text
%  mode by default is a \emph{solidus}. This is taken from
%  \pkg{textcomp} where it is denoted \cs{textfractionsolidus}.
%  This is the character used for the ready made split level
%  fractions such as \textonehalf---except in the (European) Computer
%  Modern fonts. In math mode a \emph{virgule} is used instead as
%  this is more appropriate and it is always available in the math
%  fonts. The solidus is a text symbol only.
%
%  \end{TemplateInterfaceDescription}
%
%  \begin{TemplateDescription}{xfrac}{text}
%
%  \TemplateKey{numerator-font}{tokenlist}
%    {Font family specification to use for the numerator.}
%    {\cs{f@family}}
%
%  \TemplateKey{numerator-format}{function 1 arg}
%    {Action to be taken on the numerator.}
%    {Process argument unchanged}
%
%  \TemplateKey{slash-symbol}{tokenlist}
%    {The separator symbol. If not specified the default value will be
%    used instead.}
%    {Solidus (\cs{textfractionsolidus})}
%
%  \TemplateKey{slash-symbol-font}{tokenlist}
%    {Font family specification to use for the separator symbol.}
%    {\cs{f@family}}
%
%  \TemplateKey{slash-symbol-format}{function 1 arg}
%    {Action to be taken on the separator symbol.}
%    {Process argument unchanged}
%
%  \TemplateKey{denominator-font}{tokenlist}
%    {Font family specification to use for the denominator.}
%    {\cs{f@family}}
%
%  \TemplateKey{denominator-format}{function 1 arg}
%    {Action to be taken on the denominator.}
%    {Process argument unchanged}
%
%  \TemplateKey{h-scale}{tokenlist}
%    {Factor by which the numerator and denominator should be
%    horizontally scaled. It should only be used if the real superior
%    and inferior fonts are not available. For instance Stempel
%    Garamond looks excellent if scaled 10\% extra horizontally, i.e.,
%    by a factor of 1.1.}
%    {1}
%
%  \TemplateKey{v-scale}{tokenlist}
%    {Same as \key{h-scale} only vertically. Probably not of much use
%    but added for completetion.}
%    {1}
%
%  \TemplateKey{scale-factor}{tokenlist}
%    {Fraction of the size of \key{slash-symbol}. Used for setting the
%    font size of numerator and denominator. Usually a value of app.\
%    \sfrac{5}{6} produces fine results. It should only be used if the
%    real superior and inferior fonts are not available. As an example
%    Stempel Garamond looks better if the factor is 0.9.}
%    {0.83333}
%
%  \TemplateKey{scale-relative}{choice}
%    {If set to `true' the font size of the numerator and denominator
%    is scaled with respect to the height of the \key{slash-symbol}. If
%    set to `false' the font is scaled with respect to the total height
%    of the \key{slash-symbol}.}
%    {true}
%
%  \TemplateKey{scaling}{choice}
%    {If set to `true' the fonts are allowed to scale. If set to
%    `false' they are not. See the `Janson' example for an application.}
%    {true}
%
%  \TemplateKey{numerator-top-sep}{length}
%    {Dimension specifying the space between the top of the
%    \key{slash-symbol} and the top of the numerator. If not specified,
%    the depth of the solidus will be used, because this value will
%    make the fraction look even.}
%    {Unspecified}
%
%  \TemplateKey{numerator-bot-sep}{length}
%    {Dimension specifying the lift of the numerator from the
%    baseline.}
%    {Unspecified}
%
%  \TemplateKey{denominator-bot-sep}{length}
%    {Dimension specifying the lift of the denominator from the
%    baseline.}
%    {Unspecified}
%
%  \TemplateKey{slash-right-kern}{length}
%    {Dimension specifying the kerning between the \key{slash-symbol}
%    and the numerator.}
%    {\texttt{0pt}}
%
%  \TemplateKey{slash-left-kern}{length}
%    {Dimension specifying the kerning between the \key{slash-symbol}
%    and the denominator.}
%    {\texttt{0pt}}
%
%  \TemplateKey{math-mode}{choice}
%    {Are we in math mode or not?}
%    {false}
%
%  \TemplateKey{phantom}{tokenlist}
%    {A character that suits the common cases. As we would mostly want
%    to use numbers in text mode we choose a `tall' number, while in
%    math it is somewhat different.}
%    {8}
%
%  \TemplateSemantics
%
%  This template is also the foundation for the `math' template. The
%  keys \key{slash-right-mkern} and \key{slash-left-mkern} can only
%  be used in math mode and are not shown here.
%
%  \end{TemplateDescription}
%
%
%  \begin{TemplateDescription}{xfrac}{math}
%
%  \TemplateKey{numerator-font}{tokenlist}
%    {Font family specification to use for the numerator.}
%    {\cs{number}\cs{fam}}
%
%  \TemplateKey{slash-symbol}{tokenlist}
%    {The separator symbol. If not specified the default value will be
%    used instead.}
%    {Virgule ($/$)}
%
%  \TemplateKey{slash-symbol-font}{tokenlist}
%    {Font family specification to use for the separator symbol.}
%    {\cs{number}\cs{fam}}
%
%  \TemplateKey{denominator-font}{tokenlist}
%    {Font family specification to use for the denominator.}
%    {\cs{number}\cs{fam}}
%
%  \TemplateKey{scale-factor}{tokenlist}
%    {Fraction of the size of \key{slash-symbol}. In math mode we
%    cannot rely on the fonts to be able to scale, but giving a default
%    scale of 0.7 fits into the regular size changing scheme---the
%    default scheme has values $(D,T,S,SS)=(1,1,0.7,0.5)$ whereas we
%    with a default \key{scale-factor} of 0.7 get $(1,1,0.7,0.49)$.
%    That's close enough.}
%    {0.7}
%
%  \TemplateKey{scale-relative}{choice}
%    {If set to `true' the font size of the numerator and denominator
%    is scaled with respect to the height of the \key{slash-symbol}. If
%    set to `false' the font is scaled with respect to the total height
%    of the \key{slash-symbol}.}
%    {false}
%
%  \TemplateKey{scaling}{choice}
%    {If set to `true' the fonts are allowed to scale. If set to
%    `false' they are not. See the `plainmath' example for an application.}
%    {true}
%
%  \TemplateKey{numerator-top-sep}{length}
%    {Dimension specifying the space between the top of the
%    \key{slash-symbol} and the top of the numerator. If not specified,
%    the depth of the virgule will be used, because this value will
%    make the fraction look even.}
%    {\texttt{0pt}}
%
%  \TemplateKey{denominator-bot-sep}{length}
%    {Dimension specifying the lift of the denominator from the
%    baseline.}
%    {\texttt{0pt}}
%
%  \TemplateKey{slash-right-mkern}{tokenlist}
%    {Same as \key{slash-right-kern} but for math mode only and should
%    be specified in \texttt{mu} units. This is because \pkg{calc}
%    can't use mu-expressions.}
%    {\texttt{-2mu}}
%
%  \TemplateKey{slash-left-mkern}{tokenlist}
%    {Same as \key{slash-left-kern} but for math mode only and should
%    be specified in \texttt{mu} units. This is because \pkg{calc}
%    can't use mu-expressions.}
%    {\texttt{-1mu}}
%
%  \TemplateKey{math-mode}{choice}
%    {Are we in math mode or not?}
%    {true}
%
%  \TemplateKey{phantom}{tokenlist}
%    {A character that suits the common cases. In math we have a high
%    risk of using a parenthesis, so we choose that. Text mode is
%    another story.}
%    {(}
%
%  \TemplateSemantics
%
%  This template is a restricted version of the `text' template. Only
%  the keys that are different from the `text' template are shown
%  here. Also bear in mind that the attributes \key{slash-left-kern}
%  and \key{slash-right-kern} have no meaning in this template.
%
%  \end{TemplateDescription}
% 
%  \StopEventually{}
%
%\section{Implementation}
%
%    \begin{macrocode}
%<*package>
%    \end{macrocode}
%       
% The usual lead-off: provides an experimental package!
%    \begin{macrocode}
\RequirePackage{expl3}[2009/08/05]
\ProvidesExplPackage{xfrac}{2010/02/02}{0.3}{Text fractions}
%    \end{macrocode}
%
% Some support is needed: a bit wider than the normal \pkg{xpackage}
% stuff, but not by much.
%    \begin{macrocode}
\RequirePackage { amstext , graphicx , l3keys2e , textcomp , xtemplate }
%    \end{macrocode}   
%
%\begin{macro}{\l_xfrac_cm_std_bool}
% There is one option to support.
%    \begin{macrocode}
\keys_define:nn { xfrac } {
  cm-recommended .choice:,
  cm-recommended /
    false        .code:n     =
      { \bool_set_true:N \l_xfrac_cm_std_bool },
  cm-recommended /
    true         .code:n     =
      { \bool_set_false:N \l_xfrac_cm_std_bool },
  cm-recommended .default:n  = { true },    
  cm-standard    .bool_set:N = \l_xfrac_cm_std_bool
}
\ProcessKeysOptions { xfrac }
%    \end{macrocode}
%\end{macro}
%
%\begin{macro}{\l_xfrac_slash_box}
%\begin{macro}{\l_xfrac_tmp_box}
% In keeping with the \LaTeX3 philosophy, rather than use generic 
% scratch boxes and get confused, \pkg{xfrac} reserves its own named
% working space.
%    \begin{macrocode}
\box_new:N \l_xfrac_slash_box
\box_new:N \l_xfrac_tmp_box
%    \end{macrocode}
%\end{macro}
%\end{macro}
%
%\begin{macro}{\xfrac_tmp:w}
% Used for the raised boxes: weird as it does not take an argument
% but the \cs{raisebox} does.
%    \begin{macrocode}
\cs_new:Npn \xfrac_tmp:w { }
%    \end{macrocode}
%\end{macro}
%
%\subsection{Initialisation of variables}
%
% Variables used in templates have to be set up: there is not
% much to say about these, other than that they must exist.
%    
%\begin{macro}{\l_xfrac_denominator_bot_sep_dim}
%\begin{macro}{\l_xfrac_numerator_bot_sep_dim}
%\begin{macro}{\l_xfrac_numerator_top_sep_dim}
%\begin{macro}{\l_xfrac_slash_left_sep_dim}
%\begin{macro}{\l_xfrac_slash_right_sep_dim}
% Fixed lengths.
%    \begin{macrocode}
\dim_new:N \l_xfrac_denominator_bot_sep_dim
\dim_new:N \l_xfrac_numerator_bot_sep_dim
\dim_new:N \l_xfrac_numerator_top_sep_dim
\dim_new:N \l_xfrac_slash_left_sep_dim
\dim_new:N \l_xfrac_slash_right_sep_dim
%    \end{macrocode}
%\end{macro}
%\end{macro}
%\end{macro}
%\end{macro}
%\end{macro}
%
%\begin{macro}{\l_xfrac_denominator_font_tl}
%\begin{macro}{\l_xfrac_hscale_tl}
%\begin{macro}{\l_xfrac_numerator_font_tl}
%\begin{macro}{\l_xfrac_phantom_tl }
%\begin{macro}{\l_xfrac_scale_factor_tl}
%\begin{macro}{\l_xfrac_slash_left_msep_tl}
%\begin{macro}{\l_xfrac_slash_right_msep_tl}
%\begin{macro}{\l_xfrac_slash_symbol_tl}
%\begin{macro}{\l_xfrac_slash_symbol_font_tl}
%\begin{macro}{\l_xfrac_vscale_tl}
% Token lists, which include floating-point numbers and math(s)
% skips.
%    \begin{macrocode}
\tl_new:N \l_xfrac_denominator_font_tl
\tl_new:N \l_xfrac_hscale_tl
\tl_new:N \l_xfrac_numerator_font_tl
\tl_new:N \l_xfrac_phantom_tl 
\tl_new:N \l_xfrac_scale_factor_tl
\tl_new:N \l_xfrac_slash_left_msep_tl
\tl_new:N \l_xfrac_slash_right_msep_tl
\tl_new:N \l_xfrac_slash_symbol_tl
\tl_new:N \l_xfrac_slash_symbol_font_tl
\tl_new:N \l_xfrac_vscale_tl
%    \end{macrocode}
%\end{macro}
%\end{macro}
%\end{macro}
%\end{macro}
%\end{macro}
%\end{macro}
%\end{macro}
%\end{macro}
%\end{macro}
%\end{macro}
%
%\begin{macro}{\xfrac_fontscale:}
%\begin{macro}{\xfrac_math:n}
%\begin{macro}{\xfrac_denominator_font_change:}
%\begin{macro}{\xfrac_denominator_format:n}
%\begin{macro}{\xfrac_numerator_font_change:}
%\begin{macro}{\xfrac_numerator_format:n}
%\begin{macro}{\xfrac_relscale:}
%\begin{macro}{\xfrac_slash_symbol_font_change:}
%\begin{macro}{\xfrac_slash_symbol_format:n}
%\begin{macro}{\xfrac_text_or_math:n}
% Functions, either things which are calculated `on the fly'
% (no argument required) or are functions taking one argument in the
% code.
%    \begin{macrocode}
\cs_new_nopar:Npn \xfrac_fontscale:
\cs_new:Npn \xfrac_math:n #1 { }
\cs_new_nopar:Npn \xfrac_denominator_font_change: { }
\cs_new:Npn \xfrac_denominator_format:n #1 { }
\cs_new_nopar:Npn \xfrac_numerator_font_change: { }
\cs_new:Npn \xfrac_numerator_format:n #1 { }
\cs_new_nopar:Npn \xfrac_relscale: { }
\cs_new_nopar:Npn \xfrac_slash_symbol_font_change: { }
\cs_new:Npn \xfrac_slash_symbol_format:n #1 { }
\cs_new:Npn \xfrac_text_or_math:n #1 { }
%    \end{macrocode}
%\end{macro}
%\end{macro}
%\end{macro}
%\end{macro}
%\end{macro}
%\end{macro}
%\end{macro}
%\end{macro}
%\end{macro}
%\end{macro}
%
%\subsection{The template}
%
% There is only one object type in \pkg{xfrac}, rather unimaginatively
% named \texttt{xfrac}.
%    \begin{macrocode}
\DeclareObjectType { xfrac } { 3 }
%    \end{macrocode}
%
% A single template interface is used for both text and math(s), which
% does make a few things a little complex later.
%    \begin{macrocode}
\DeclareTemplateInterface { xfrac } { text } { 3 } {
  denominator-bot-sep : length     = \c_max_dim           ,
  denominator-font    : tokenlist  = \f@family            ,
  denominator-format  : function 1 = #1                   ,
  h-scale             : tokenlist  = 1                    ,
  math-mode           : choice { false , true }
                                   = false                ,
  numerator-font      : tokenlist  = \f@family            ,
  numerator-format    : function 1 = #1                   ,
  numerator-bot-sep   : length     = \c_max_dim           ,
  numerator-top-sep   : length     = \c_max_dim           ,
  phantom             : tokenlist  = 8                    ,
  scale-factor        : tokenlist  = 0.83333              ,
  scale-relative      : choice { false , true }
                                   = true                 ,
  scaling             : choice { false , true }
                                   = true                 ,
  slash-left-kern     : length     = 0 pt                 ,
  slash-left-mkern    : tokenlist  = -2 mu                ,
  slash-right-kern    : length     = 0 pt                 ,
  slash-right-mkern   : tokenlist  = -1 mu                ,
  slash-symbol        : tokenlist  = \textfractionsolidus ,
  slash-symbol-font   : tokenlist  = \f@family            ,
  slash-symbol-format : function 1 = #1                   ,
  v-scale             : tokenlist  = 1                    ,
}
%    \end{macrocode}
%   
% Most of the variable binding is quite simple: of course, the choices
% are a little more complicated. That is particularly true where 
% these have to set up `on the fly' functions.   
%    \begin{macrocode}
\DeclareTemplateCode { xfrac } { text } { 3 } 
  {
    denominator-bot-sep = \l_xfrac_denominator_bot_sep_dim ,
    denominator-font    = \l_xfrac_denominator_font_tl     ,
    denominator-format  = \xfrac_denominator_format:n      ,
    h-scale             = \l_xfrac_hscale_tl               ,
    math-mode           =
      {
        false = \cs_set_eq:NN \xfrac_math:n \use:n,
        true  = \cs_set_eq:NN \xfrac_math:n \ensuremath
      },
    numerator-font      = \l_xfrac_numerator_font_tl       ,
    numerator-format    = \xfrac_numerator_format:n        ,
    numerator-bot-sep   = \l_xfrac_numerator_bot_sep_dim   ,
    numerator-top-sep   = \l_xfrac_numerator_top_sep_dim   ,
    phantom             = \l_xfrac_phantom_tl              ,
    scale-factor        = \l_xfrac_scale_factor_tl         ,
    scale-relative      =
      {
        false = 
          \cs_set_nopar:Npn \xfrac_relscale:
            {
              \dim_eval:n 
                { 
                    \box_ht:N \l_xfrac_tmp_box 
                  + \box_dp:N \l_xfrac_tmp_box
                } 
            },
        true  = 
          \cs_set_nopar:Npn \xfrac_relscale: 
            { \box_ht:N \l_xfrac_slash_box }
      },
    scaling             =
      {
        false = \cs_set_eq:NN \xfrac_fontscale: \prg_do_nothing:, 
        true  =  
          \cs_set_nopar:Npn \xfrac_fontscale:
            {
              \fontsize { \l_xfrac_scale_factor_tl \xfrac_relscale: }
                { \c_zero_dim } 
              \selectfont
            }
      },
    slash-left-kern     = \l_xfrac_slash_left_sep_dim      ,
    slash-left-mkern    = \l_xfrac_slash_left_msep_tl      ,
    slash-right-kern    = \l_xfrac_slash_right_sep_dim     ,
    slash-right-mkern   = \l_xfrac_slash_right_msep_tl     ,
    slash-symbol        = \l_xfrac_slash_symbol_tl         ,
    slash-symbol-font   = \l_xfrac_slash_symbol_font_tl    ,
    slash-symbol-format = \xfrac_slash_symbol_format:n     ,
    v-scale             = \l_xfrac_vscale_tl        
  }
%    \end{macrocode}
% The implementation part starts with applying all of the settings
% from above. The first part of the set up is then to determine
% whether the surroundings are text or math(s), and react accordingly.
%    \begin{macrocode}
  {
    \AssignTemplateKeys
    \mode_if_math:TF
      {
        \cs_set_eq:NN \xfrac_text_or_math:n \text
        \cs_set_nopar:Npx \xfrac_denominator_font_change:
          { \tex_fam:D \l_xfrac_denominator_font_tl }
        \cs_set_nopar:Npx \xfrac_numerator_font_change:
          { \tex_fam:D \l_xfrac_numerator_font_tl }
        \cs_set_nopar:Npx \xfrac_slash_symbol_font_change:
          { \tex_fam:D \l_xfrac_slash_symbol_font_tl }
      }
      {
        \cs_set_eq:NN \xfrac_text_or_math:n \mbox
        \cs_set_nopar:Npn \xfrac_denominator_font_change:
          { 
            \fontfamily { \l_xfrac_denominator_font_tl }
            \selectfont
          }
        \cs_set_nopar:Npn \xfrac_numerator_font_change:
          { 
            \fontfamily { \l_xfrac_numerator_font_tl }
            \selectfont
          }
        \cs_set_nopar:Npn \xfrac_slash_symbol_font_change:
          { 
            \fontfamily { \l_xfrac_slash_symbol_font_tl }
            \selectfont
          }
      }
%    \end{macrocode}
%\changes{v0.11a}{2004/08/24}{Added \cs{m@th}}
% Everything is now either inside \cs{text} or an \cs{mbox}, depending
% upon the surroundings. First, there are some boxes to set up.
%    \begin{macrocode}
    \xfrac_text_or_math:n 
      {
        \m@th
        \hbox_set:Nn \l_xfrac_tmp_box 
          { \xfrac_math:n { \vphantom { ( ) } } }
        \hbox_set:Nn \l_xfrac_slash_box 
          {
            \xfrac_math:n 
              { 
                \xfrac_slash_symbol_format:n 
                  {
                    \xfrac_math:n 
                      {
                        \xfrac_slash_symbol_font_change:
                        \IfNoValueTF {#2} 
                          { \l_xfrac_slash_symbol_tl } {#2}
                      }
                  }
              }
          }
%    \end{macrocode}
% Check on the numerator separator dimensions. The code starts with the
% assumption that neither has been given, as this can then be used to
% set up a default, which is also used when both values are set 
% erroneously.
%    \begin{macrocode}
        \cs_set_nopar:Npn \xfrac_tmp:w
          {
            \raisebox 
              {
                  \box_ht:N \l_xfrac_slash_box
                - \box_dp:N \l_xfrac_slash_box
                - \height
              }
          }
        \dim_compare:nNnTF 
          { \l_xfrac_numerator_top_sep_dim } = { \c_max_dim }
          {
            \dim_compare:nNnF
              { \l_xfrac_numerator_bot_sep_dim } = { \c_max_dim } 
              {
                \cs_set_nopar:Npn \xfrac_tmp:w
                  { 
                    \raisebox 
                      { \dim_use:N \l_xfrac_numerator_bot_sep_dim } 
                  }
              }
          }
          {
            \dim_compare:nNnTF
              { \l_xfrac_numerator_bot_sep_dim } = { \c_max_dim } 
                { 
                  \cs_set_nopar:Npn \xfrac_tmp:w
                    { 
                      \raisebox 
                        { 
                            \box_ht:N \l_xfrac_slash_box
                          - \dim_use:N \l_xfrac_numerator_top_sep_dim 
                          - \height
                        }
                    }
                }
                { 
                  \msg_error:nn { xfrac } 
                    { over-specified-numerator-sep }
                }
          }
%    \end{macrocode}
%  Typeset the numerator.
%    \begin{macrocode}
        \xfrac_tmp:w
          {
            \xfrac_fontscale:
            \xfrac_numerator_format:n
              {
                \scalebox { \l_xfrac_hscale_tl } [ \l_xfrac_vscale_tl ]
                  { 
                    \xfrac_math:n 
                      { 
                        \xfrac_numerator_font_change:
                        {
                          \vphantom { \l_xfrac_phantom_tl }
                          #1
                        }
                      } 
                  }
              }
          }
        \xfrac_math:n
          { % THIS IS JUST WRONG! 
            \mode_if_math:TF
              { \tex_mskip:D \l_xfrac_slash_right_msep_tl }
              { \tex_hskip:D \l_xfrac_slash_right_sep_dim }
          }  
%    \end{macrocode}
%  Typeset the separator.
%    \begin{macrocode}
        \box_use:N \l_xfrac_slash_box
        \xfrac_math:n 
          {
            \mode_if_math:TF
              { \tex_mskip:D \l_xfrac_slash_left_msep_tl }
              { \tex_hskip:D \l_xfrac_slash_left_sep_dim }
          }
%    \end{macrocode}
%  Typeset the denominator.
%    \begin{macrocode}
        \dim_compare:nNnTF 
          { \l_xfrac_denominator_bot_sep_dim } = { \c_max_dim }
          {
            \cs_set_nopar:Npn \xfrac_tmp:w
              { \raisebox { - \box_dp:N \l_xfrac_slash_box } } 
          }
          {
            \cs_set_nopar:Npn \xfrac_tmp:w
              { 
                \raisebox 
                  { \dim_use:N \l_xfrac_denominator_bot_sep_dim }
              }
          }  
        \xfrac_tmp:w
          {
            \xfrac_fontscale:
            \xfrac_denominator_format:n
              {
                \scalebox { \l_xfrac_hscale_tl } [ \l_xfrac_vscale_tl ]
                  { 
                    \xfrac_math:n 
                      { 
                        \xfrac_denominator_font_change:
                        {
                          \vphantom { \l_xfrac_phantom_tl }
                          #3
                        }
                      } 
                  }
              }
          }
      } 
  }
%    \end{macrocode}
%    
% Since math(s) and text mode are wildly different entities we define a
% separate template for each. You already saw the `text' 
% template, and here is the `math' template.
%    \begin{macrocode}
\DeclareRestrictedTemplate { xfrac } { text } { math } {
  numerator-font      = \number \fam ,
  slash-symbol        = /            ,
  slash-symbol-font   = \number \fam ,
  denominator-font    = \number \fam ,
  scale-factor        = 0.7          ,
  scale-relative      = false        ,
  scaling             = true         ,
  numerator-top-sep   = 0 pt         ,
  denominator-bot-sep = 0 pt         ,
  math-mode           = true         ,
  phantom             = (
}
%    \end{macrocode}
%    
%\subsection{The standard instances}
%
% For the default instances we just use the relevant templates with
% the default settings.
% 
%  The default `text' instance.
%    \begin{macrocode}
\DeclareInstance { xfrac } { default } { text } { }
%    \end{macrocode}
%    
%  The default `math(s)' instance.
%    \begin{macrocode}
\DeclareInstance { xfrac } { mathdefault } { math } { }
%    \end{macrocode}
%    \begin{macrocode}
\DeclareCollectionInstance { plainmath } { xfrac } { mathdefault } 
  { math }{
  denominator-bot-sep = 0 pt       ,
  numerator-bot-sep   = 0 pt       ,
  numerator-top-sep   = \c_max_dim ,
  scale-factor        = 1          ,
  scale-relative      = false      ,
  scaling             = true       ,
  slash-right-mkern   = 0mu        ,
  slash-left-mkern    = 0mu
}
%    \end{macrocode}
%    
% Default Computer Modern setup. Far from optimal, but better than
% nothing.
%    \begin{macrocode}
\DeclareInstance { xfrac } { cmr } { text } {
  denominator-bot-sep = 0 pt    ,
  numerator-top-sep   = 0.2 ex  ,
  slash-left-kern     = -0.1 em ,
  slash-right-kern    = -0.1 em 
}
\DeclareInstance { xfrac } { cmss } { text } {
  denominator-bot-sep = 0 pt    ,
  numerator-top-sep   = 0.2 ex  ,
  slash-left-kern     = -0.1 em ,
  slash-right-kern    = -0.1 em 
}
\DeclareInstance { xfrac } { cmtt } { text } {
  denominator-bot-sep = 0 pt    ,
  numerator-top-sep   = 0.2 ex  ,
  slash-left-kern     = -0.1 em ,
  slash-right-kern    = -0.1 em 
}
%    \end{macrocode}
%    
% We can do better for the Computer Modern fonts. For cmr and cmss 
% we choose Times, and for cmtt use Palatino.
%    \begin{macrocode}
\bool_if:NF \l_xfrac_cm_std_bool
  {
    \DeclareInstance { xfrac } { cmr } { text }
      { slash-symbol-font = ptm }
    \DeclareInstance { xfrac } { cmss } { text }
      { slash-symbol-font = ptm }
    \DeclareInstance { xfrac } { cmtt } { text }
      { slash-symbol-font = ppl }
  }
%    \end{macrocode}
%    
% Things works slightly better with Latin Modern.
%    \begin{macrocode}
\DeclareInstance { xfrac } { lmr } { text } {
  denominator-bot-sep = 0 pt     ,
  numerator-top-sep   = 0.1 ex   ,
  slash-left-kern     = -0.15 em ,
  slash-right-kern    = -0.15 em 
}
\DeclareInstance { xfrac } { lmss } { text } {
  denominator-bot-sep = 0 pt     ,
  numerator-top-sep   = 0 pt     ,
  slash-left-kern     = -0.15 em ,
  slash-right-kern    = -0.15 em 
}
\DeclareInstance { xfrac } { lmtt } { text } {
  denominator-bot-sep = 0 pt     ,
  numerator-top-sep   = 0 pt     ,
  slash-left-kern     = -0.15 em ,
  slash-right-kern    = -0.15 em 
}
%    \end{macrocode}
%
%\subsection{Messages}
%
% Just the one.    
%    \begin{macrocode}
\msg_new:nnnn { xfrac } { over-specified-numerator-sep }
  {You have specified both numerator-top-sep and numerator-bot-sep}
  {I will pretend that you didn't specify either of them}
%    \end{macrocode}    
%    
%\subsection{The user command}
%
% Currently there is just a single user command. \cs{sfrac} takes
% two mandatory arguments: numerator and denominator. It can take an
% optional argument between the mandatory specifying the separator
% like this:
%\begin{verbatim}
%  \sfrac{7}[/]{12}
%\end{verbatim}
% It also has an optional argument that comes before the first
% mandatory argument. If used it will use that instance instead of
% the auto-detected one, so a user who has defined the instance
% `cmr2' may use
%\begin{verbatim}
%  \sfrac[cmr2]{7}{12}
%\end{verbatim}
% and get the settings from `cmr2' instead of the settings of 
% the current font family.
%    \begin{macrocode}
\NewDocumentCommand \sfrac { o m o m } {
  \mode_if_math:TF 
    {
      \IfInstanceExistTF { xfrac } { mathfam \number \fam }
        { \UseInstance { xfrac } { mathfam \number \fam } }
        { \UseInstance { xfrac } { mathdefault } }
      {#2} {#3} {#4}  
    }
    {
      \IfInstanceExistTF { xfrac } {#1}
        { \UseInstance { xfrac } {#1} }
        { 
          \IfInstanceExistTF { xfrac } { \f@family }
            { \UseInstance { xfrac } { \f@family } }
            { \UseInstance { xfrac } { default } }
        }
      {#2} {#3} {#4}   
    }
}
%    \end{macrocode}
%        (quote the arguments according to the demands of your shell)
%
% Documentation:
%    (a) If xfrac.drv is present:
%           latex xfrac.drv
%    (b) Without xfrac.drv:
%           latex xfrac.dtx; ...
%    The class ltxdoc loads the configuration file ltxdoc.cfg
%    if available. Here you can specify further options, e.g.
%    use A4 as paper format:
%       \PassOptionsToClass{a4paper}{article}
%
%    Programm calls to get the documentation (example):
%       pdflatex xfrac.dtx
%       makeindex -s gind.ist xfrac.idx
%       pdflatex xfrac.dtx
%       makeindex -s gind.ist xfrac.idx
%       pdflatex xfrac.dtx
%
% Installation:
%    TDS:tex/latex/mh/xfrac.sty
%    TDS:doc/latex/mh/xfrac.pdf
%    TDS:source/latex/mh/xfrac.dtx
%
%<*ignore>
\begingroup
  \def\x{LaTeX2e}
\expandafter\endgroup
\ifcase 0\ifx\install y1\fi\expandafter
         \ifx\csname processbatchFile\endcsname\relax\else1\fi
         \ifx\fmtname\x\else 1\fi\relax
\else\csname fi\endcsname
%</ignore>
%<*install>
\input docstrip.tex
\Msg{************************************************************************}
\Msg{* Installation}
\Msg{* Package: xfrac 2009/08/10 v0.3 textstyle fractions (MH)}
\Msg{************************************************************************}

\keepsilent
\askforoverwritefalse

\preamble

This is a generated file.

Copyright (C) 2004, 2008-2010 by Morten Hoegholm <mh.ctan@gmail.com>

This work may be distributed and/or modified under the
conditions of the LaTeX Project Public License, either
version 1.3c of this license or (at your option) any later
version. The latest version of this license is in
   http://www.latex-project.org/lppl.txt

This work has the LPPL maintenance status "maintained".

This Current Maintainer of this work is Morten Hoegholm.

This work consists of the main source file xfrac.dtx
and the derived files
   xfrac.sty, xfrac.pdf, xfrac.ins, xfrac.drv.

\endpreamble

\generate{%
  \file{xfrac.ins}{\from{xfrac.dtx}{install}}%
  \file{xfrac.drv}{\from{xfrac.dtx}{driver}}%
  \usedir{tex/latex/mh}%
  \file{xfrac.sty}{\from{xfrac.dtx}{package}}%
}

\obeyspaces
\Msg{************************************************************************}
\Msg{*}
\Msg{* To finish the installation you have to move the following}
\Msg{* file into a directory searched by TeX:}
\Msg{*}
\Msg{*     xfrac.sty}
\Msg{*}
\Msg{* To produce the documentation run the file `xfrac.drv'}
\Msg{* through LaTeX.}
\Msg{*}
\Msg{* Happy TeXing!}
\Msg{*}
\Msg{************************************************************************}

\endbatchfile
%</install>
%<*ignore>
\fi
%</ignore>
%<*driver>
\NeedsTeXFormat{LaTeX2e}
\ProvidesFile{xfrac.drv}%
  [2010/02/02 v0.3 Text style fractions]
\RequirePackage{fix-cm}
\documentclass{ltxdoc}


\usepackage{xfrac}
\usepackage{nicefrac}
\usepackage[latin1]{inputenc}
\usepackage[T1]{fontenc}
\makeatletter
\newenvironment{TemplateInterfaceDescription}[1]
  {\subsection{The Template Type `#1'}%
   \begingroup
   \@beginparpenalty\@M
   \description
   \def\TemplateArgument##1##2{\item[Arg: ##1]##2\par}%
   \def\TemplateSemantics{\enddescription\endgroup
       \subsubsection*{Semantics:}}%
  }
  {\par\bigskip}

\newenvironment{TemplateDescription}[2]
  {\subsection{The Template `#2' (type #1)}%
   \subsubsection*{Attributes:}%
   \begingroup
   \@beginparpenalty\@M
   \description
   \def\TemplateKey##1##2##3##4{\item[##1 (##2)]##3%
     \ifx\TemplateKey##4\TemplateKey\else
%         \hskip0ptplus3em\penalty-500\hskip 0pt plus 1filll Default:~##4%
         \hfill\penalty500\relax\qquad \hbox{}\hfill Default:~##4%
         \nobreak\hskip-\parfillskip\hskip0pt\relax
     \fi
     \par}%
   \def\TemplateSemantics{\enddescription\endgroup
       \subsubsection*{Semantics \& Comments:}}%
  }
  {\par\bigskip}

\renewcommand*\MakePrivateLetters{\makeatletter\catcode`\_=11\relax}
\makeatother

\providecommand*\eTeX{\ensuremath{\varepsilon}-\kern-.125em\TeX}
\providecommand*\LaTeXplus{\LaTeXe$*$}
\providecommand*\key[1]{\textbf{#1}}
\providecommand*\pkg[1]{\textsf{#1}}
\newcommand*\switch[2]{{\fontfamily{#1}\selectfont #2}}

\usepackage{lmodern}
\begin{document}
  \catcode`\_=12
  \DocInput{xfrac.dtx}
\end{document}
%</driver>
% \fi
%
%  \GetFileInfo{xfrac.drv}
%  \CheckSum{249}
%  
%  \changes{v0.3}{2010/02/02}{Update to new \pkg{xtemplate} system}
%  \changes{v0.3}{2010/02/02}{Include custom values for
%    \pkg{lmodern}}
%
%  \changes{v0.2a}{2009/08/10}{Update for new version of expl3}
%
%  \changes{v0.2}{2008/08/04}{Require expl3 and get rid of .ins file}
%  \changes{v0.2}{2008/08/04}{Move to macros/latex/contrib/mh on CTAN}
%
%  \changes{v0.11}{2004/05/08}{Consistent naming}
%
%  \changes{v0.10a}{2004/04/06}{Added a dependency of the latest
%  \LaTeX{} release}
%  \changes{v0.10a}{2004/04/06}{First version on \textsc{ctan}}
%
%  \changes{v0.10}{2004/04/05}{First publicly available prototype
%  implementation}
%
%  \title{The \pkg{xfrac} package\thanks{This file has version number
%  \fileversion, last revised \filedate.}}
%
%  \author{Morten H\o gholm}
%  \date{\filedate}
%
%  \maketitle
%
%  \begin{abstract}
%  This package uses a template interface to produce nicely looking
%  \emph{split level} fractions like \nicefrac{7}{9}\ldots{}
%  ehrm\ldots{} I mean \sfrac{7}{9}.
%  \end{abstract}
%
%  \tableofcontents
%
%  \section{User Interface}
%
%  The \pkg{xfrac} package defines a document command \cs{sfrac}
%  with the following syntax:
%  \begin{quote}
%  \cs{sfrac}\oarg{instance}\marg{num}\oarg{sep}\marg{denom}
%  \end{quote}
%  Let's show a few examples:
%  \begin{verbatim}
%  \sfrac{1}{2},  $\sfrac{1}{2}$,
%  $\mathbf{3\times\sfrac{1}{2}}$
%  \quad \fontfamily{ppl}\selectfont Palatino: \sfrac{1}{2}
%  \quad \fontfamily{ptm}\selectfont Times: \sfrac{1}{2}
%  \end{verbatim}
%  \begin{quote}
%  \sfrac{1}{2},  $\sfrac{1}{2}$, $\mathbf{3\times\sfrac{1}{2}}$
%  \quad \fontfamily{ppl}\selectfont Palatino: \sfrac{1}{2}
%  \quad \fontfamily{ptm}\selectfont Times: \sfrac{1}{2}
%  \end{quote}
%  You'll notice something interesting: Not only does the \cs{sfrac}
%  command work as it should in math mode, it also gets the job done
%  for other fonts as well.
%
%
%  \section{A Bit of History}
%
%  \subsection{The Past}
%
%  One of the first exercises in \emph{The \TeX Book} is to design a
%  macro for split level fractions. The solution presented is fairly
%  simple, using a \emph{virgule} (a slash) for separating the two
%  components. It looks okay because the text font and math font of
%  Computer Modern look almost identical.
%
%  The proper symbol to use instead of the virgule is a \emph{solidus}
%  which does not exist in Computer Modern. It is however available in
%  the European Computer Modern fonts, but I'll get back to that.
%
%  \subsection{The Present}
%
%
%  The most common way to produce split level fractions within \LaTeX\
%  is by means of the \pkg{nicefrac} package. Part of the reason it
%  has found widespread use is due to the strange design of the
%  built-in text fractions of the EC fonts, which look like this:
%  \textonehalf. The package is very simple to use but there are a few
%  issues:
% \begin{itemize}
%  \item It uses the virgule instead of the solidus.
%  \item Font size of numerator and denominator is bigger than in the
%    built-in symbol. Compare Palatino: \switch{ppl}{\nicefrac{1}{2}}
%    vs. \switch{ppl}{\textonehalf }.
%  \item It doesn't correct for fonts using text figures such as in the
%    \pkg{eco} package. Compare \switch{cmor}{\nicefrac{1}{2}} and
%    \switch{cmor}{\nicefrac{8}{9}}.
%  \item In math mode, it doesn't always pick up the correct math
%    alphabet.
% \end{itemize}
% In short: \pkg{nicefrac} doesn't attempt to be the answer to
% everything and so this is not a criticism of the package. It works
% quite well for Computer Modern which was pretty much what was widely
% available at the time it was developed. Users these days, however,
% have a choice of many fonts when they write their documents.
%
%
%
%
%
%  \subsection{The Future}
%
%  Fonts are wildly different; one macro that works fine for Computer
%  Modern obviously doesn't work well at all in Palatino. For one we
%  have to make the separator symbol configurable, and we need to
%  take care of several details as well: font scaling of the
%  numerator/denominator pair (ND), font selection of ND etc. If we
%  are to have a single package for this in the future\footnote{As
%  this is intended to be about the future, the \pkg{xfrac} package
%  requires the \eTeX{} extensions.} we have to define a totally
%  generic interface for the fraction commands and then adjust
%  parameters depending on the current font. What you see in this
%  prototype implementation of \pkg{xfrac} is just that.
%
%
%  \section{Advanced User Interface}
%
%
%  \subsection{Text mode}
%
%  The usual problem in text mode has a name: Computer Modern. The
%  solidi of all the Computer Modern fonts leave a lot to be desired,
%  although things are potentially looking better as the Latin Modern
%  fonts are becoming more stable and widespread. As long as the
%  default fonts are Computer Modern variants we must however work
%  around this. One idea that comes to mind is to see what happens
%  when you use a solidus from another font instead. Let's try with
%  Times:
%  \begin{quote}
%  \DeclareInstance{xfrac}{cmr2}{text}{
%    slash-symbol-font = ptm,
%  }
%  ``You take \sfrac[cmr2]{1}{2} cup of sugar, \ldots''
%  \end{quote}
%  That looks quite good actually, so it was probably very difficult
%  to obtain that result. Nope, it was extremely easy---if you happen
%  to know about \emph{instances}:
%  \begin{verbatim}
%  \DeclareInstance{xfrac}{cmr}{text}{
%    slash-symbol-font = ptm,
%  }
%  \end{verbatim}
%  So we define an instance with the name `cmr' from the template
%  `text' which in turn is of template type `xfrac'. You'll notice
%  the `cmr' is also the name of the font family for Computer Modern
%  Roman and the reasoning behind is that every font family should
%  have it's own settings, and if a document command is to work well
%  in that scheme, letting it use the name of the current font family
%  seems like a good idea. Thus the \cs{sfrac} command checks to see
%  whether an instance with same name as the current font family
%  exists and uses it if the test is true; otherwise the default
%  setting is used. Here we defined the instance to be used for the
%  font family `cmr' and just told it to use the Times font for
%  typesetting the slash symbol which turns out to be a solidus by
%  default.
%
%  The option \texttt{cm-recommended} which is loaded by default uses
%  the Times solidus for Computer Modern Roman and Computer Modern
%  Sans Serif and the Palatino solidus for Computer Modern Typewriter
%  Type. This looks quite good. Should you however not want this you
%  can use the option \texttt{cm-standard} which produces somewhat
%  acceptable results using Computer Modern exclusively.
%
%  So what about old style figures? If you use the \pkg{eco}
%  package you might define an instance similar to this (`cmor' is
%  the name of the roman font activated by \pkg{eco}):
%   \DeclareInstance{xfrac}{cmor}{text}{
%    slash-symbol-font = ptm,
%    numerator-font    = cmr,
%    denominator-font  = cmr,
%  }
%  \begin{verbatim}
%  \DeclareInstance{xfrac}{cmor}{text}{
%    slash-symbol-font = ptm,
%    numerator-font    = cmr,
%    denominator-font  = cmr,
%  }
%  \end{verbatim}
%  We also use regular Computer Modern Roman for typesetting ND, so
%  we end up with \switch{cmor}{\sfrac{1}{2}} and
%  \switch{cmor}{\sfrac{8}{9}} instead of
%  \switch{cmor}{\nicefrac{1}{2}} and \switch{cmor}{\nicefrac{8}{9}}.
%  Much better.
%
%  There are also situations where other tricks are useful. If you
%  don't have the inferior and superior figures available in a font,
%  or the font doesn't have a wider design for small font sizes, you
%  can cheat by manually scaling the ND-pair. I got nice results for
%  Adobe's Stempel Garamond (with small caps and old style figures)
%  with the following setup:
%  \begin{verbatim}
%  \DeclareInstance{xfrac}{pegj}{text}{
%    numerator-font   = pegx,
%    denominator-font = pegx,
%    scale-factor     = .9,
%    h-scale          = 1.1,
%  }
%  \end{verbatim}
%  We use the font family `pegx' (Stempel Garamond with real small
%  caps) for typesetting the ND-pair. Additionally the key
%  \key{scale-factor} specifies that the font size used for the
%  ND-pair should be $0.9$ of the height of the solidus, and the key
%  \key{h-scale} specifies that the ND-pair should be scaled an extra
%  10\% horizontally.
%
%  Should you be so fortunate the have a font with inferior and
%  superior figures like in the Monotype Janson example from Philipp
%  Lehman's excellent \emph{The Font Installation Guide}. In that
%  example Philipp defines the font families `mjn0' for the inferior
%  figures and `mjn1' for the superior. Thus to get the \cs{sfrac}
%  command to use them on the fly for the font family `mjnj' (Janson,
%  old style figures) we would say
%  \begin{verbatim}
%  \DeclareInstance{xfrac}{mjnj}{text}{
%    numerator-font      = mjn1,
%    denominator-font    = mjn0,
%    scaling             = false,
%    numerator-bot-sep   = 0pt,
%    denominator-bot-sep = 0pt,
%  }
%  \end{verbatim}
%  I think this example is a very clean way to do it. An alternative
%  approach could be to use the keys \key{numerator-format} and
%  \key{denominator-format} to process the arguments and let them
%  determine what to do.
%
%  As a side note Harald Harders was so kind to test it, and it
%  \emph{does} actually work---I hadn't tested it myself.
%
%  \subsection{Math Mode}
%
%  In math mode the choices are a lot fewer because first of all
%  \TeX{} comes with a built-in limitation of 16 math families.
%  Additionally we will not need a solidus for typesetting split
%  fractions in math, as tradition is to use a virgule instead. We
%  define the basic `mathdefault' instance to simply use the math
%  family in use when the instance is run. So if we're in normal math
%  like |$\sfrac{7}{9}$| we simply get family~$-1$. If we're inside a
%  \cs{mathbf} we're in family~$4$ (in the standard setup at least),
%  and so the fraction is typeset with the same math family. Simple,
%  isn't?
%
%  You can also declare instances for the math families, but I really
%  don't see why you would. If you do then name them according to the
%  scheme `mathfam\meta{N}', where \meta{N} is the family number, and
%  only do it if you \emph{really} know how to set up math fonts.
%  That is, if \cs{DeclareMathAlphabet} is unbeknownst to you, then
%  just don't go there.
%
%
%  Another example: If we want \cs{sfrac} to produce split fractions
%  without doing anything at all, we can choose the collection
%  `plainmath', which is defined as
%  \begin{verbatim}
%  \DeclareCollectionInstance{plainmath}{xfrac}{mathdefault}{math}{
%    denominator-bot-sep = 0pt,
%    numerator-bot-sep   = 0pt,
%    numerator-top-sep   = \c_max_dim,
%    scaling             = false,
%    slash-right-mkern   = 0mu,
%    slash-left-mkern    = 0mu,
%  }
%  \end{verbatim}
%  This creates an alternative version of the instance `mathdefault'
%  with settings as specified by the keys. In the default math setup
%  \key{numerator-top-sep} is set to 0~pt, and here we set
%  \key{numerator-bot-sep} to 0~pt as well, so in order to avoid
%  over-specification (and an error message) we must set
%  \key{numerator-top-sep} to \cs{c_max_dim}. We activate (obeying
%  normal scoping rules) it with:
%  \begin{verbatim}
%  \UseCollection{xfrac}{plainmath}
%  \end{verbatim}
%  Then |$\sfrac{8}{13}$| produces\begingroup
%    \UseCollection{xfrac}{plainmath}
%    $\sfrac{8}{13}$ and just typing |$8/13$| gives the same result:
%    $8/13$.
%  \endgroup
%
%
%  \section{The Template Interface}
%
%  \begin{TemplateInterfaceDescription}{xfrac}
%
%  \TemplateArgument{1}
%      {The numerator}
%
%  \TemplateArgument{2}
%       {The separator}
%
%   \TemplateArgument{3}
%      {The denominator}
%
%  \TemplateSemantics
%
%  Typesets arguments 1 and 3 separated by argument 2, which in text
%  mode by default is a \emph{solidus}. This is taken from
%  \pkg{textcomp} where it is denoted \cs{textfractionsolidus}.
%  This is the character used for the ready made split level
%  fractions such as \textonehalf---except in the (European) Computer
%  Modern fonts. In math mode a \emph{virgule} is used instead as
%  this is more appropriate and it is always available in the math
%  fonts. The solidus is a text symbol only.
%
%  \end{TemplateInterfaceDescription}
%
%  \begin{TemplateDescription}{xfrac}{text}
%
%  \TemplateKey{numerator-font}{tokenlist}
%    {Font family specification to use for the numerator.}
%    {\cs{f@family}}
%
%  \TemplateKey{numerator-format}{function 1 arg}
%    {Action to be taken on the numerator.}
%    {Process argument unchanged}
%
%  \TemplateKey{slash-symbol}{tokenlist}
%    {The separator symbol. If not specified the default value will be
%    used instead.}
%    {Solidus (\cs{textfractionsolidus})}
%
%  \TemplateKey{slash-symbol-font}{tokenlist}
%    {Font family specification to use for the separator symbol.}
%    {\cs{f@family}}
%
%  \TemplateKey{slash-symbol-format}{function 1 arg}
%    {Action to be taken on the separator symbol.}
%    {Process argument unchanged}
%
%  \TemplateKey{denominator-font}{tokenlist}
%    {Font family specification to use for the denominator.}
%    {\cs{f@family}}
%
%  \TemplateKey{denominator-format}{function 1 arg}
%    {Action to be taken on the denominator.}
%    {Process argument unchanged}
%
%  \TemplateKey{h-scale}{tokenlist}
%    {Factor by which the numerator and denominator should be
%    horizontally scaled. It should only be used if the real superior
%    and inferior fonts are not available. For instance Stempel
%    Garamond looks excellent if scaled 10\% extra horizontally, i.e.,
%    by a factor of 1.1.}
%    {1}
%
%  \TemplateKey{v-scale}{tokenlist}
%    {Same as \key{h-scale} only vertically. Probably not of much use
%    but added for completetion.}
%    {1}
%
%  \TemplateKey{scale-factor}{tokenlist}
%    {Fraction of the size of \key{slash-symbol}. Used for setting the
%    font size of numerator and denominator. Usually a value of app.\
%    \sfrac{5}{6} produces fine results. It should only be used if the
%    real superior and inferior fonts are not available. As an example
%    Stempel Garamond looks better if the factor is 0.9.}
%    {0.83333}
%
%  \TemplateKey{scale-relative}{choice}
%    {If set to `true' the font size of the numerator and denominator
%    is scaled with respect to the height of the \key{slash-symbol}. If
%    set to `false' the font is scaled with respect to the total height
%    of the \key{slash-symbol}.}
%    {true}
%
%  \TemplateKey{scaling}{choice}
%    {If set to `true' the fonts are allowed to scale. If set to
%    `false' they are not. See the `Janson' example for an application.}
%    {true}
%
%  \TemplateKey{numerator-top-sep}{length}
%    {Dimension specifying the space between the top of the
%    \key{slash-symbol} and the top of the numerator. If not specified,
%    the depth of the solidus will be used, because this value will
%    make the fraction look even.}
%    {Unspecified}
%
%  \TemplateKey{numerator-bot-sep}{length}
%    {Dimension specifying the lift of the numerator from the
%    baseline.}
%    {Unspecified}
%
%  \TemplateKey{denominator-bot-sep}{length}
%    {Dimension specifying the lift of the denominator from the
%    baseline.}
%    {Unspecified}
%
%  \TemplateKey{slash-right-kern}{length}
%    {Dimension specifying the kerning between the \key{slash-symbol}
%    and the numerator.}
%    {\texttt{0pt}}
%
%  \TemplateKey{slash-left-kern}{length}
%    {Dimension specifying the kerning between the \key{slash-symbol}
%    and the denominator.}
%    {\texttt{0pt}}
%
%  \TemplateKey{math-mode}{choice}
%    {Are we in math mode or not?}
%    {false}
%
%  \TemplateKey{phantom}{tokenlist}
%    {A character that suits the common cases. As we would mostly want
%    to use numbers in text mode we choose a `tall' number, while in
%    math it is somewhat different.}
%    {8}
%
%  \TemplateSemantics
%
%  This template is also the foundation for the `math' template. The
%  keys \key{slash-right-mkern} and \key{slash-left-mkern} can only
%  be used in math mode and are not shown here.
%
%  \end{TemplateDescription}
%
%
%  \begin{TemplateDescription}{xfrac}{math}
%
%  \TemplateKey{numerator-font}{tokenlist}
%    {Font family specification to use for the numerator.}
%    {\cs{number}\cs{fam}}
%
%  \TemplateKey{slash-symbol}{tokenlist}
%    {The separator symbol. If not specified the default value will be
%    used instead.}
%    {Virgule ($/$)}
%
%  \TemplateKey{slash-symbol-font}{tokenlist}
%    {Font family specification to use for the separator symbol.}
%    {\cs{number}\cs{fam}}
%
%  \TemplateKey{denominator-font}{tokenlist}
%    {Font family specification to use for the denominator.}
%    {\cs{number}\cs{fam}}
%
%  \TemplateKey{scale-factor}{tokenlist}
%    {Fraction of the size of \key{slash-symbol}. In math mode we
%    cannot rely on the fonts to be able to scale, but giving a default
%    scale of 0.7 fits into the regular size changing scheme---the
%    default scheme has values $(D,T,S,SS)=(1,1,0.7,0.5)$ whereas we
%    with a default \key{scale-factor} of 0.7 get $(1,1,0.7,0.49)$.
%    That's close enough.}
%    {0.7}
%
%  \TemplateKey{scale-relative}{choice}
%    {If set to `true' the font size of the numerator and denominator
%    is scaled with respect to the height of the \key{slash-symbol}. If
%    set to `false' the font is scaled with respect to the total height
%    of the \key{slash-symbol}.}
%    {false}
%
%  \TemplateKey{scaling}{choice}
%    {If set to `true' the fonts are allowed to scale. If set to
%    `false' they are not. See the `plainmath' example for an application.}
%    {true}
%
%  \TemplateKey{numerator-top-sep}{length}
%    {Dimension specifying the space between the top of the
%    \key{slash-symbol} and the top of the numerator. If not specified,
%    the depth of the virgule will be used, because this value will
%    make the fraction look even.}
%    {\texttt{0pt}}
%
%  \TemplateKey{denominator-bot-sep}{length}
%    {Dimension specifying the lift of the denominator from the
%    baseline.}
%    {\texttt{0pt}}
%
%  \TemplateKey{slash-right-mkern}{tokenlist}
%    {Same as \key{slash-right-kern} but for math mode only and should
%    be specified in \texttt{mu} units. This is because \pkg{calc}
%    can't use mu-expressions.}
%    {\texttt{-2mu}}
%
%  \TemplateKey{slash-left-mkern}{tokenlist}
%    {Same as \key{slash-left-kern} but for math mode only and should
%    be specified in \texttt{mu} units. This is because \pkg{calc}
%    can't use mu-expressions.}
%    {\texttt{-1mu}}
%
%  \TemplateKey{math-mode}{choice}
%    {Are we in math mode or not?}
%    {true}
%
%  \TemplateKey{phantom}{tokenlist}
%    {A character that suits the common cases. In math we have a high
%    risk of using a parenthesis, so we choose that. Text mode is
%    another story.}
%    {(}
%
%  \TemplateSemantics
%
%  This template is a restricted version of the `text' template. Only
%  the keys that are different from the `text' template are shown
%  here. Also bear in mind that the attributes \key{slash-left-kern}
%  and \key{slash-right-kern} have no meaning in this template.
%
%  \end{TemplateDescription}
% 
%  \StopEventually{}
%
%\section{Implementation}
%
%    \begin{macrocode}
%<*package>
%    \end{macrocode}
%       
% The usual lead-off: provides an experimental package!
%    \begin{macrocode}
\RequirePackage{expl3}[2009/08/05]
\ProvidesExplPackage{xfrac}{2010/02/02}{0.3}{Text fractions}
%    \end{macrocode}
%
% Some support is needed: a bit wider than the normal \pkg{xpackage}
% stuff, but not by much.
%    \begin{macrocode}
\RequirePackage { amstext , graphicx , l3keys2e , textcomp , xtemplate }
%    \end{macrocode}   
%
%\begin{macro}{\l_xfrac_cm_std_bool}
% There is one option to support.
%    \begin{macrocode}
\keys_define:nn { xfrac } {
  cm-recommended .choice:,
  cm-recommended /
    false        .code:n     =
      { \bool_set_true:N \l_xfrac_cm_std_bool },
  cm-recommended /
    true         .code:n     =
      { \bool_set_false:N \l_xfrac_cm_std_bool },
  cm-recommended .default:n  = { true },    
  cm-standard    .bool_set:N = \l_xfrac_cm_std_bool
}
\ProcessKeysOptions { xfrac }
%    \end{macrocode}
%\end{macro}
%
%\begin{macro}{\l_xfrac_slash_box}
%\begin{macro}{\l_xfrac_tmp_box}
% In keeping with the \LaTeX3 philosophy, rather than use generic 
% scratch boxes and get confused, \pkg{xfrac} reserves its own named
% working space.
%    \begin{macrocode}
\box_new:N \l_xfrac_slash_box
\box_new:N \l_xfrac_tmp_box
%    \end{macrocode}
%\end{macro}
%\end{macro}
%
%\begin{macro}{\xfrac_tmp:w}
% Used for the raised boxes: weird as it does not take an argument
% but the \cs{raisebox} does.
%    \begin{macrocode}
\cs_new:Npn \xfrac_tmp:w { }
%    \end{macrocode}
%\end{macro}
%
%\subsection{Initialisation of variables}
%
% Variables used in templates have to be set up: there is not
% much to say about these, other than that they must exist.
%    
%\begin{macro}{\l_xfrac_denominator_bot_sep_dim}
%\begin{macro}{\l_xfrac_numerator_bot_sep_dim}
%\begin{macro}{\l_xfrac_numerator_top_sep_dim}
%\begin{macro}{\l_xfrac_slash_left_sep_dim}
%\begin{macro}{\l_xfrac_slash_right_sep_dim}
% Fixed lengths.
%    \begin{macrocode}
\dim_new:N \l_xfrac_denominator_bot_sep_dim
\dim_new:N \l_xfrac_numerator_bot_sep_dim
\dim_new:N \l_xfrac_numerator_top_sep_dim
\dim_new:N \l_xfrac_slash_left_sep_dim
\dim_new:N \l_xfrac_slash_right_sep_dim
%    \end{macrocode}
%\end{macro}
%\end{macro}
%\end{macro}
%\end{macro}
%\end{macro}
%
%\begin{macro}{\l_xfrac_denominator_font_tl}
%\begin{macro}{\l_xfrac_hscale_tl}
%\begin{macro}{\l_xfrac_numerator_font_tl}
%\begin{macro}{\l_xfrac_phantom_tl }
%\begin{macro}{\l_xfrac_scale_factor_tl}
%\begin{macro}{\l_xfrac_slash_left_msep_tl}
%\begin{macro}{\l_xfrac_slash_right_msep_tl}
%\begin{macro}{\l_xfrac_slash_symbol_tl}
%\begin{macro}{\l_xfrac_slash_symbol_font_tl}
%\begin{macro}{\l_xfrac_vscale_tl}
% Token lists, which include floating-point numbers and math(s)
% skips.
%    \begin{macrocode}
\tl_new:N \l_xfrac_denominator_font_tl
\tl_new:N \l_xfrac_hscale_tl
\tl_new:N \l_xfrac_numerator_font_tl
\tl_new:N \l_xfrac_phantom_tl 
\tl_new:N \l_xfrac_scale_factor_tl
\tl_new:N \l_xfrac_slash_left_msep_tl
\tl_new:N \l_xfrac_slash_right_msep_tl
\tl_new:N \l_xfrac_slash_symbol_tl
\tl_new:N \l_xfrac_slash_symbol_font_tl
\tl_new:N \l_xfrac_vscale_tl
%    \end{macrocode}
%\end{macro}
%\end{macro}
%\end{macro}
%\end{macro}
%\end{macro}
%\end{macro}
%\end{macro}
%\end{macro}
%\end{macro}
%\end{macro}
%
%\begin{macro}{\xfrac_fontscale:}
%\begin{macro}{\xfrac_math:n}
%\begin{macro}{\xfrac_denominator_font_change:}
%\begin{macro}{\xfrac_denominator_format:n}
%\begin{macro}{\xfrac_numerator_font_change:}
%\begin{macro}{\xfrac_numerator_format:n}
%\begin{macro}{\xfrac_relscale:}
%\begin{macro}{\xfrac_slash_symbol_font_change:}
%\begin{macro}{\xfrac_slash_symbol_format:n}
%\begin{macro}{\xfrac_text_or_math:n}
% Functions, either things which are calculated `on the fly'
% (no argument required) or are functions taking one argument in the
% code.
%    \begin{macrocode}
\cs_new_nopar:Npn \xfrac_fontscale:
\cs_new:Npn \xfrac_math:n #1 { }
\cs_new_nopar:Npn \xfrac_denominator_font_change: { }
\cs_new:Npn \xfrac_denominator_format:n #1 { }
\cs_new_nopar:Npn \xfrac_numerator_font_change: { }
\cs_new:Npn \xfrac_numerator_format:n #1 { }
\cs_new_nopar:Npn \xfrac_relscale: { }
\cs_new_nopar:Npn \xfrac_slash_symbol_font_change: { }
\cs_new:Npn \xfrac_slash_symbol_format:n #1 { }
\cs_new:Npn \xfrac_text_or_math:n #1 { }
%    \end{macrocode}
%\end{macro}
%\end{macro}
%\end{macro}
%\end{macro}
%\end{macro}
%\end{macro}
%\end{macro}
%\end{macro}
%\end{macro}
%\end{macro}
%
%\subsection{The template}
%
% There is only one object type in \pkg{xfrac}, rather unimaginatively
% named \texttt{xfrac}.
%    \begin{macrocode}
\DeclareObjectType { xfrac } { 3 }
%    \end{macrocode}
%
% A single template interface is used for both text and math(s), which
% does make a few things a little complex later.
%    \begin{macrocode}
\DeclareTemplateInterface { xfrac } { text } { 3 } {
  denominator-bot-sep : length     = \c_max_dim           ,
  denominator-font    : tokenlist  = \f@family            ,
  denominator-format  : function 1 = #1                   ,
  h-scale             : tokenlist  = 1                    ,
  math-mode           : choice { false , true }
                                   = false                ,
  numerator-font      : tokenlist  = \f@family            ,
  numerator-format    : function 1 = #1                   ,
  numerator-bot-sep   : length     = \c_max_dim           ,
  numerator-top-sep   : length     = \c_max_dim           ,
  phantom             : tokenlist  = 8                    ,
  scale-factor        : tokenlist  = 0.83333              ,
  scale-relative      : choice { false , true }
                                   = true                 ,
  scaling             : choice { false , true }
                                   = true                 ,
  slash-left-kern     : length     = 0 pt                 ,
  slash-left-mkern    : tokenlist  = -2 mu                ,
  slash-right-kern    : length     = 0 pt                 ,
  slash-right-mkern   : tokenlist  = -1 mu                ,
  slash-symbol        : tokenlist  = \textfractionsolidus ,
  slash-symbol-font   : tokenlist  = \f@family            ,
  slash-symbol-format : function 1 = #1                   ,
  v-scale             : tokenlist  = 1                    ,
}
%    \end{macrocode}
%   
% Most of the variable binding is quite simple: of course, the choices
% are a little more complicated. That is particularly true where 
% these have to set up `on the fly' functions.   
%    \begin{macrocode}
\DeclareTemplateCode { xfrac } { text } { 3 } 
  {
    denominator-bot-sep = \l_xfrac_denominator_bot_sep_dim ,
    denominator-font    = \l_xfrac_denominator_font_tl     ,
    denominator-format  = \xfrac_denominator_format:n      ,
    h-scale             = \l_xfrac_hscale_tl               ,
    math-mode           =
      {
        false = \cs_set_eq:NN \xfrac_math:n \use:n,
        true  = \cs_set_eq:NN \xfrac_math:n \ensuremath
      },
    numerator-font      = \l_xfrac_numerator_font_tl       ,
    numerator-format    = \xfrac_numerator_format:n        ,
    numerator-bot-sep   = \l_xfrac_numerator_bot_sep_dim   ,
    numerator-top-sep   = \l_xfrac_numerator_top_sep_dim   ,
    phantom             = \l_xfrac_phantom_tl              ,
    scale-factor        = \l_xfrac_scale_factor_tl         ,
    scale-relative      =
      {
        false = 
          \cs_set_nopar:Npn \xfrac_relscale:
            {
              \dim_eval:n 
                { 
                    \box_ht:N \l_xfrac_tmp_box 
                  + \box_dp:N \l_xfrac_tmp_box
                } 
            },
        true  = 
          \cs_set_nopar:Npn \xfrac_relscale: 
            { \box_ht:N \l_xfrac_slash_box }
      },
    scaling             =
      {
        false = \cs_set_eq:NN \xfrac_fontscale: \prg_do_nothing:, 
        true  =  
          \cs_set_nopar:Npn \xfrac_fontscale:
            {
              \fontsize { \l_xfrac_scale_factor_tl \xfrac_relscale: }
                { \c_zero_dim } 
              \selectfont
            }
      },
    slash-left-kern     = \l_xfrac_slash_left_sep_dim      ,
    slash-left-mkern    = \l_xfrac_slash_left_msep_tl      ,
    slash-right-kern    = \l_xfrac_slash_right_sep_dim     ,
    slash-right-mkern   = \l_xfrac_slash_right_msep_tl     ,
    slash-symbol        = \l_xfrac_slash_symbol_tl         ,
    slash-symbol-font   = \l_xfrac_slash_symbol_font_tl    ,
    slash-symbol-format = \xfrac_slash_symbol_format:n     ,
    v-scale             = \l_xfrac_vscale_tl        
  }
%    \end{macrocode}
% The implementation part starts with applying all of the settings
% from above. The first part of the set up is then to determine
% whether the surroundings are text or math(s), and react accordingly.
%    \begin{macrocode}
  {
    \AssignTemplateKeys
    \mode_if_math:TF
      {
        \cs_set_eq:NN \xfrac_text_or_math:n \text
        \cs_set_nopar:Npx \xfrac_denominator_font_change:
          { \tex_fam:D \l_xfrac_denominator_font_tl }
        \cs_set_nopar:Npx \xfrac_numerator_font_change:
          { \tex_fam:D \l_xfrac_numerator_font_tl }
        \cs_set_nopar:Npx \xfrac_slash_symbol_font_change:
          { \tex_fam:D \l_xfrac_slash_symbol_font_tl }
      }
      {
        \cs_set_eq:NN \xfrac_text_or_math:n \mbox
        \cs_set_nopar:Npn \xfrac_denominator_font_change:
          { 
            \fontfamily { \l_xfrac_denominator_font_tl }
            \selectfont
          }
        \cs_set_nopar:Npn \xfrac_numerator_font_change:
          { 
            \fontfamily { \l_xfrac_numerator_font_tl }
            \selectfont
          }
        \cs_set_nopar:Npn \xfrac_slash_symbol_font_change:
          { 
            \fontfamily { \l_xfrac_slash_symbol_font_tl }
            \selectfont
          }
      }
%    \end{macrocode}
%\changes{v0.11a}{2004/08/24}{Added \cs{m@th}}
% Everything is now either inside \cs{text} or an \cs{mbox}, depending
% upon the surroundings. First, there are some boxes to set up.
%    \begin{macrocode}
    \xfrac_text_or_math:n 
      {
        \m@th
        \hbox_set:Nn \l_xfrac_tmp_box 
          { \xfrac_math:n { \vphantom { ( ) } } }
        \hbox_set:Nn \l_xfrac_slash_box 
          {
            \xfrac_math:n 
              { 
                \xfrac_slash_symbol_format:n 
                  {
                    \xfrac_math:n 
                      {
                        \xfrac_slash_symbol_font_change:
                        \IfNoValueTF {#2} 
                          { \l_xfrac_slash_symbol_tl } {#2}
                      }
                  }
              }
          }
%    \end{macrocode}
% Check on the numerator separator dimensions. The code starts with the
% assumption that neither has been given, as this can then be used to
% set up a default, which is also used when both values are set 
% erroneously.
%    \begin{macrocode}
        \cs_set_nopar:Npn \xfrac_tmp:w
          {
            \raisebox 
              {
                  \box_ht:N \l_xfrac_slash_box
                - \box_dp:N \l_xfrac_slash_box
                - \height
              }
          }
        \dim_compare:nNnTF 
          { \l_xfrac_numerator_top_sep_dim } = { \c_max_dim }
          {
            \dim_compare:nNnF
              { \l_xfrac_numerator_bot_sep_dim } = { \c_max_dim } 
              {
                \cs_set_nopar:Npn \xfrac_tmp:w
                  { 
                    \raisebox 
                      { \dim_use:N \l_xfrac_numerator_bot_sep_dim } 
                  }
              }
          }
          {
            \dim_compare:nNnTF
              { \l_xfrac_numerator_bot_sep_dim } = { \c_max_dim } 
                { 
                  \cs_set_nopar:Npn \xfrac_tmp:w
                    { 
                      \raisebox 
                        { 
                            \box_ht:N \l_xfrac_slash_box
                          - \dim_use:N \l_xfrac_numerator_top_sep_dim 
                          - \height
                        }
                    }
                }
                { 
                  \msg_error:nn { xfrac } 
                    { over-specified-numerator-sep }
                }
          }
%    \end{macrocode}
%  Typeset the numerator.
%    \begin{macrocode}
        \xfrac_tmp:w
          {
            \xfrac_fontscale:
            \xfrac_numerator_format:n
              {
                \scalebox { \l_xfrac_hscale_tl } [ \l_xfrac_vscale_tl ]
                  { 
                    \xfrac_math:n 
                      { 
                        \xfrac_numerator_font_change:
                        {
                          \vphantom { \l_xfrac_phantom_tl }
                          #1
                        }
                      } 
                  }
              }
          }
        \xfrac_math:n
          { % THIS IS JUST WRONG! 
            \mode_if_math:TF
              { \tex_mskip:D \l_xfrac_slash_right_msep_tl }
              { \tex_hskip:D \l_xfrac_slash_right_sep_dim }
          }  
%    \end{macrocode}
%  Typeset the separator.
%    \begin{macrocode}
        \box_use:N \l_xfrac_slash_box
        \xfrac_math:n 
          {
            \mode_if_math:TF
              { \tex_mskip:D \l_xfrac_slash_left_msep_tl }
              { \tex_hskip:D \l_xfrac_slash_left_sep_dim }
          }
%    \end{macrocode}
%  Typeset the denominator.
%    \begin{macrocode}
        \dim_compare:nNnTF 
          { \l_xfrac_denominator_bot_sep_dim } = { \c_max_dim }
          {
            \cs_set_nopar:Npn \xfrac_tmp:w
              { \raisebox { - \box_dp:N \l_xfrac_slash_box } } 
          }
          {
            \cs_set_nopar:Npn \xfrac_tmp:w
              { 
                \raisebox 
                  { \dim_use:N \l_xfrac_denominator_bot_sep_dim }
              }
          }  
        \xfrac_tmp:w
          {
            \xfrac_fontscale:
            \xfrac_denominator_format:n
              {
                \scalebox { \l_xfrac_hscale_tl } [ \l_xfrac_vscale_tl ]
                  { 
                    \xfrac_math:n 
                      { 
                        \xfrac_denominator_font_change:
                        {
                          \vphantom { \l_xfrac_phantom_tl }
                          #3
                        }
                      } 
                  }
              }
          }
      } 
  }
%    \end{macrocode}
%    
% Since math(s) and text mode are wildly different entities we define a
% separate template for each. You already saw the `text' 
% template, and here is the `math' template.
%    \begin{macrocode}
\DeclareRestrictedTemplate { xfrac } { text } { math } {
  numerator-font      = \number \fam ,
  slash-symbol        = /            ,
  slash-symbol-font   = \number \fam ,
  denominator-font    = \number \fam ,
  scale-factor        = 0.7          ,
  scale-relative      = false        ,
  scaling             = true         ,
  numerator-top-sep   = 0 pt         ,
  denominator-bot-sep = 0 pt         ,
  math-mode           = true         ,
  phantom             = (
}
%    \end{macrocode}
%    
%\subsection{The standard instances}
%
% For the default instances we just use the relevant templates with
% the default settings.
% 
%  The default `text' instance.
%    \begin{macrocode}
\DeclareInstance { xfrac } { default } { text } { }
%    \end{macrocode}
%    
%  The default `math(s)' instance.
%    \begin{macrocode}
\DeclareInstance { xfrac } { mathdefault } { math } { }
%    \end{macrocode}
%    \begin{macrocode}
\DeclareCollectionInstance { plainmath } { xfrac } { mathdefault } 
  { math }{
  denominator-bot-sep = 0 pt       ,
  numerator-bot-sep   = 0 pt       ,
  numerator-top-sep   = \c_max_dim ,
  scale-factor        = 1          ,
  scale-relative      = false      ,
  scaling             = true       ,
  slash-right-mkern   = 0mu        ,
  slash-left-mkern    = 0mu
}
%    \end{macrocode}
%    
% Default Computer Modern setup. Far from optimal, but better than
% nothing.
%    \begin{macrocode}
\DeclareInstance { xfrac } { cmr } { text } {
  denominator-bot-sep = 0 pt    ,
  numerator-top-sep   = 0.2 ex  ,
  slash-left-kern     = -0.1 em ,
  slash-right-kern    = -0.1 em 
}
\DeclareInstance { xfrac } { cmss } { text } {
  denominator-bot-sep = 0 pt    ,
  numerator-top-sep   = 0.2 ex  ,
  slash-left-kern     = -0.1 em ,
  slash-right-kern    = -0.1 em 
}
\DeclareInstance { xfrac } { cmtt } { text } {
  denominator-bot-sep = 0 pt    ,
  numerator-top-sep   = 0.2 ex  ,
  slash-left-kern     = -0.1 em ,
  slash-right-kern    = -0.1 em 
}
%    \end{macrocode}
%    
% We can do better for the Computer Modern fonts. For cmr and cmss 
% we choose Times, and for cmtt use Palatino.
%    \begin{macrocode}
\bool_if:NF \l_xfrac_cm_std_bool
  {
    \DeclareInstance { xfrac } { cmr } { text }
      { slash-symbol-font = ptm }
    \DeclareInstance { xfrac } { cmss } { text }
      { slash-symbol-font = ptm }
    \DeclareInstance { xfrac } { cmtt } { text }
      { slash-symbol-font = ppl }
  }
%    \end{macrocode}
%    
% Things works slightly better with Latin Modern.
%    \begin{macrocode}
\DeclareInstance { xfrac } { lmr } { text } {
  denominator-bot-sep = 0 pt     ,
  numerator-top-sep   = 0.1 ex   ,
  slash-left-kern     = -0.15 em ,
  slash-right-kern    = -0.15 em 
}
\DeclareInstance { xfrac } { lmss } { text } {
  denominator-bot-sep = 0 pt     ,
  numerator-top-sep   = 0 pt     ,
  slash-left-kern     = -0.15 em ,
  slash-right-kern    = -0.15 em 
}
\DeclareInstance { xfrac } { lmtt } { text } {
  denominator-bot-sep = 0 pt     ,
  numerator-top-sep   = 0 pt     ,
  slash-left-kern     = -0.15 em ,
  slash-right-kern    = -0.15 em 
}
%    \end{macrocode}
%
%\subsection{Messages}
%
% Just the one.    
%    \begin{macrocode}
\msg_new:nnnn { xfrac } { over-specified-numerator-sep }
  {You have specified both numerator-top-sep and numerator-bot-sep}
  {I will pretend that you didn't specify either of them}
%    \end{macrocode}    
%    
%\subsection{The user command}
%
% Currently there is just a single user command. \cs{sfrac} takes
% two mandatory arguments: numerator and denominator. It can take an
% optional argument between the mandatory specifying the separator
% like this:
%\begin{verbatim}
%  \sfrac{7}[/]{12}
%\end{verbatim}
% It also has an optional argument that comes before the first
% mandatory argument. If used it will use that instance instead of
% the auto-detected one, so a user who has defined the instance
% `cmr2' may use
%\begin{verbatim}
%  \sfrac[cmr2]{7}{12}
%\end{verbatim}
% and get the settings from `cmr2' instead of the settings of 
% the current font family.
%    \begin{macrocode}
\NewDocumentCommand \sfrac { o m o m } {
  \mode_if_math:TF 
    {
      \IfInstanceExistTF { xfrac } { mathfam \number \fam }
        { \UseInstance { xfrac } { mathfam \number \fam } }
        { \UseInstance { xfrac } { mathdefault } }
      {#2} {#3} {#4}  
    }
    {
      \IfInstanceExistTF { xfrac } {#1}
        { \UseInstance { xfrac } {#1} }
        { 
          \IfInstanceExistTF { xfrac } { \f@family }
            { \UseInstance { xfrac } { \f@family } }
            { \UseInstance { xfrac } { default } }
        }
      {#2} {#3} {#4}   
    }
}
%    \end{macrocode}
%        (quote the arguments according to the demands of your shell)
%
% Documentation:
%    (a) If xfrac.drv is present:
%           latex xfrac.drv
%    (b) Without xfrac.drv:
%           latex xfrac.dtx; ...
%    The class ltxdoc loads the configuration file ltxdoc.cfg
%    if available. Here you can specify further options, e.g.
%    use A4 as paper format:
%       \PassOptionsToClass{a4paper}{article}
%
%    Programm calls to get the documentation (example):
%       pdflatex xfrac.dtx
%       makeindex -s gind.ist xfrac.idx
%       pdflatex xfrac.dtx
%       makeindex -s gind.ist xfrac.idx
%       pdflatex xfrac.dtx
%
% Installation:
%    TDS:tex/latex/mh/xfrac.sty
%    TDS:doc/latex/mh/xfrac.pdf
%    TDS:source/latex/mh/xfrac.dtx
%
%<*ignore>
\begingroup
  \def\x{LaTeX2e}
\expandafter\endgroup
\ifcase 0\ifx\install y1\fi\expandafter
         \ifx\csname processbatchFile\endcsname\relax\else1\fi
         \ifx\fmtname\x\else 1\fi\relax
\else\csname fi\endcsname
%</ignore>
%<*install>
\input docstrip.tex
\Msg{************************************************************************}
\Msg{* Installation}
\Msg{* Package: xfrac 2009/08/10 v0.3 textstyle fractions (MH)}
\Msg{************************************************************************}

\keepsilent
\askforoverwritefalse

\preamble

This is a generated file.

Copyright (C) 2004, 2008-2010 by Morten Hoegholm <mh.ctan@gmail.com>

This work may be distributed and/or modified under the
conditions of the LaTeX Project Public License, either
version 1.3c of this license or (at your option) any later
version. The latest version of this license is in
   http://www.latex-project.org/lppl.txt

This work has the LPPL maintenance status "maintained".

This Current Maintainer of this work is Morten Hoegholm.

This work consists of the main source file xfrac.dtx
and the derived files
   xfrac.sty, xfrac.pdf, xfrac.ins, xfrac.drv.

\endpreamble

\generate{%
  \file{xfrac.ins}{\from{xfrac.dtx}{install}}%
  \file{xfrac.drv}{\from{xfrac.dtx}{driver}}%
  \usedir{tex/latex/mh}%
  \file{xfrac.sty}{\from{xfrac.dtx}{package}}%
}

\obeyspaces
\Msg{************************************************************************}
\Msg{*}
\Msg{* To finish the installation you have to move the following}
\Msg{* file into a directory searched by TeX:}
\Msg{*}
\Msg{*     xfrac.sty}
\Msg{*}
\Msg{* To produce the documentation run the file `xfrac.drv'}
\Msg{* through LaTeX.}
\Msg{*}
\Msg{* Happy TeXing!}
\Msg{*}
\Msg{************************************************************************}

\endbatchfile
%</install>
%<*ignore>
\fi
%</ignore>
%<*driver>
\NeedsTeXFormat{LaTeX2e}
\ProvidesFile{xfrac.drv}%
  [2010/02/02 v0.3 Text style fractions]
\RequirePackage{fix-cm}
\documentclass{ltxdoc}


\usepackage{xfrac}
\usepackage{nicefrac}
\usepackage[latin1]{inputenc}
\usepackage[T1]{fontenc}
\makeatletter
\newenvironment{TemplateInterfaceDescription}[1]
  {\subsection{The Template Type `#1'}%
   \begingroup
   \@beginparpenalty\@M
   \description
   \def\TemplateArgument##1##2{\item[Arg: ##1]##2\par}%
   \def\TemplateSemantics{\enddescription\endgroup
       \subsubsection*{Semantics:}}%
  }
  {\par\bigskip}

\newenvironment{TemplateDescription}[2]
  {\subsection{The Template `#2' (type #1)}%
   \subsubsection*{Attributes:}%
   \begingroup
   \@beginparpenalty\@M
   \description
   \def\TemplateKey##1##2##3##4{\item[##1 (##2)]##3%
     \ifx\TemplateKey##4\TemplateKey\else
%         \hskip0ptplus3em\penalty-500\hskip 0pt plus 1filll Default:~##4%
         \hfill\penalty500\relax\qquad \hbox{}\hfill Default:~##4%
         \nobreak\hskip-\parfillskip\hskip0pt\relax
     \fi
     \par}%
   \def\TemplateSemantics{\enddescription\endgroup
       \subsubsection*{Semantics \& Comments:}}%
  }
  {\par\bigskip}

\renewcommand*\MakePrivateLetters{\makeatletter\catcode`\_=11\relax}
\makeatother

\providecommand*\eTeX{\ensuremath{\varepsilon}-\kern-.125em\TeX}
\providecommand*\LaTeXplus{\LaTeXe$*$}
\providecommand*\key[1]{\textbf{#1}}
\providecommand*\pkg[1]{\textsf{#1}}
\newcommand*\switch[2]{{\fontfamily{#1}\selectfont #2}}

\usepackage{lmodern}
\begin{document}
  \catcode`\_=12
  \DocInput{xfrac.dtx}
\end{document}
%</driver>
% \fi
%
%  \GetFileInfo{xfrac.drv}
%  \CheckSum{249}
%  
%  \changes{v0.3}{2010/02/02}{Update to new \pkg{xtemplate} system}
%  \changes{v0.3}{2010/02/02}{Include custom values for
%    \pkg{lmodern}}
%
%  \changes{v0.2a}{2009/08/10}{Update for new version of expl3}
%
%  \changes{v0.2}{2008/08/04}{Require expl3 and get rid of .ins file}
%  \changes{v0.2}{2008/08/04}{Move to macros/latex/contrib/mh on CTAN}
%
%  \changes{v0.11}{2004/05/08}{Consistent naming}
%
%  \changes{v0.10a}{2004/04/06}{Added a dependency of the latest
%  \LaTeX{} release}
%  \changes{v0.10a}{2004/04/06}{First version on \textsc{ctan}}
%
%  \changes{v0.10}{2004/04/05}{First publicly available prototype
%  implementation}
%
%  \title{The \pkg{xfrac} package\thanks{This file has version number
%  \fileversion, last revised \filedate.}}
%
%  \author{Morten H\o gholm}
%  \date{\filedate}
%
%  \maketitle
%
%  \begin{abstract}
%  This package uses a template interface to produce nicely looking
%  \emph{split level} fractions like \nicefrac{7}{9}\ldots{}
%  ehrm\ldots{} I mean \sfrac{7}{9}.
%  \end{abstract}
%
%  \tableofcontents
%
%  \section{User Interface}
%
%  The \pkg{xfrac} package defines a document command \cs{sfrac}
%  with the following syntax:
%  \begin{quote}
%  \cs{sfrac}\oarg{instance}\marg{num}\oarg{sep}\marg{denom}
%  \end{quote}
%  Let's show a few examples:
%  \begin{verbatim}
%  \sfrac{1}{2},  $\sfrac{1}{2}$,
%  $\mathbf{3\times\sfrac{1}{2}}$
%  \quad \fontfamily{ppl}\selectfont Palatino: \sfrac{1}{2}
%  \quad \fontfamily{ptm}\selectfont Times: \sfrac{1}{2}
%  \end{verbatim}
%  \begin{quote}
%  \sfrac{1}{2},  $\sfrac{1}{2}$, $\mathbf{3\times\sfrac{1}{2}}$
%  \quad \fontfamily{ppl}\selectfont Palatino: \sfrac{1}{2}
%  \quad \fontfamily{ptm}\selectfont Times: \sfrac{1}{2}
%  \end{quote}
%  You'll notice something interesting: Not only does the \cs{sfrac}
%  command work as it should in math mode, it also gets the job done
%  for other fonts as well.
%
%
%  \section{A Bit of History}
%
%  \subsection{The Past}
%
%  One of the first exercises in \emph{The \TeX Book} is to design a
%  macro for split level fractions. The solution presented is fairly
%  simple, using a \emph{virgule} (a slash) for separating the two
%  components. It looks okay because the text font and math font of
%  Computer Modern look almost identical.
%
%  The proper symbol to use instead of the virgule is a \emph{solidus}
%  which does not exist in Computer Modern. It is however available in
%  the European Computer Modern fonts, but I'll get back to that.
%
%  \subsection{The Present}
%
%
%  The most common way to produce split level fractions within \LaTeX\
%  is by means of the \pkg{nicefrac} package. Part of the reason it
%  has found widespread use is due to the strange design of the
%  built-in text fractions of the EC fonts, which look like this:
%  \textonehalf. The package is very simple to use but there are a few
%  issues:
% \begin{itemize}
%  \item It uses the virgule instead of the solidus.
%  \item Font size of numerator and denominator is bigger than in the
%    built-in symbol. Compare Palatino: \switch{ppl}{\nicefrac{1}{2}}
%    vs. \switch{ppl}{\textonehalf }.
%  \item It doesn't correct for fonts using text figures such as in the
%    \pkg{eco} package. Compare \switch{cmor}{\nicefrac{1}{2}} and
%    \switch{cmor}{\nicefrac{8}{9}}.
%  \item In math mode, it doesn't always pick up the correct math
%    alphabet.
% \end{itemize}
% In short: \pkg{nicefrac} doesn't attempt to be the answer to
% everything and so this is not a criticism of the package. It works
% quite well for Computer Modern which was pretty much what was widely
% available at the time it was developed. Users these days, however,
% have a choice of many fonts when they write their documents.
%
%
%
%
%
%  \subsection{The Future}
%
%  Fonts are wildly different; one macro that works fine for Computer
%  Modern obviously doesn't work well at all in Palatino. For one we
%  have to make the separator symbol configurable, and we need to
%  take care of several details as well: font scaling of the
%  numerator/denominator pair (ND), font selection of ND etc. If we
%  are to have a single package for this in the future\footnote{As
%  this is intended to be about the future, the \pkg{xfrac} package
%  requires the \eTeX{} extensions.} we have to define a totally
%  generic interface for the fraction commands and then adjust
%  parameters depending on the current font. What you see in this
%  prototype implementation of \pkg{xfrac} is just that.
%
%
%  \section{Advanced User Interface}
%
%
%  \subsection{Text mode}
%
%  The usual problem in text mode has a name: Computer Modern. The
%  solidi of all the Computer Modern fonts leave a lot to be desired,
%  although things are potentially looking better as the Latin Modern
%  fonts are becoming more stable and widespread. As long as the
%  default fonts are Computer Modern variants we must however work
%  around this. One idea that comes to mind is to see what happens
%  when you use a solidus from another font instead. Let's try with
%  Times:
%  \begin{quote}
%  \DeclareInstance{xfrac}{cmr2}{text}{
%    slash-symbol-font = ptm,
%  }
%  ``You take \sfrac[cmr2]{1}{2} cup of sugar, \ldots''
%  \end{quote}
%  That looks quite good actually, so it was probably very difficult
%  to obtain that result. Nope, it was extremely easy---if you happen
%  to know about \emph{instances}:
%  \begin{verbatim}
%  \DeclareInstance{xfrac}{cmr}{text}{
%    slash-symbol-font = ptm,
%  }
%  \end{verbatim}
%  So we define an instance with the name `cmr' from the template
%  `text' which in turn is of template type `xfrac'. You'll notice
%  the `cmr' is also the name of the font family for Computer Modern
%  Roman and the reasoning behind is that every font family should
%  have it's own settings, and if a document command is to work well
%  in that scheme, letting it use the name of the current font family
%  seems like a good idea. Thus the \cs{sfrac} command checks to see
%  whether an instance with same name as the current font family
%  exists and uses it if the test is true; otherwise the default
%  setting is used. Here we defined the instance to be used for the
%  font family `cmr' and just told it to use the Times font for
%  typesetting the slash symbol which turns out to be a solidus by
%  default.
%
%  The option \texttt{cm-recommended} which is loaded by default uses
%  the Times solidus for Computer Modern Roman and Computer Modern
%  Sans Serif and the Palatino solidus for Computer Modern Typewriter
%  Type. This looks quite good. Should you however not want this you
%  can use the option \texttt{cm-standard} which produces somewhat
%  acceptable results using Computer Modern exclusively.
%
%  So what about old style figures? If you use the \pkg{eco}
%  package you might define an instance similar to this (`cmor' is
%  the name of the roman font activated by \pkg{eco}):
%   \DeclareInstance{xfrac}{cmor}{text}{
%    slash-symbol-font = ptm,
%    numerator-font    = cmr,
%    denominator-font  = cmr,
%  }
%  \begin{verbatim}
%  \DeclareInstance{xfrac}{cmor}{text}{
%    slash-symbol-font = ptm,
%    numerator-font    = cmr,
%    denominator-font  = cmr,
%  }
%  \end{verbatim}
%  We also use regular Computer Modern Roman for typesetting ND, so
%  we end up with \switch{cmor}{\sfrac{1}{2}} and
%  \switch{cmor}{\sfrac{8}{9}} instead of
%  \switch{cmor}{\nicefrac{1}{2}} and \switch{cmor}{\nicefrac{8}{9}}.
%  Much better.
%
%  There are also situations where other tricks are useful. If you
%  don't have the inferior and superior figures available in a font,
%  or the font doesn't have a wider design for small font sizes, you
%  can cheat by manually scaling the ND-pair. I got nice results for
%  Adobe's Stempel Garamond (with small caps and old style figures)
%  with the following setup:
%  \begin{verbatim}
%  \DeclareInstance{xfrac}{pegj}{text}{
%    numerator-font   = pegx,
%    denominator-font = pegx,
%    scale-factor     = .9,
%    h-scale          = 1.1,
%  }
%  \end{verbatim}
%  We use the font family `pegx' (Stempel Garamond with real small
%  caps) for typesetting the ND-pair. Additionally the key
%  \key{scale-factor} specifies that the font size used for the
%  ND-pair should be $0.9$ of the height of the solidus, and the key
%  \key{h-scale} specifies that the ND-pair should be scaled an extra
%  10\% horizontally.
%
%  Should you be so fortunate the have a font with inferior and
%  superior figures like in the Monotype Janson example from Philipp
%  Lehman's excellent \emph{The Font Installation Guide}. In that
%  example Philipp defines the font families `mjn0' for the inferior
%  figures and `mjn1' for the superior. Thus to get the \cs{sfrac}
%  command to use them on the fly for the font family `mjnj' (Janson,
%  old style figures) we would say
%  \begin{verbatim}
%  \DeclareInstance{xfrac}{mjnj}{text}{
%    numerator-font      = mjn1,
%    denominator-font    = mjn0,
%    scaling             = false,
%    numerator-bot-sep   = 0pt,
%    denominator-bot-sep = 0pt,
%  }
%  \end{verbatim}
%  I think this example is a very clean way to do it. An alternative
%  approach could be to use the keys \key{numerator-format} and
%  \key{denominator-format} to process the arguments and let them
%  determine what to do.
%
%  As a side note Harald Harders was so kind to test it, and it
%  \emph{does} actually work---I hadn't tested it myself.
%
%  \subsection{Math Mode}
%
%  In math mode the choices are a lot fewer because first of all
%  \TeX{} comes with a built-in limitation of 16 math families.
%  Additionally we will not need a solidus for typesetting split
%  fractions in math, as tradition is to use a virgule instead. We
%  define the basic `mathdefault' instance to simply use the math
%  family in use when the instance is run. So if we're in normal math
%  like |$\sfrac{7}{9}$| we simply get family~$-1$. If we're inside a
%  \cs{mathbf} we're in family~$4$ (in the standard setup at least),
%  and so the fraction is typeset with the same math family. Simple,
%  isn't?
%
%  You can also declare instances for the math families, but I really
%  don't see why you would. If you do then name them according to the
%  scheme `mathfam\meta{N}', where \meta{N} is the family number, and
%  only do it if you \emph{really} know how to set up math fonts.
%  That is, if \cs{DeclareMathAlphabet} is unbeknownst to you, then
%  just don't go there.
%
%
%  Another example: If we want \cs{sfrac} to produce split fractions
%  without doing anything at all, we can choose the collection
%  `plainmath', which is defined as
%  \begin{verbatim}
%  \DeclareCollectionInstance{plainmath}{xfrac}{mathdefault}{math}{
%    denominator-bot-sep = 0pt,
%    numerator-bot-sep   = 0pt,
%    numerator-top-sep   = \c_max_dim,
%    scaling             = false,
%    slash-right-mkern   = 0mu,
%    slash-left-mkern    = 0mu,
%  }
%  \end{verbatim}
%  This creates an alternative version of the instance `mathdefault'
%  with settings as specified by the keys. In the default math setup
%  \key{numerator-top-sep} is set to 0~pt, and here we set
%  \key{numerator-bot-sep} to 0~pt as well, so in order to avoid
%  over-specification (and an error message) we must set
%  \key{numerator-top-sep} to \cs{c_max_dim}. We activate (obeying
%  normal scoping rules) it with:
%  \begin{verbatim}
%  \UseCollection{xfrac}{plainmath}
%  \end{verbatim}
%  Then |$\sfrac{8}{13}$| produces\begingroup
%    \UseCollection{xfrac}{plainmath}
%    $\sfrac{8}{13}$ and just typing |$8/13$| gives the same result:
%    $8/13$.
%  \endgroup
%
%
%  \section{The Template Interface}
%
%  \begin{TemplateInterfaceDescription}{xfrac}
%
%  \TemplateArgument{1}
%      {The numerator}
%
%  \TemplateArgument{2}
%       {The separator}
%
%   \TemplateArgument{3}
%      {The denominator}
%
%  \TemplateSemantics
%
%  Typesets arguments 1 and 3 separated by argument 2, which in text
%  mode by default is a \emph{solidus}. This is taken from
%  \pkg{textcomp} where it is denoted \cs{textfractionsolidus}.
%  This is the character used for the ready made split level
%  fractions such as \textonehalf---except in the (European) Computer
%  Modern fonts. In math mode a \emph{virgule} is used instead as
%  this is more appropriate and it is always available in the math
%  fonts. The solidus is a text symbol only.
%
%  \end{TemplateInterfaceDescription}
%
%  \begin{TemplateDescription}{xfrac}{text}
%
%  \TemplateKey{numerator-font}{tokenlist}
%    {Font family specification to use for the numerator.}
%    {\cs{f@family}}
%
%  \TemplateKey{numerator-format}{function 1 arg}
%    {Action to be taken on the numerator.}
%    {Process argument unchanged}
%
%  \TemplateKey{slash-symbol}{tokenlist}
%    {The separator symbol. If not specified the default value will be
%    used instead.}
%    {Solidus (\cs{textfractionsolidus})}
%
%  \TemplateKey{slash-symbol-font}{tokenlist}
%    {Font family specification to use for the separator symbol.}
%    {\cs{f@family}}
%
%  \TemplateKey{slash-symbol-format}{function 1 arg}
%    {Action to be taken on the separator symbol.}
%    {Process argument unchanged}
%
%  \TemplateKey{denominator-font}{tokenlist}
%    {Font family specification to use for the denominator.}
%    {\cs{f@family}}
%
%  \TemplateKey{denominator-format}{function 1 arg}
%    {Action to be taken on the denominator.}
%    {Process argument unchanged}
%
%  \TemplateKey{h-scale}{tokenlist}
%    {Factor by which the numerator and denominator should be
%    horizontally scaled. It should only be used if the real superior
%    and inferior fonts are not available. For instance Stempel
%    Garamond looks excellent if scaled 10\% extra horizontally, i.e.,
%    by a factor of 1.1.}
%    {1}
%
%  \TemplateKey{v-scale}{tokenlist}
%    {Same as \key{h-scale} only vertically. Probably not of much use
%    but added for completetion.}
%    {1}
%
%  \TemplateKey{scale-factor}{tokenlist}
%    {Fraction of the size of \key{slash-symbol}. Used for setting the
%    font size of numerator and denominator. Usually a value of app.\
%    \sfrac{5}{6} produces fine results. It should only be used if the
%    real superior and inferior fonts are not available. As an example
%    Stempel Garamond looks better if the factor is 0.9.}
%    {0.83333}
%
%  \TemplateKey{scale-relative}{choice}
%    {If set to `true' the font size of the numerator and denominator
%    is scaled with respect to the height of the \key{slash-symbol}. If
%    set to `false' the font is scaled with respect to the total height
%    of the \key{slash-symbol}.}
%    {true}
%
%  \TemplateKey{scaling}{choice}
%    {If set to `true' the fonts are allowed to scale. If set to
%    `false' they are not. See the `Janson' example for an application.}
%    {true}
%
%  \TemplateKey{numerator-top-sep}{length}
%    {Dimension specifying the space between the top of the
%    \key{slash-symbol} and the top of the numerator. If not specified,
%    the depth of the solidus will be used, because this value will
%    make the fraction look even.}
%    {Unspecified}
%
%  \TemplateKey{numerator-bot-sep}{length}
%    {Dimension specifying the lift of the numerator from the
%    baseline.}
%    {Unspecified}
%
%  \TemplateKey{denominator-bot-sep}{length}
%    {Dimension specifying the lift of the denominator from the
%    baseline.}
%    {Unspecified}
%
%  \TemplateKey{slash-right-kern}{length}
%    {Dimension specifying the kerning between the \key{slash-symbol}
%    and the numerator.}
%    {\texttt{0pt}}
%
%  \TemplateKey{slash-left-kern}{length}
%    {Dimension specifying the kerning between the \key{slash-symbol}
%    and the denominator.}
%    {\texttt{0pt}}
%
%  \TemplateKey{math-mode}{choice}
%    {Are we in math mode or not?}
%    {false}
%
%  \TemplateKey{phantom}{tokenlist}
%    {A character that suits the common cases. As we would mostly want
%    to use numbers in text mode we choose a `tall' number, while in
%    math it is somewhat different.}
%    {8}
%
%  \TemplateSemantics
%
%  This template is also the foundation for the `math' template. The
%  keys \key{slash-right-mkern} and \key{slash-left-mkern} can only
%  be used in math mode and are not shown here.
%
%  \end{TemplateDescription}
%
%
%  \begin{TemplateDescription}{xfrac}{math}
%
%  \TemplateKey{numerator-font}{tokenlist}
%    {Font family specification to use for the numerator.}
%    {\cs{number}\cs{fam}}
%
%  \TemplateKey{slash-symbol}{tokenlist}
%    {The separator symbol. If not specified the default value will be
%    used instead.}
%    {Virgule ($/$)}
%
%  \TemplateKey{slash-symbol-font}{tokenlist}
%    {Font family specification to use for the separator symbol.}
%    {\cs{number}\cs{fam}}
%
%  \TemplateKey{denominator-font}{tokenlist}
%    {Font family specification to use for the denominator.}
%    {\cs{number}\cs{fam}}
%
%  \TemplateKey{scale-factor}{tokenlist}
%    {Fraction of the size of \key{slash-symbol}. In math mode we
%    cannot rely on the fonts to be able to scale, but giving a default
%    scale of 0.7 fits into the regular size changing scheme---the
%    default scheme has values $(D,T,S,SS)=(1,1,0.7,0.5)$ whereas we
%    with a default \key{scale-factor} of 0.7 get $(1,1,0.7,0.49)$.
%    That's close enough.}
%    {0.7}
%
%  \TemplateKey{scale-relative}{choice}
%    {If set to `true' the font size of the numerator and denominator
%    is scaled with respect to the height of the \key{slash-symbol}. If
%    set to `false' the font is scaled with respect to the total height
%    of the \key{slash-symbol}.}
%    {false}
%
%  \TemplateKey{scaling}{choice}
%    {If set to `true' the fonts are allowed to scale. If set to
%    `false' they are not. See the `plainmath' example for an application.}
%    {true}
%
%  \TemplateKey{numerator-top-sep}{length}
%    {Dimension specifying the space between the top of the
%    \key{slash-symbol} and the top of the numerator. If not specified,
%    the depth of the virgule will be used, because this value will
%    make the fraction look even.}
%    {\texttt{0pt}}
%
%  \TemplateKey{denominator-bot-sep}{length}
%    {Dimension specifying the lift of the denominator from the
%    baseline.}
%    {\texttt{0pt}}
%
%  \TemplateKey{slash-right-mkern}{tokenlist}
%    {Same as \key{slash-right-kern} but for math mode only and should
%    be specified in \texttt{mu} units. This is because \pkg{calc}
%    can't use mu-expressions.}
%    {\texttt{-2mu}}
%
%  \TemplateKey{slash-left-mkern}{tokenlist}
%    {Same as \key{slash-left-kern} but for math mode only and should
%    be specified in \texttt{mu} units. This is because \pkg{calc}
%    can't use mu-expressions.}
%    {\texttt{-1mu}}
%
%  \TemplateKey{math-mode}{choice}
%    {Are we in math mode or not?}
%    {true}
%
%  \TemplateKey{phantom}{tokenlist}
%    {A character that suits the common cases. In math we have a high
%    risk of using a parenthesis, so we choose that. Text mode is
%    another story.}
%    {(}
%
%  \TemplateSemantics
%
%  This template is a restricted version of the `text' template. Only
%  the keys that are different from the `text' template are shown
%  here. Also bear in mind that the attributes \key{slash-left-kern}
%  and \key{slash-right-kern} have no meaning in this template.
%
%  \end{TemplateDescription}
% 
%  \StopEventually{}
%
%\section{Implementation}
%
%    \begin{macrocode}
%<*package>
%    \end{macrocode}
%       
% The usual lead-off: provides an experimental package!
%    \begin{macrocode}
\RequirePackage{expl3}[2009/08/05]
\ProvidesExplPackage{xfrac}{2010/02/02}{0.3}{Text fractions}
%    \end{macrocode}
%
% Some support is needed: a bit wider than the normal \pkg{xpackage}
% stuff, but not by much.
%    \begin{macrocode}
\RequirePackage { amstext , graphicx , l3keys2e , textcomp , xtemplate }
%    \end{macrocode}   
%
%\begin{macro}{\l_xfrac_cm_std_bool}
% There is one option to support.
%    \begin{macrocode}
\keys_define:nn { xfrac } {
  cm-recommended .choice:,
  cm-recommended /
    false        .code:n     =
      { \bool_set_true:N \l_xfrac_cm_std_bool },
  cm-recommended /
    true         .code:n     =
      { \bool_set_false:N \l_xfrac_cm_std_bool },
  cm-recommended .default:n  = { true },    
  cm-standard    .bool_set:N = \l_xfrac_cm_std_bool
}
\ProcessKeysOptions { xfrac }
%    \end{macrocode}
%\end{macro}
%
%\begin{macro}{\l_xfrac_slash_box}
%\begin{macro}{\l_xfrac_tmp_box}
% In keeping with the \LaTeX3 philosophy, rather than use generic 
% scratch boxes and get confused, \pkg{xfrac} reserves its own named
% working space.
%    \begin{macrocode}
\box_new:N \l_xfrac_slash_box
\box_new:N \l_xfrac_tmp_box
%    \end{macrocode}
%\end{macro}
%\end{macro}
%
%\begin{macro}{\xfrac_tmp:w}
% Used for the raised boxes: weird as it does not take an argument
% but the \cs{raisebox} does.
%    \begin{macrocode}
\cs_new:Npn \xfrac_tmp:w { }
%    \end{macrocode}
%\end{macro}
%
%\subsection{Initialisation of variables}
%
% Variables used in templates have to be set up: there is not
% much to say about these, other than that they must exist.
%    
%\begin{macro}{\l_xfrac_denominator_bot_sep_dim}
%\begin{macro}{\l_xfrac_numerator_bot_sep_dim}
%\begin{macro}{\l_xfrac_numerator_top_sep_dim}
%\begin{macro}{\l_xfrac_slash_left_sep_dim}
%\begin{macro}{\l_xfrac_slash_right_sep_dim}
% Fixed lengths.
%    \begin{macrocode}
\dim_new:N \l_xfrac_denominator_bot_sep_dim
\dim_new:N \l_xfrac_numerator_bot_sep_dim
\dim_new:N \l_xfrac_numerator_top_sep_dim
\dim_new:N \l_xfrac_slash_left_sep_dim
\dim_new:N \l_xfrac_slash_right_sep_dim
%    \end{macrocode}
%\end{macro}
%\end{macro}
%\end{macro}
%\end{macro}
%\end{macro}
%
%\begin{macro}{\l_xfrac_denominator_font_tl}
%\begin{macro}{\l_xfrac_hscale_tl}
%\begin{macro}{\l_xfrac_numerator_font_tl}
%\begin{macro}{\l_xfrac_phantom_tl }
%\begin{macro}{\l_xfrac_scale_factor_tl}
%\begin{macro}{\l_xfrac_slash_left_msep_tl}
%\begin{macro}{\l_xfrac_slash_right_msep_tl}
%\begin{macro}{\l_xfrac_slash_symbol_tl}
%\begin{macro}{\l_xfrac_slash_symbol_font_tl}
%\begin{macro}{\l_xfrac_vscale_tl}
% Token lists, which include floating-point numbers and math(s)
% skips.
%    \begin{macrocode}
\tl_new:N \l_xfrac_denominator_font_tl
\tl_new:N \l_xfrac_hscale_tl
\tl_new:N \l_xfrac_numerator_font_tl
\tl_new:N \l_xfrac_phantom_tl 
\tl_new:N \l_xfrac_scale_factor_tl
\tl_new:N \l_xfrac_slash_left_msep_tl
\tl_new:N \l_xfrac_slash_right_msep_tl
\tl_new:N \l_xfrac_slash_symbol_tl
\tl_new:N \l_xfrac_slash_symbol_font_tl
\tl_new:N \l_xfrac_vscale_tl
%    \end{macrocode}
%\end{macro}
%\end{macro}
%\end{macro}
%\end{macro}
%\end{macro}
%\end{macro}
%\end{macro}
%\end{macro}
%\end{macro}
%\end{macro}
%
%\begin{macro}{\xfrac_fontscale:}
%\begin{macro}{\xfrac_math:n}
%\begin{macro}{\xfrac_denominator_font_change:}
%\begin{macro}{\xfrac_denominator_format:n}
%\begin{macro}{\xfrac_numerator_font_change:}
%\begin{macro}{\xfrac_numerator_format:n}
%\begin{macro}{\xfrac_relscale:}
%\begin{macro}{\xfrac_slash_symbol_font_change:}
%\begin{macro}{\xfrac_slash_symbol_format:n}
%\begin{macro}{\xfrac_text_or_math:n}
% Functions, either things which are calculated `on the fly'
% (no argument required) or are functions taking one argument in the
% code.
%    \begin{macrocode}
\cs_new_nopar:Npn \xfrac_fontscale:
\cs_new:Npn \xfrac_math:n #1 { }
\cs_new_nopar:Npn \xfrac_denominator_font_change: { }
\cs_new:Npn \xfrac_denominator_format:n #1 { }
\cs_new_nopar:Npn \xfrac_numerator_font_change: { }
\cs_new:Npn \xfrac_numerator_format:n #1 { }
\cs_new_nopar:Npn \xfrac_relscale: { }
\cs_new_nopar:Npn \xfrac_slash_symbol_font_change: { }
\cs_new:Npn \xfrac_slash_symbol_format:n #1 { }
\cs_new:Npn \xfrac_text_or_math:n #1 { }
%    \end{macrocode}
%\end{macro}
%\end{macro}
%\end{macro}
%\end{macro}
%\end{macro}
%\end{macro}
%\end{macro}
%\end{macro}
%\end{macro}
%\end{macro}
%
%\subsection{The template}
%
% There is only one object type in \pkg{xfrac}, rather unimaginatively
% named \texttt{xfrac}.
%    \begin{macrocode}
\DeclareObjectType { xfrac } { 3 }
%    \end{macrocode}
%
% A single template interface is used for both text and math(s), which
% does make a few things a little complex later.
%    \begin{macrocode}
\DeclareTemplateInterface { xfrac } { text } { 3 } {
  denominator-bot-sep : length     = \c_max_dim           ,
  denominator-font    : tokenlist  = \f@family            ,
  denominator-format  : function 1 = #1                   ,
  h-scale             : tokenlist  = 1                    ,
  math-mode           : choice { false , true }
                                   = false                ,
  numerator-font      : tokenlist  = \f@family            ,
  numerator-format    : function 1 = #1                   ,
  numerator-bot-sep   : length     = \c_max_dim           ,
  numerator-top-sep   : length     = \c_max_dim           ,
  phantom             : tokenlist  = 8                    ,
  scale-factor        : tokenlist  = 0.83333              ,
  scale-relative      : choice { false , true }
                                   = true                 ,
  scaling             : choice { false , true }
                                   = true                 ,
  slash-left-kern     : length     = 0 pt                 ,
  slash-left-mkern    : tokenlist  = -2 mu                ,
  slash-right-kern    : length     = 0 pt                 ,
  slash-right-mkern   : tokenlist  = -1 mu                ,
  slash-symbol        : tokenlist  = \textfractionsolidus ,
  slash-symbol-font   : tokenlist  = \f@family            ,
  slash-symbol-format : function 1 = #1                   ,
  v-scale             : tokenlist  = 1                    ,
}
%    \end{macrocode}
%   
% Most of the variable binding is quite simple: of course, the choices
% are a little more complicated. That is particularly true where 
% these have to set up `on the fly' functions.   
%    \begin{macrocode}
\DeclareTemplateCode { xfrac } { text } { 3 } 
  {
    denominator-bot-sep = \l_xfrac_denominator_bot_sep_dim ,
    denominator-font    = \l_xfrac_denominator_font_tl     ,
    denominator-format  = \xfrac_denominator_format:n      ,
    h-scale             = \l_xfrac_hscale_tl               ,
    math-mode           =
      {
        false = \cs_set_eq:NN \xfrac_math:n \use:n,
        true  = \cs_set_eq:NN \xfrac_math:n \ensuremath
      },
    numerator-font      = \l_xfrac_numerator_font_tl       ,
    numerator-format    = \xfrac_numerator_format:n        ,
    numerator-bot-sep   = \l_xfrac_numerator_bot_sep_dim   ,
    numerator-top-sep   = \l_xfrac_numerator_top_sep_dim   ,
    phantom             = \l_xfrac_phantom_tl              ,
    scale-factor        = \l_xfrac_scale_factor_tl         ,
    scale-relative      =
      {
        false = 
          \cs_set_nopar:Npn \xfrac_relscale:
            {
              \dim_eval:n 
                { 
                    \box_ht:N \l_xfrac_tmp_box 
                  + \box_dp:N \l_xfrac_tmp_box
                } 
            },
        true  = 
          \cs_set_nopar:Npn \xfrac_relscale: 
            { \box_ht:N \l_xfrac_slash_box }
      },
    scaling             =
      {
        false = \cs_set_eq:NN \xfrac_fontscale: \prg_do_nothing:, 
        true  =  
          \cs_set_nopar:Npn \xfrac_fontscale:
            {
              \fontsize { \l_xfrac_scale_factor_tl \xfrac_relscale: }
                { \c_zero_dim } 
              \selectfont
            }
      },
    slash-left-kern     = \l_xfrac_slash_left_sep_dim      ,
    slash-left-mkern    = \l_xfrac_slash_left_msep_tl      ,
    slash-right-kern    = \l_xfrac_slash_right_sep_dim     ,
    slash-right-mkern   = \l_xfrac_slash_right_msep_tl     ,
    slash-symbol        = \l_xfrac_slash_symbol_tl         ,
    slash-symbol-font   = \l_xfrac_slash_symbol_font_tl    ,
    slash-symbol-format = \xfrac_slash_symbol_format:n     ,
    v-scale             = \l_xfrac_vscale_tl        
  }
%    \end{macrocode}
% The implementation part starts with applying all of the settings
% from above. The first part of the set up is then to determine
% whether the surroundings are text or math(s), and react accordingly.
%    \begin{macrocode}
  {
    \AssignTemplateKeys
    \mode_if_math:TF
      {
        \cs_set_eq:NN \xfrac_text_or_math:n \text
        \cs_set_nopar:Npx \xfrac_denominator_font_change:
          { \tex_fam:D \l_xfrac_denominator_font_tl }
        \cs_set_nopar:Npx \xfrac_numerator_font_change:
          { \tex_fam:D \l_xfrac_numerator_font_tl }
        \cs_set_nopar:Npx \xfrac_slash_symbol_font_change:
          { \tex_fam:D \l_xfrac_slash_symbol_font_tl }
      }
      {
        \cs_set_eq:NN \xfrac_text_or_math:n \mbox
        \cs_set_nopar:Npn \xfrac_denominator_font_change:
          { 
            \fontfamily { \l_xfrac_denominator_font_tl }
            \selectfont
          }
        \cs_set_nopar:Npn \xfrac_numerator_font_change:
          { 
            \fontfamily { \l_xfrac_numerator_font_tl }
            \selectfont
          }
        \cs_set_nopar:Npn \xfrac_slash_symbol_font_change:
          { 
            \fontfamily { \l_xfrac_slash_symbol_font_tl }
            \selectfont
          }
      }
%    \end{macrocode}
%\changes{v0.11a}{2004/08/24}{Added \cs{m@th}}
% Everything is now either inside \cs{text} or an \cs{mbox}, depending
% upon the surroundings. First, there are some boxes to set up.
%    \begin{macrocode}
    \xfrac_text_or_math:n 
      {
        \m@th
        \hbox_set:Nn \l_xfrac_tmp_box 
          { \xfrac_math:n { \vphantom { ( ) } } }
        \hbox_set:Nn \l_xfrac_slash_box 
          {
            \xfrac_math:n 
              { 
                \xfrac_slash_symbol_format:n 
                  {
                    \xfrac_math:n 
                      {
                        \xfrac_slash_symbol_font_change:
                        \IfNoValueTF {#2} 
                          { \l_xfrac_slash_symbol_tl } {#2}
                      }
                  }
              }
          }
%    \end{macrocode}
% Check on the numerator separator dimensions. The code starts with the
% assumption that neither has been given, as this can then be used to
% set up a default, which is also used when both values are set 
% erroneously.
%    \begin{macrocode}
        \cs_set_nopar:Npn \xfrac_tmp:w
          {
            \raisebox 
              {
                  \box_ht:N \l_xfrac_slash_box
                - \box_dp:N \l_xfrac_slash_box
                - \height
              }
          }
        \dim_compare:nNnTF 
          { \l_xfrac_numerator_top_sep_dim } = { \c_max_dim }
          {
            \dim_compare:nNnF
              { \l_xfrac_numerator_bot_sep_dim } = { \c_max_dim } 
              {
                \cs_set_nopar:Npn \xfrac_tmp:w
                  { 
                    \raisebox 
                      { \dim_use:N \l_xfrac_numerator_bot_sep_dim } 
                  }
              }
          }
          {
            \dim_compare:nNnTF
              { \l_xfrac_numerator_bot_sep_dim } = { \c_max_dim } 
                { 
                  \cs_set_nopar:Npn \xfrac_tmp:w
                    { 
                      \raisebox 
                        { 
                            \box_ht:N \l_xfrac_slash_box
                          - \dim_use:N \l_xfrac_numerator_top_sep_dim 
                          - \height
                        }
                    }
                }
                { 
                  \msg_error:nn { xfrac } 
                    { over-specified-numerator-sep }
                }
          }
%    \end{macrocode}
%  Typeset the numerator.
%    \begin{macrocode}
        \xfrac_tmp:w
          {
            \xfrac_fontscale:
            \xfrac_numerator_format:n
              {
                \scalebox { \l_xfrac_hscale_tl } [ \l_xfrac_vscale_tl ]
                  { 
                    \xfrac_math:n 
                      { 
                        \xfrac_numerator_font_change:
                        {
                          \vphantom { \l_xfrac_phantom_tl }
                          #1
                        }
                      } 
                  }
              }
          }
        \xfrac_math:n
          { % THIS IS JUST WRONG! 
            \mode_if_math:TF
              { \tex_mskip:D \l_xfrac_slash_right_msep_tl }
              { \tex_hskip:D \l_xfrac_slash_right_sep_dim }
          }  
%    \end{macrocode}
%  Typeset the separator.
%    \begin{macrocode}
        \box_use:N \l_xfrac_slash_box
        \xfrac_math:n 
          {
            \mode_if_math:TF
              { \tex_mskip:D \l_xfrac_slash_left_msep_tl }
              { \tex_hskip:D \l_xfrac_slash_left_sep_dim }
          }
%    \end{macrocode}
%  Typeset the denominator.
%    \begin{macrocode}
        \dim_compare:nNnTF 
          { \l_xfrac_denominator_bot_sep_dim } = { \c_max_dim }
          {
            \cs_set_nopar:Npn \xfrac_tmp:w
              { \raisebox { - \box_dp:N \l_xfrac_slash_box } } 
          }
          {
            \cs_set_nopar:Npn \xfrac_tmp:w
              { 
                \raisebox 
                  { \dim_use:N \l_xfrac_denominator_bot_sep_dim }
              }
          }  
        \xfrac_tmp:w
          {
            \xfrac_fontscale:
            \xfrac_denominator_format:n
              {
                \scalebox { \l_xfrac_hscale_tl } [ \l_xfrac_vscale_tl ]
                  { 
                    \xfrac_math:n 
                      { 
                        \xfrac_denominator_font_change:
                        {
                          \vphantom { \l_xfrac_phantom_tl }
                          #3
                        }
                      } 
                  }
              }
          }
      } 
  }
%    \end{macrocode}
%    
% Since math(s) and text mode are wildly different entities we define a
% separate template for each. You already saw the `text' 
% template, and here is the `math' template.
%    \begin{macrocode}
\DeclareRestrictedTemplate { xfrac } { text } { math } {
  numerator-font      = \number \fam ,
  slash-symbol        = /            ,
  slash-symbol-font   = \number \fam ,
  denominator-font    = \number \fam ,
  scale-factor        = 0.7          ,
  scale-relative      = false        ,
  scaling             = true         ,
  numerator-top-sep   = 0 pt         ,
  denominator-bot-sep = 0 pt         ,
  math-mode           = true         ,
  phantom             = (
}
%    \end{macrocode}
%    
%\subsection{The standard instances}
%
% For the default instances we just use the relevant templates with
% the default settings.
% 
%  The default `text' instance.
%    \begin{macrocode}
\DeclareInstance { xfrac } { default } { text } { }
%    \end{macrocode}
%    
%  The default `math(s)' instance.
%    \begin{macrocode}
\DeclareInstance { xfrac } { mathdefault } { math } { }
%    \end{macrocode}
%    \begin{macrocode}
\DeclareCollectionInstance { plainmath } { xfrac } { mathdefault } 
  { math }{
  denominator-bot-sep = 0 pt       ,
  numerator-bot-sep   = 0 pt       ,
  numerator-top-sep   = \c_max_dim ,
  scale-factor        = 1          ,
  scale-relative      = false      ,
  scaling             = true       ,
  slash-right-mkern   = 0mu        ,
  slash-left-mkern    = 0mu
}
%    \end{macrocode}
%    
% Default Computer Modern setup. Far from optimal, but better than
% nothing.
%    \begin{macrocode}
\DeclareInstance { xfrac } { cmr } { text } {
  denominator-bot-sep = 0 pt    ,
  numerator-top-sep   = 0.2 ex  ,
  slash-left-kern     = -0.1 em ,
  slash-right-kern    = -0.1 em 
}
\DeclareInstance { xfrac } { cmss } { text } {
  denominator-bot-sep = 0 pt    ,
  numerator-top-sep   = 0.2 ex  ,
  slash-left-kern     = -0.1 em ,
  slash-right-kern    = -0.1 em 
}
\DeclareInstance { xfrac } { cmtt } { text } {
  denominator-bot-sep = 0 pt    ,
  numerator-top-sep   = 0.2 ex  ,
  slash-left-kern     = -0.1 em ,
  slash-right-kern    = -0.1 em 
}
%    \end{macrocode}
%    
% We can do better for the Computer Modern fonts. For cmr and cmss 
% we choose Times, and for cmtt use Palatino.
%    \begin{macrocode}
\bool_if:NF \l_xfrac_cm_std_bool
  {
    \DeclareInstance { xfrac } { cmr } { text }
      { slash-symbol-font = ptm }
    \DeclareInstance { xfrac } { cmss } { text }
      { slash-symbol-font = ptm }
    \DeclareInstance { xfrac } { cmtt } { text }
      { slash-symbol-font = ppl }
  }
%    \end{macrocode}
%    
% Things works slightly better with Latin Modern.
%    \begin{macrocode}
\DeclareInstance { xfrac } { lmr } { text } {
  denominator-bot-sep = 0 pt     ,
  numerator-top-sep   = 0.1 ex   ,
  slash-left-kern     = -0.15 em ,
  slash-right-kern    = -0.15 em 
}
\DeclareInstance { xfrac } { lmss } { text } {
  denominator-bot-sep = 0 pt     ,
  numerator-top-sep   = 0 pt     ,
  slash-left-kern     = -0.15 em ,
  slash-right-kern    = -0.15 em 
}
\DeclareInstance { xfrac } { lmtt } { text } {
  denominator-bot-sep = 0 pt     ,
  numerator-top-sep   = 0 pt     ,
  slash-left-kern     = -0.15 em ,
  slash-right-kern    = -0.15 em 
}
%    \end{macrocode}
%
%\subsection{Messages}
%
% Just the one.    
%    \begin{macrocode}
\msg_new:nnnn { xfrac } { over-specified-numerator-sep }
  {You have specified both numerator-top-sep and numerator-bot-sep}
  {I will pretend that you didn't specify either of them}
%    \end{macrocode}    
%    
%\subsection{The user command}
%
% Currently there is just a single user command. \cs{sfrac} takes
% two mandatory arguments: numerator and denominator. It can take an
% optional argument between the mandatory specifying the separator
% like this:
%\begin{verbatim}
%  \sfrac{7}[/]{12}
%\end{verbatim}
% It also has an optional argument that comes before the first
% mandatory argument. If used it will use that instance instead of
% the auto-detected one, so a user who has defined the instance
% `cmr2' may use
%\begin{verbatim}
%  \sfrac[cmr2]{7}{12}
%\end{verbatim}
% and get the settings from `cmr2' instead of the settings of 
% the current font family.
%    \begin{macrocode}
\NewDocumentCommand \sfrac { o m o m } {
  \mode_if_math:TF 
    {
      \IfInstanceExistTF { xfrac } { mathfam \number \fam }
        { \UseInstance { xfrac } { mathfam \number \fam } }
        { \UseInstance { xfrac } { mathdefault } }
      {#2} {#3} {#4}  
    }
    {
      \IfInstanceExistTF { xfrac } {#1}
        { \UseInstance { xfrac } {#1} }
        { 
          \IfInstanceExistTF { xfrac } { \f@family }
            { \UseInstance { xfrac } { \f@family } }
            { \UseInstance { xfrac } { default } }
        }
      {#2} {#3} {#4}   
    }
}
%    \end{macrocode}
%        (quote the arguments according to the demands of your shell)
%
% Documentation:
%    (a) If xfrac.drv is present:
%           latex xfrac.drv
%    (b) Without xfrac.drv:
%           latex xfrac.dtx; ...
%    The class ltxdoc loads the configuration file ltxdoc.cfg
%    if available. Here you can specify further options, e.g.
%    use A4 as paper format:
%       \PassOptionsToClass{a4paper}{article}
%
%    Programm calls to get the documentation (example):
%       pdflatex xfrac.dtx
%       makeindex -s gind.ist xfrac.idx
%       pdflatex xfrac.dtx
%       makeindex -s gind.ist xfrac.idx
%       pdflatex xfrac.dtx
%
% Installation:
%    TDS:tex/latex/mh/xfrac.sty
%    TDS:doc/latex/mh/xfrac.pdf
%    TDS:source/latex/mh/xfrac.dtx
%
%<*ignore>
\begingroup
  \def\x{LaTeX2e}
\expandafter\endgroup
\ifcase 0\ifx\install y1\fi\expandafter
         \ifx\csname processbatchFile\endcsname\relax\else1\fi
         \ifx\fmtname\x\else 1\fi\relax
\else\csname fi\endcsname
%</ignore>
%<*install>
\input docstrip.tex
\Msg{************************************************************************}
\Msg{* Installation}
\Msg{* Package: xfrac 2009/08/10 v0.3 textstyle fractions (MH)}
\Msg{************************************************************************}

\keepsilent
\askforoverwritefalse

\preamble

This is a generated file.

Copyright (C) 2004, 2008-2010 by Morten Hoegholm <mh.ctan@gmail.com>

This work may be distributed and/or modified under the
conditions of the LaTeX Project Public License, either
version 1.3c of this license or (at your option) any later
version. The latest version of this license is in
   http://www.latex-project.org/lppl.txt

This work has the LPPL maintenance status "maintained".

This Current Maintainer of this work is Morten Hoegholm.

This work consists of the main source file xfrac.dtx
and the derived files
   xfrac.sty, xfrac.pdf, xfrac.ins, xfrac.drv.

\endpreamble

\generate{%
  \file{xfrac.ins}{\from{xfrac.dtx}{install}}%
  \file{xfrac.drv}{\from{xfrac.dtx}{driver}}%
  \usedir{tex/latex/mh}%
  \file{xfrac.sty}{\from{xfrac.dtx}{package}}%
}

\obeyspaces
\Msg{************************************************************************}
\Msg{*}
\Msg{* To finish the installation you have to move the following}
\Msg{* file into a directory searched by TeX:}
\Msg{*}
\Msg{*     xfrac.sty}
\Msg{*}
\Msg{* To produce the documentation run the file `xfrac.drv'}
\Msg{* through LaTeX.}
\Msg{*}
\Msg{* Happy TeXing!}
\Msg{*}
\Msg{************************************************************************}

\endbatchfile
%</install>
%<*ignore>
\fi
%</ignore>
%<*driver>
\NeedsTeXFormat{LaTeX2e}
\ProvidesFile{xfrac.drv}%
  [2010/02/02 v0.3 Text style fractions]
\RequirePackage{fix-cm}
\documentclass{ltxdoc}


\usepackage{xfrac}
\usepackage{nicefrac}
\usepackage[latin1]{inputenc}
\usepackage[T1]{fontenc}
\makeatletter
\newenvironment{TemplateInterfaceDescription}[1]
  {\subsection{The Template Type `#1'}%
   \begingroup
   \@beginparpenalty\@M
   \description
   \def\TemplateArgument##1##2{\item[Arg: ##1]##2\par}%
   \def\TemplateSemantics{\enddescription\endgroup
       \subsubsection*{Semantics:}}%
  }
  {\par\bigskip}

\newenvironment{TemplateDescription}[2]
  {\subsection{The Template `#2' (type #1)}%
   \subsubsection*{Attributes:}%
   \begingroup
   \@beginparpenalty\@M
   \description
   \def\TemplateKey##1##2##3##4{\item[##1 (##2)]##3%
     \ifx\TemplateKey##4\TemplateKey\else
%         \hskip0ptplus3em\penalty-500\hskip 0pt plus 1filll Default:~##4%
         \hfill\penalty500\relax\qquad \hbox{}\hfill Default:~##4%
         \nobreak\hskip-\parfillskip\hskip0pt\relax
     \fi
     \par}%
   \def\TemplateSemantics{\enddescription\endgroup
       \subsubsection*{Semantics \& Comments:}}%
  }
  {\par\bigskip}

\renewcommand*\MakePrivateLetters{\makeatletter\catcode`\_=11\relax}
\makeatother

\providecommand*\eTeX{\ensuremath{\varepsilon}-\kern-.125em\TeX}
\providecommand*\LaTeXplus{\LaTeXe$*$}
\providecommand*\key[1]{\textbf{#1}}
\providecommand*\pkg[1]{\textsf{#1}}
\newcommand*\switch[2]{{\fontfamily{#1}\selectfont #2}}

\usepackage{lmodern}
\begin{document}
  \catcode`\_=12
  \DocInput{xfrac.dtx}
\end{document}
%</driver>
% \fi
%
%  \GetFileInfo{xfrac.drv}
%  \CheckSum{249}
%  
%  \changes{v0.3}{2010/02/02}{Update to new \pkg{xtemplate} system}
%  \changes{v0.3}{2010/02/02}{Include custom values for
%    \pkg{lmodern}}
%
%  \changes{v0.2a}{2009/08/10}{Update for new version of expl3}
%
%  \changes{v0.2}{2008/08/04}{Require expl3 and get rid of .ins file}
%  \changes{v0.2}{2008/08/04}{Move to macros/latex/contrib/mh on CTAN}
%
%  \changes{v0.11}{2004/05/08}{Consistent naming}
%
%  \changes{v0.10a}{2004/04/06}{Added a dependency of the latest
%  \LaTeX{} release}
%  \changes{v0.10a}{2004/04/06}{First version on \textsc{ctan}}
%
%  \changes{v0.10}{2004/04/05}{First publicly available prototype
%  implementation}
%
%  \title{The \pkg{xfrac} package\thanks{This file has version number
%  \fileversion, last revised \filedate.}}
%
%  \author{Morten H\o gholm}
%  \date{\filedate}
%
%  \maketitle
%
%  \begin{abstract}
%  This package uses a template interface to produce nicely looking
%  \emph{split level} fractions like \nicefrac{7}{9}\ldots{}
%  ehrm\ldots{} I mean \sfrac{7}{9}.
%  \end{abstract}
%
%  \tableofcontents
%
%  \section{User Interface}
%
%  The \pkg{xfrac} package defines a document command \cs{sfrac}
%  with the following syntax:
%  \begin{quote}
%  \cs{sfrac}\oarg{instance}\marg{num}\oarg{sep}\marg{denom}
%  \end{quote}
%  Let's show a few examples:
%  \begin{verbatim}
%  \sfrac{1}{2},  $\sfrac{1}{2}$,
%  $\mathbf{3\times\sfrac{1}{2}}$
%  \quad \fontfamily{ppl}\selectfont Palatino: \sfrac{1}{2}
%  \quad \fontfamily{ptm}\selectfont Times: \sfrac{1}{2}
%  \end{verbatim}
%  \begin{quote}
%  \sfrac{1}{2},  $\sfrac{1}{2}$, $\mathbf{3\times\sfrac{1}{2}}$
%  \quad \fontfamily{ppl}\selectfont Palatino: \sfrac{1}{2}
%  \quad \fontfamily{ptm}\selectfont Times: \sfrac{1}{2}
%  \end{quote}
%  You'll notice something interesting: Not only does the \cs{sfrac}
%  command work as it should in math mode, it also gets the job done
%  for other fonts as well.
%
%
%  \section{A Bit of History}
%
%  \subsection{The Past}
%
%  One of the first exercises in \emph{The \TeX Book} is to design a
%  macro for split level fractions. The solution presented is fairly
%  simple, using a \emph{virgule} (a slash) for separating the two
%  components. It looks okay because the text font and math font of
%  Computer Modern look almost identical.
%
%  The proper symbol to use instead of the virgule is a \emph{solidus}
%  which does not exist in Computer Modern. It is however available in
%  the European Computer Modern fonts, but I'll get back to that.
%
%  \subsection{The Present}
%
%
%  The most common way to produce split level fractions within \LaTeX\
%  is by means of the \pkg{nicefrac} package. Part of the reason it
%  has found widespread use is due to the strange design of the
%  built-in text fractions of the EC fonts, which look like this:
%  \textonehalf. The package is very simple to use but there are a few
%  issues:
% \begin{itemize}
%  \item It uses the virgule instead of the solidus.
%  \item Font size of numerator and denominator is bigger than in the
%    built-in symbol. Compare Palatino: \switch{ppl}{\nicefrac{1}{2}}
%    vs. \switch{ppl}{\textonehalf }.
%  \item It doesn't correct for fonts using text figures such as in the
%    \pkg{eco} package. Compare \switch{cmor}{\nicefrac{1}{2}} and
%    \switch{cmor}{\nicefrac{8}{9}}.
%  \item In math mode, it doesn't always pick up the correct math
%    alphabet.
% \end{itemize}
% In short: \pkg{nicefrac} doesn't attempt to be the answer to
% everything and so this is not a criticism of the package. It works
% quite well for Computer Modern which was pretty much what was widely
% available at the time it was developed. Users these days, however,
% have a choice of many fonts when they write their documents.
%
%
%
%
%
%  \subsection{The Future}
%
%  Fonts are wildly different; one macro that works fine for Computer
%  Modern obviously doesn't work well at all in Palatino. For one we
%  have to make the separator symbol configurable, and we need to
%  take care of several details as well: font scaling of the
%  numerator/denominator pair (ND), font selection of ND etc. If we
%  are to have a single package for this in the future\footnote{As
%  this is intended to be about the future, the \pkg{xfrac} package
%  requires the \eTeX{} extensions.} we have to define a totally
%  generic interface for the fraction commands and then adjust
%  parameters depending on the current font. What you see in this
%  prototype implementation of \pkg{xfrac} is just that.
%
%
%  \section{Advanced User Interface}
%
%
%  \subsection{Text mode}
%
%  The usual problem in text mode has a name: Computer Modern. The
%  solidi of all the Computer Modern fonts leave a lot to be desired,
%  although things are potentially looking better as the Latin Modern
%  fonts are becoming more stable and widespread. As long as the
%  default fonts are Computer Modern variants we must however work
%  around this. One idea that comes to mind is to see what happens
%  when you use a solidus from another font instead. Let's try with
%  Times:
%  \begin{quote}
%  \DeclareInstance{xfrac}{cmr2}{text}{
%    slash-symbol-font = ptm,
%  }
%  ``You take \sfrac[cmr2]{1}{2} cup of sugar, \ldots''
%  \end{quote}
%  That looks quite good actually, so it was probably very difficult
%  to obtain that result. Nope, it was extremely easy---if you happen
%  to know about \emph{instances}:
%  \begin{verbatim}
%  \DeclareInstance{xfrac}{cmr}{text}{
%    slash-symbol-font = ptm,
%  }
%  \end{verbatim}
%  So we define an instance with the name `cmr' from the template
%  `text' which in turn is of template type `xfrac'. You'll notice
%  the `cmr' is also the name of the font family for Computer Modern
%  Roman and the reasoning behind is that every font family should
%  have it's own settings, and if a document command is to work well
%  in that scheme, letting it use the name of the current font family
%  seems like a good idea. Thus the \cs{sfrac} command checks to see
%  whether an instance with same name as the current font family
%  exists and uses it if the test is true; otherwise the default
%  setting is used. Here we defined the instance to be used for the
%  font family `cmr' and just told it to use the Times font for
%  typesetting the slash symbol which turns out to be a solidus by
%  default.
%
%  The option \texttt{cm-recommended} which is loaded by default uses
%  the Times solidus for Computer Modern Roman and Computer Modern
%  Sans Serif and the Palatino solidus for Computer Modern Typewriter
%  Type. This looks quite good. Should you however not want this you
%  can use the option \texttt{cm-standard} which produces somewhat
%  acceptable results using Computer Modern exclusively.
%
%  So what about old style figures? If you use the \pkg{eco}
%  package you might define an instance similar to this (`cmor' is
%  the name of the roman font activated by \pkg{eco}):
%   \DeclareInstance{xfrac}{cmor}{text}{
%    slash-symbol-font = ptm,
%    numerator-font    = cmr,
%    denominator-font  = cmr,
%  }
%  \begin{verbatim}
%  \DeclareInstance{xfrac}{cmor}{text}{
%    slash-symbol-font = ptm,
%    numerator-font    = cmr,
%    denominator-font  = cmr,
%  }
%  \end{verbatim}
%  We also use regular Computer Modern Roman for typesetting ND, so
%  we end up with \switch{cmor}{\sfrac{1}{2}} and
%  \switch{cmor}{\sfrac{8}{9}} instead of
%  \switch{cmor}{\nicefrac{1}{2}} and \switch{cmor}{\nicefrac{8}{9}}.
%  Much better.
%
%  There are also situations where other tricks are useful. If you
%  don't have the inferior and superior figures available in a font,
%  or the font doesn't have a wider design for small font sizes, you
%  can cheat by manually scaling the ND-pair. I got nice results for
%  Adobe's Stempel Garamond (with small caps and old style figures)
%  with the following setup:
%  \begin{verbatim}
%  \DeclareInstance{xfrac}{pegj}{text}{
%    numerator-font   = pegx,
%    denominator-font = pegx,
%    scale-factor     = .9,
%    h-scale          = 1.1,
%  }
%  \end{verbatim}
%  We use the font family `pegx' (Stempel Garamond with real small
%  caps) for typesetting the ND-pair. Additionally the key
%  \key{scale-factor} specifies that the font size used for the
%  ND-pair should be $0.9$ of the height of the solidus, and the key
%  \key{h-scale} specifies that the ND-pair should be scaled an extra
%  10\% horizontally.
%
%  Should you be so fortunate the have a font with inferior and
%  superior figures like in the Monotype Janson example from Philipp
%  Lehman's excellent \emph{The Font Installation Guide}. In that
%  example Philipp defines the font families `mjn0' for the inferior
%  figures and `mjn1' for the superior. Thus to get the \cs{sfrac}
%  command to use them on the fly for the font family `mjnj' (Janson,
%  old style figures) we would say
%  \begin{verbatim}
%  \DeclareInstance{xfrac}{mjnj}{text}{
%    numerator-font      = mjn1,
%    denominator-font    = mjn0,
%    scaling             = false,
%    numerator-bot-sep   = 0pt,
%    denominator-bot-sep = 0pt,
%  }
%  \end{verbatim}
%  I think this example is a very clean way to do it. An alternative
%  approach could be to use the keys \key{numerator-format} and
%  \key{denominator-format} to process the arguments and let them
%  determine what to do.
%
%  As a side note Harald Harders was so kind to test it, and it
%  \emph{does} actually work---I hadn't tested it myself.
%
%  \subsection{Math Mode}
%
%  In math mode the choices are a lot fewer because first of all
%  \TeX{} comes with a built-in limitation of 16 math families.
%  Additionally we will not need a solidus for typesetting split
%  fractions in math, as tradition is to use a virgule instead. We
%  define the basic `mathdefault' instance to simply use the math
%  family in use when the instance is run. So if we're in normal math
%  like |$\sfrac{7}{9}$| we simply get family~$-1$. If we're inside a
%  \cs{mathbf} we're in family~$4$ (in the standard setup at least),
%  and so the fraction is typeset with the same math family. Simple,
%  isn't?
%
%  You can also declare instances for the math families, but I really
%  don't see why you would. If you do then name them according to the
%  scheme `mathfam\meta{N}', where \meta{N} is the family number, and
%  only do it if you \emph{really} know how to set up math fonts.
%  That is, if \cs{DeclareMathAlphabet} is unbeknownst to you, then
%  just don't go there.
%
%
%  Another example: If we want \cs{sfrac} to produce split fractions
%  without doing anything at all, we can choose the collection
%  `plainmath', which is defined as
%  \begin{verbatim}
%  \DeclareCollectionInstance{plainmath}{xfrac}{mathdefault}{math}{
%    denominator-bot-sep = 0pt,
%    numerator-bot-sep   = 0pt,
%    numerator-top-sep   = \c_max_dim,
%    scaling             = false,
%    slash-right-mkern   = 0mu,
%    slash-left-mkern    = 0mu,
%  }
%  \end{verbatim}
%  This creates an alternative version of the instance `mathdefault'
%  with settings as specified by the keys. In the default math setup
%  \key{numerator-top-sep} is set to 0~pt, and here we set
%  \key{numerator-bot-sep} to 0~pt as well, so in order to avoid
%  over-specification (and an error message) we must set
%  \key{numerator-top-sep} to \cs{c_max_dim}. We activate (obeying
%  normal scoping rules) it with:
%  \begin{verbatim}
%  \UseCollection{xfrac}{plainmath}
%  \end{verbatim}
%  Then |$\sfrac{8}{13}$| produces\begingroup
%    \UseCollection{xfrac}{plainmath}
%    $\sfrac{8}{13}$ and just typing |$8/13$| gives the same result:
%    $8/13$.
%  \endgroup
%
%
%  \section{The Template Interface}
%
%  \begin{TemplateInterfaceDescription}{xfrac}
%
%  \TemplateArgument{1}
%      {The numerator}
%
%  \TemplateArgument{2}
%       {The separator}
%
%   \TemplateArgument{3}
%      {The denominator}
%
%  \TemplateSemantics
%
%  Typesets arguments 1 and 3 separated by argument 2, which in text
%  mode by default is a \emph{solidus}. This is taken from
%  \pkg{textcomp} where it is denoted \cs{textfractionsolidus}.
%  This is the character used for the ready made split level
%  fractions such as \textonehalf---except in the (European) Computer
%  Modern fonts. In math mode a \emph{virgule} is used instead as
%  this is more appropriate and it is always available in the math
%  fonts. The solidus is a text symbol only.
%
%  \end{TemplateInterfaceDescription}
%
%  \begin{TemplateDescription}{xfrac}{text}
%
%  \TemplateKey{numerator-font}{tokenlist}
%    {Font family specification to use for the numerator.}
%    {\cs{f@family}}
%
%  \TemplateKey{numerator-format}{function 1 arg}
%    {Action to be taken on the numerator.}
%    {Process argument unchanged}
%
%  \TemplateKey{slash-symbol}{tokenlist}
%    {The separator symbol. If not specified the default value will be
%    used instead.}
%    {Solidus (\cs{textfractionsolidus})}
%
%  \TemplateKey{slash-symbol-font}{tokenlist}
%    {Font family specification to use for the separator symbol.}
%    {\cs{f@family}}
%
%  \TemplateKey{slash-symbol-format}{function 1 arg}
%    {Action to be taken on the separator symbol.}
%    {Process argument unchanged}
%
%  \TemplateKey{denominator-font}{tokenlist}
%    {Font family specification to use for the denominator.}
%    {\cs{f@family}}
%
%  \TemplateKey{denominator-format}{function 1 arg}
%    {Action to be taken on the denominator.}
%    {Process argument unchanged}
%
%  \TemplateKey{h-scale}{tokenlist}
%    {Factor by which the numerator and denominator should be
%    horizontally scaled. It should only be used if the real superior
%    and inferior fonts are not available. For instance Stempel
%    Garamond looks excellent if scaled 10\% extra horizontally, i.e.,
%    by a factor of 1.1.}
%    {1}
%
%  \TemplateKey{v-scale}{tokenlist}
%    {Same as \key{h-scale} only vertically. Probably not of much use
%    but added for completetion.}
%    {1}
%
%  \TemplateKey{scale-factor}{tokenlist}
%    {Fraction of the size of \key{slash-symbol}. Used for setting the
%    font size of numerator and denominator. Usually a value of app.\
%    \sfrac{5}{6} produces fine results. It should only be used if the
%    real superior and inferior fonts are not available. As an example
%    Stempel Garamond looks better if the factor is 0.9.}
%    {0.83333}
%
%  \TemplateKey{scale-relative}{choice}
%    {If set to `true' the font size of the numerator and denominator
%    is scaled with respect to the height of the \key{slash-symbol}. If
%    set to `false' the font is scaled with respect to the total height
%    of the \key{slash-symbol}.}
%    {true}
%
%  \TemplateKey{scaling}{choice}
%    {If set to `true' the fonts are allowed to scale. If set to
%    `false' they are not. See the `Janson' example for an application.}
%    {true}
%
%  \TemplateKey{numerator-top-sep}{length}
%    {Dimension specifying the space between the top of the
%    \key{slash-symbol} and the top of the numerator. If not specified,
%    the depth of the solidus will be used, because this value will
%    make the fraction look even.}
%    {Unspecified}
%
%  \TemplateKey{numerator-bot-sep}{length}
%    {Dimension specifying the lift of the numerator from the
%    baseline.}
%    {Unspecified}
%
%  \TemplateKey{denominator-bot-sep}{length}
%    {Dimension specifying the lift of the denominator from the
%    baseline.}
%    {Unspecified}
%
%  \TemplateKey{slash-right-kern}{length}
%    {Dimension specifying the kerning between the \key{slash-symbol}
%    and the numerator.}
%    {\texttt{0pt}}
%
%  \TemplateKey{slash-left-kern}{length}
%    {Dimension specifying the kerning between the \key{slash-symbol}
%    and the denominator.}
%    {\texttt{0pt}}
%
%  \TemplateKey{math-mode}{choice}
%    {Are we in math mode or not?}
%    {false}
%
%  \TemplateKey{phantom}{tokenlist}
%    {A character that suits the common cases. As we would mostly want
%    to use numbers in text mode we choose a `tall' number, while in
%    math it is somewhat different.}
%    {8}
%
%  \TemplateSemantics
%
%  This template is also the foundation for the `math' template. The
%  keys \key{slash-right-mkern} and \key{slash-left-mkern} can only
%  be used in math mode and are not shown here.
%
%  \end{TemplateDescription}
%
%
%  \begin{TemplateDescription}{xfrac}{math}
%
%  \TemplateKey{numerator-font}{tokenlist}
%    {Font family specification to use for the numerator.}
%    {\cs{number}\cs{fam}}
%
%  \TemplateKey{slash-symbol}{tokenlist}
%    {The separator symbol. If not specified the default value will be
%    used instead.}
%    {Virgule ($/$)}
%
%  \TemplateKey{slash-symbol-font}{tokenlist}
%    {Font family specification to use for the separator symbol.}
%    {\cs{number}\cs{fam}}
%
%  \TemplateKey{denominator-font}{tokenlist}
%    {Font family specification to use for the denominator.}
%    {\cs{number}\cs{fam}}
%
%  \TemplateKey{scale-factor}{tokenlist}
%    {Fraction of the size of \key{slash-symbol}. In math mode we
%    cannot rely on the fonts to be able to scale, but giving a default
%    scale of 0.7 fits into the regular size changing scheme---the
%    default scheme has values $(D,T,S,SS)=(1,1,0.7,0.5)$ whereas we
%    with a default \key{scale-factor} of 0.7 get $(1,1,0.7,0.49)$.
%    That's close enough.}
%    {0.7}
%
%  \TemplateKey{scale-relative}{choice}
%    {If set to `true' the font size of the numerator and denominator
%    is scaled with respect to the height of the \key{slash-symbol}. If
%    set to `false' the font is scaled with respect to the total height
%    of the \key{slash-symbol}.}
%    {false}
%
%  \TemplateKey{scaling}{choice}
%    {If set to `true' the fonts are allowed to scale. If set to
%    `false' they are not. See the `plainmath' example for an application.}
%    {true}
%
%  \TemplateKey{numerator-top-sep}{length}
%    {Dimension specifying the space between the top of the
%    \key{slash-symbol} and the top of the numerator. If not specified,
%    the depth of the virgule will be used, because this value will
%    make the fraction look even.}
%    {\texttt{0pt}}
%
%  \TemplateKey{denominator-bot-sep}{length}
%    {Dimension specifying the lift of the denominator from the
%    baseline.}
%    {\texttt{0pt}}
%
%  \TemplateKey{slash-right-mkern}{tokenlist}
%    {Same as \key{slash-right-kern} but for math mode only and should
%    be specified in \texttt{mu} units. This is because \pkg{calc}
%    can't use mu-expressions.}
%    {\texttt{-2mu}}
%
%  \TemplateKey{slash-left-mkern}{tokenlist}
%    {Same as \key{slash-left-kern} but for math mode only and should
%    be specified in \texttt{mu} units. This is because \pkg{calc}
%    can't use mu-expressions.}
%    {\texttt{-1mu}}
%
%  \TemplateKey{math-mode}{choice}
%    {Are we in math mode or not?}
%    {true}
%
%  \TemplateKey{phantom}{tokenlist}
%    {A character that suits the common cases. In math we have a high
%    risk of using a parenthesis, so we choose that. Text mode is
%    another story.}
%    {(}
%
%  \TemplateSemantics
%
%  This template is a restricted version of the `text' template. Only
%  the keys that are different from the `text' template are shown
%  here. Also bear in mind that the attributes \key{slash-left-kern}
%  and \key{slash-right-kern} have no meaning in this template.
%
%  \end{TemplateDescription}
% 
%  \StopEventually{}
%
%\section{Implementation}
%
%    \begin{macrocode}
%<*package>
%    \end{macrocode}
%       
% The usual lead-off: provides an experimental package!
%    \begin{macrocode}
\RequirePackage{expl3}[2009/08/05]
\ProvidesExplPackage{xfrac}{2010/02/02}{0.3}{Text fractions}
%    \end{macrocode}
%
% Some support is needed: a bit wider than the normal \pkg{xpackage}
% stuff, but not by much.
%    \begin{macrocode}
\RequirePackage { amstext , graphicx , l3keys2e , textcomp , xtemplate }
%    \end{macrocode}   
%
%\begin{macro}{\l_xfrac_cm_std_bool}
% There is one option to support.
%    \begin{macrocode}
\keys_define:nn { xfrac } {
  cm-recommended .choice:,
  cm-recommended /
    false        .code:n     =
      { \bool_set_true:N \l_xfrac_cm_std_bool },
  cm-recommended /
    true         .code:n     =
      { \bool_set_false:N \l_xfrac_cm_std_bool },
  cm-recommended .default:n  = { true },    
  cm-standard    .bool_set:N = \l_xfrac_cm_std_bool
}
\ProcessKeysOptions { xfrac }
%    \end{macrocode}
%\end{macro}
%
%\begin{macro}{\l_xfrac_slash_box}
%\begin{macro}{\l_xfrac_tmp_box}
% In keeping with the \LaTeX3 philosophy, rather than use generic 
% scratch boxes and get confused, \pkg{xfrac} reserves its own named
% working space.
%    \begin{macrocode}
\box_new:N \l_xfrac_slash_box
\box_new:N \l_xfrac_tmp_box
%    \end{macrocode}
%\end{macro}
%\end{macro}
%
%\begin{macro}{\xfrac_tmp:w}
% Used for the raised boxes: weird as it does not take an argument
% but the \cs{raisebox} does.
%    \begin{macrocode}
\cs_new:Npn \xfrac_tmp:w { }
%    \end{macrocode}
%\end{macro}
%
%\subsection{Initialisation of variables}
%
% Variables used in templates have to be set up: there is not
% much to say about these, other than that they must exist.
%    
%\begin{macro}{\l_xfrac_denominator_bot_sep_dim}
%\begin{macro}{\l_xfrac_numerator_bot_sep_dim}
%\begin{macro}{\l_xfrac_numerator_top_sep_dim}
%\begin{macro}{\l_xfrac_slash_left_sep_dim}
%\begin{macro}{\l_xfrac_slash_right_sep_dim}
% Fixed lengths.
%    \begin{macrocode}
\dim_new:N \l_xfrac_denominator_bot_sep_dim
\dim_new:N \l_xfrac_numerator_bot_sep_dim
\dim_new:N \l_xfrac_numerator_top_sep_dim
\dim_new:N \l_xfrac_slash_left_sep_dim
\dim_new:N \l_xfrac_slash_right_sep_dim
%    \end{macrocode}
%\end{macro}
%\end{macro}
%\end{macro}
%\end{macro}
%\end{macro}
%
%\begin{macro}{\l_xfrac_denominator_font_tl}
%\begin{macro}{\l_xfrac_hscale_tl}
%\begin{macro}{\l_xfrac_numerator_font_tl}
%\begin{macro}{\l_xfrac_phantom_tl }
%\begin{macro}{\l_xfrac_scale_factor_tl}
%\begin{macro}{\l_xfrac_slash_left_msep_tl}
%\begin{macro}{\l_xfrac_slash_right_msep_tl}
%\begin{macro}{\l_xfrac_slash_symbol_tl}
%\begin{macro}{\l_xfrac_slash_symbol_font_tl}
%\begin{macro}{\l_xfrac_vscale_tl}
% Token lists, which include floating-point numbers and math(s)
% skips.
%    \begin{macrocode}
\tl_new:N \l_xfrac_denominator_font_tl
\tl_new:N \l_xfrac_hscale_tl
\tl_new:N \l_xfrac_numerator_font_tl
\tl_new:N \l_xfrac_phantom_tl 
\tl_new:N \l_xfrac_scale_factor_tl
\tl_new:N \l_xfrac_slash_left_msep_tl
\tl_new:N \l_xfrac_slash_right_msep_tl
\tl_new:N \l_xfrac_slash_symbol_tl
\tl_new:N \l_xfrac_slash_symbol_font_tl
\tl_new:N \l_xfrac_vscale_tl
%    \end{macrocode}
%\end{macro}
%\end{macro}
%\end{macro}
%\end{macro}
%\end{macro}
%\end{macro}
%\end{macro}
%\end{macro}
%\end{macro}
%\end{macro}
%
%\begin{macro}{\xfrac_fontscale:}
%\begin{macro}{\xfrac_math:n}
%\begin{macro}{\xfrac_denominator_font_change:}
%\begin{macro}{\xfrac_denominator_format:n}
%\begin{macro}{\xfrac_numerator_font_change:}
%\begin{macro}{\xfrac_numerator_format:n}
%\begin{macro}{\xfrac_relscale:}
%\begin{macro}{\xfrac_slash_symbol_font_change:}
%\begin{macro}{\xfrac_slash_symbol_format:n}
%\begin{macro}{\xfrac_text_or_math:n}
% Functions, either things which are calculated `on the fly'
% (no argument required) or are functions taking one argument in the
% code.
%    \begin{macrocode}
\cs_new_nopar:Npn \xfrac_fontscale:
\cs_new:Npn \xfrac_math:n #1 { }
\cs_new_nopar:Npn \xfrac_denominator_font_change: { }
\cs_new:Npn \xfrac_denominator_format:n #1 { }
\cs_new_nopar:Npn \xfrac_numerator_font_change: { }
\cs_new:Npn \xfrac_numerator_format:n #1 { }
\cs_new_nopar:Npn \xfrac_relscale: { }
\cs_new_nopar:Npn \xfrac_slash_symbol_font_change: { }
\cs_new:Npn \xfrac_slash_symbol_format:n #1 { }
\cs_new:Npn \xfrac_text_or_math:n #1 { }
%    \end{macrocode}
%\end{macro}
%\end{macro}
%\end{macro}
%\end{macro}
%\end{macro}
%\end{macro}
%\end{macro}
%\end{macro}
%\end{macro}
%\end{macro}
%
%\subsection{The template}
%
% There is only one object type in \pkg{xfrac}, rather unimaginatively
% named \texttt{xfrac}.
%    \begin{macrocode}
\DeclareObjectType { xfrac } { 3 }
%    \end{macrocode}
%
% A single template interface is used for both text and math(s), which
% does make a few things a little complex later.
%    \begin{macrocode}
\DeclareTemplateInterface { xfrac } { text } { 3 } {
  denominator-bot-sep : length     = \c_max_dim           ,
  denominator-font    : tokenlist  = \f@family            ,
  denominator-format  : function 1 = #1                   ,
  h-scale             : tokenlist  = 1                    ,
  math-mode           : choice { false , true }
                                   = false                ,
  numerator-font      : tokenlist  = \f@family            ,
  numerator-format    : function 1 = #1                   ,
  numerator-bot-sep   : length     = \c_max_dim           ,
  numerator-top-sep   : length     = \c_max_dim           ,
  phantom             : tokenlist  = 8                    ,
  scale-factor        : tokenlist  = 0.83333              ,
  scale-relative      : choice { false , true }
                                   = true                 ,
  scaling             : choice { false , true }
                                   = true                 ,
  slash-left-kern     : length     = 0 pt                 ,
  slash-left-mkern    : tokenlist  = -2 mu                ,
  slash-right-kern    : length     = 0 pt                 ,
  slash-right-mkern   : tokenlist  = -1 mu                ,
  slash-symbol        : tokenlist  = \textfractionsolidus ,
  slash-symbol-font   : tokenlist  = \f@family            ,
  slash-symbol-format : function 1 = #1                   ,
  v-scale             : tokenlist  = 1                    ,
}
%    \end{macrocode}
%   
% Most of the variable binding is quite simple: of course, the choices
% are a little more complicated. That is particularly true where 
% these have to set up `on the fly' functions.   
%    \begin{macrocode}
\DeclareTemplateCode { xfrac } { text } { 3 } 
  {
    denominator-bot-sep = \l_xfrac_denominator_bot_sep_dim ,
    denominator-font    = \l_xfrac_denominator_font_tl     ,
    denominator-format  = \xfrac_denominator_format:n      ,
    h-scale             = \l_xfrac_hscale_tl               ,
    math-mode           =
      {
        false = \cs_set_eq:NN \xfrac_math:n \use:n,
        true  = \cs_set_eq:NN \xfrac_math:n \ensuremath
      },
    numerator-font      = \l_xfrac_numerator_font_tl       ,
    numerator-format    = \xfrac_numerator_format:n        ,
    numerator-bot-sep   = \l_xfrac_numerator_bot_sep_dim   ,
    numerator-top-sep   = \l_xfrac_numerator_top_sep_dim   ,
    phantom             = \l_xfrac_phantom_tl              ,
    scale-factor        = \l_xfrac_scale_factor_tl         ,
    scale-relative      =
      {
        false = 
          \cs_set_nopar:Npn \xfrac_relscale:
            {
              \dim_eval:n 
                { 
                    \box_ht:N \l_xfrac_tmp_box 
                  + \box_dp:N \l_xfrac_tmp_box
                } 
            },
        true  = 
          \cs_set_nopar:Npn \xfrac_relscale: 
            { \box_ht:N \l_xfrac_slash_box }
      },
    scaling             =
      {
        false = \cs_set_eq:NN \xfrac_fontscale: \prg_do_nothing:, 
        true  =  
          \cs_set_nopar:Npn \xfrac_fontscale:
            {
              \fontsize { \l_xfrac_scale_factor_tl \xfrac_relscale: }
                { \c_zero_dim } 
              \selectfont
            }
      },
    slash-left-kern     = \l_xfrac_slash_left_sep_dim      ,
    slash-left-mkern    = \l_xfrac_slash_left_msep_tl      ,
    slash-right-kern    = \l_xfrac_slash_right_sep_dim     ,
    slash-right-mkern   = \l_xfrac_slash_right_msep_tl     ,
    slash-symbol        = \l_xfrac_slash_symbol_tl         ,
    slash-symbol-font   = \l_xfrac_slash_symbol_font_tl    ,
    slash-symbol-format = \xfrac_slash_symbol_format:n     ,
    v-scale             = \l_xfrac_vscale_tl        
  }
%    \end{macrocode}
% The implementation part starts with applying all of the settings
% from above. The first part of the set up is then to determine
% whether the surroundings are text or math(s), and react accordingly.
%    \begin{macrocode}
  {
    \AssignTemplateKeys
    \mode_if_math:TF
      {
        \cs_set_eq:NN \xfrac_text_or_math:n \text
        \cs_set_nopar:Npx \xfrac_denominator_font_change:
          { \tex_fam:D \l_xfrac_denominator_font_tl }
        \cs_set_nopar:Npx \xfrac_numerator_font_change:
          { \tex_fam:D \l_xfrac_numerator_font_tl }
        \cs_set_nopar:Npx \xfrac_slash_symbol_font_change:
          { \tex_fam:D \l_xfrac_slash_symbol_font_tl }
      }
      {
        \cs_set_eq:NN \xfrac_text_or_math:n \mbox
        \cs_set_nopar:Npn \xfrac_denominator_font_change:
          { 
            \fontfamily { \l_xfrac_denominator_font_tl }
            \selectfont
          }
        \cs_set_nopar:Npn \xfrac_numerator_font_change:
          { 
            \fontfamily { \l_xfrac_numerator_font_tl }
            \selectfont
          }
        \cs_set_nopar:Npn \xfrac_slash_symbol_font_change:
          { 
            \fontfamily { \l_xfrac_slash_symbol_font_tl }
            \selectfont
          }
      }
%    \end{macrocode}
%\changes{v0.11a}{2004/08/24}{Added \cs{m@th}}
% Everything is now either inside \cs{text} or an \cs{mbox}, depending
% upon the surroundings. First, there are some boxes to set up.
%    \begin{macrocode}
    \xfrac_text_or_math:n 
      {
        \m@th
        \hbox_set:Nn \l_xfrac_tmp_box 
          { \xfrac_math:n { \vphantom { ( ) } } }
        \hbox_set:Nn \l_xfrac_slash_box 
          {
            \xfrac_math:n 
              { 
                \xfrac_slash_symbol_format:n 
                  {
                    \xfrac_math:n 
                      {
                        \xfrac_slash_symbol_font_change:
                        \IfNoValueTF {#2} 
                          { \l_xfrac_slash_symbol_tl } {#2}
                      }
                  }
              }
          }
%    \end{macrocode}
% Check on the numerator separator dimensions. The code starts with the
% assumption that neither has been given, as this can then be used to
% set up a default, which is also used when both values are set 
% erroneously.
%    \begin{macrocode}
        \cs_set_nopar:Npn \xfrac_tmp:w
          {
            \raisebox 
              {
                  \box_ht:N \l_xfrac_slash_box
                - \box_dp:N \l_xfrac_slash_box
                - \height
              }
          }
        \dim_compare:nNnTF 
          { \l_xfrac_numerator_top_sep_dim } = { \c_max_dim }
          {
            \dim_compare:nNnF
              { \l_xfrac_numerator_bot_sep_dim } = { \c_max_dim } 
              {
                \cs_set_nopar:Npn \xfrac_tmp:w
                  { 
                    \raisebox 
                      { \dim_use:N \l_xfrac_numerator_bot_sep_dim } 
                  }
              }
          }
          {
            \dim_compare:nNnTF
              { \l_xfrac_numerator_bot_sep_dim } = { \c_max_dim } 
                { 
                  \cs_set_nopar:Npn \xfrac_tmp:w
                    { 
                      \raisebox 
                        { 
                            \box_ht:N \l_xfrac_slash_box
                          - \dim_use:N \l_xfrac_numerator_top_sep_dim 
                          - \height
                        }
                    }
                }
                { 
                  \msg_error:nn { xfrac } 
                    { over-specified-numerator-sep }
                }
          }
%    \end{macrocode}
%  Typeset the numerator.
%    \begin{macrocode}
        \xfrac_tmp:w
          {
            \xfrac_fontscale:
            \xfrac_numerator_format:n
              {
                \scalebox { \l_xfrac_hscale_tl } [ \l_xfrac_vscale_tl ]
                  { 
                    \xfrac_math:n 
                      { 
                        \xfrac_numerator_font_change:
                        {
                          \vphantom { \l_xfrac_phantom_tl }
                          #1
                        }
                      } 
                  }
              }
          }
        \xfrac_math:n
          { % THIS IS JUST WRONG! 
            \mode_if_math:TF
              { \tex_mskip:D \l_xfrac_slash_right_msep_tl }
              { \tex_hskip:D \l_xfrac_slash_right_sep_dim }
          }  
%    \end{macrocode}
%  Typeset the separator.
%    \begin{macrocode}
        \box_use:N \l_xfrac_slash_box
        \xfrac_math:n 
          {
            \mode_if_math:TF
              { \tex_mskip:D \l_xfrac_slash_left_msep_tl }
              { \tex_hskip:D \l_xfrac_slash_left_sep_dim }
          }
%    \end{macrocode}
%  Typeset the denominator.
%    \begin{macrocode}
        \dim_compare:nNnTF 
          { \l_xfrac_denominator_bot_sep_dim } = { \c_max_dim }
          {
            \cs_set_nopar:Npn \xfrac_tmp:w
              { \raisebox { - \box_dp:N \l_xfrac_slash_box } } 
          }
          {
            \cs_set_nopar:Npn \xfrac_tmp:w
              { 
                \raisebox 
                  { \dim_use:N \l_xfrac_denominator_bot_sep_dim }
              }
          }  
        \xfrac_tmp:w
          {
            \xfrac_fontscale:
            \xfrac_denominator_format:n
              {
                \scalebox { \l_xfrac_hscale_tl } [ \l_xfrac_vscale_tl ]
                  { 
                    \xfrac_math:n 
                      { 
                        \xfrac_denominator_font_change:
                        {
                          \vphantom { \l_xfrac_phantom_tl }
                          #3
                        }
                      } 
                  }
              }
          }
      } 
  }
%    \end{macrocode}
%    
% Since math(s) and text mode are wildly different entities we define a
% separate template for each. You already saw the `text' 
% template, and here is the `math' template.
%    \begin{macrocode}
\DeclareRestrictedTemplate { xfrac } { text } { math } {
  numerator-font      = \number \fam ,
  slash-symbol        = /            ,
  slash-symbol-font   = \number \fam ,
  denominator-font    = \number \fam ,
  scale-factor        = 0.7          ,
  scale-relative      = false        ,
  scaling             = true         ,
  numerator-top-sep   = 0 pt         ,
  denominator-bot-sep = 0 pt         ,
  math-mode           = true         ,
  phantom             = (
}
%    \end{macrocode}
%    
%\subsection{The standard instances}
%
% For the default instances we just use the relevant templates with
% the default settings.
% 
%  The default `text' instance.
%    \begin{macrocode}
\DeclareInstance { xfrac } { default } { text } { }
%    \end{macrocode}
%    
%  The default `math(s)' instance.
%    \begin{macrocode}
\DeclareInstance { xfrac } { mathdefault } { math } { }
%    \end{macrocode}
%    \begin{macrocode}
\DeclareCollectionInstance { plainmath } { xfrac } { mathdefault } 
  { math }{
  denominator-bot-sep = 0 pt       ,
  numerator-bot-sep   = 0 pt       ,
  numerator-top-sep   = \c_max_dim ,
  scale-factor        = 1          ,
  scale-relative      = false      ,
  scaling             = true       ,
  slash-right-mkern   = 0mu        ,
  slash-left-mkern    = 0mu
}
%    \end{macrocode}
%    
% Default Computer Modern setup. Far from optimal, but better than
% nothing.
%    \begin{macrocode}
\DeclareInstance { xfrac } { cmr } { text } {
  denominator-bot-sep = 0 pt    ,
  numerator-top-sep   = 0.2 ex  ,
  slash-left-kern     = -0.1 em ,
  slash-right-kern    = -0.1 em 
}
\DeclareInstance { xfrac } { cmss } { text } {
  denominator-bot-sep = 0 pt    ,
  numerator-top-sep   = 0.2 ex  ,
  slash-left-kern     = -0.1 em ,
  slash-right-kern    = -0.1 em 
}
\DeclareInstance { xfrac } { cmtt } { text } {
  denominator-bot-sep = 0 pt    ,
  numerator-top-sep   = 0.2 ex  ,
  slash-left-kern     = -0.1 em ,
  slash-right-kern    = -0.1 em 
}
%    \end{macrocode}
%    
% We can do better for the Computer Modern fonts. For cmr and cmss 
% we choose Times, and for cmtt use Palatino.
%    \begin{macrocode}
\bool_if:NF \l_xfrac_cm_std_bool
  {
    \DeclareInstance { xfrac } { cmr } { text }
      { slash-symbol-font = ptm }
    \DeclareInstance { xfrac } { cmss } { text }
      { slash-symbol-font = ptm }
    \DeclareInstance { xfrac } { cmtt } { text }
      { slash-symbol-font = ppl }
  }
%    \end{macrocode}
%    
% Things works slightly better with Latin Modern.
%    \begin{macrocode}
\DeclareInstance { xfrac } { lmr } { text } {
  denominator-bot-sep = 0 pt     ,
  numerator-top-sep   = 0.1 ex   ,
  slash-left-kern     = -0.15 em ,
  slash-right-kern    = -0.15 em 
}
\DeclareInstance { xfrac } { lmss } { text } {
  denominator-bot-sep = 0 pt     ,
  numerator-top-sep   = 0 pt     ,
  slash-left-kern     = -0.15 em ,
  slash-right-kern    = -0.15 em 
}
\DeclareInstance { xfrac } { lmtt } { text } {
  denominator-bot-sep = 0 pt     ,
  numerator-top-sep   = 0 pt     ,
  slash-left-kern     = -0.15 em ,
  slash-right-kern    = -0.15 em 
}
%    \end{macrocode}
%
%\subsection{Messages}
%
% Just the one.    
%    \begin{macrocode}
\msg_new:nnnn { xfrac } { over-specified-numerator-sep }
  {You have specified both numerator-top-sep and numerator-bot-sep}
  {I will pretend that you didn't specify either of them}
%    \end{macrocode}    
%    
%\subsection{The user command}
%
% Currently there is just a single user command. \cs{sfrac} takes
% two mandatory arguments: numerator and denominator. It can take an
% optional argument between the mandatory specifying the separator
% like this:
%\begin{verbatim}
%  \sfrac{7}[/]{12}
%\end{verbatim}
% It also has an optional argument that comes before the first
% mandatory argument. If used it will use that instance instead of
% the auto-detected one, so a user who has defined the instance
% `cmr2' may use
%\begin{verbatim}
%  \sfrac[cmr2]{7}{12}
%\end{verbatim}
% and get the settings from `cmr2' instead of the settings of 
% the current font family.
%    \begin{macrocode}
\NewDocumentCommand \sfrac { o m o m } {
  \mode_if_math:TF 
    {
      \IfInstanceExistTF { xfrac } { mathfam \number \fam }
        { \UseInstance { xfrac } { mathfam \number \fam } }
        { \UseInstance { xfrac } { mathdefault } }
      {#2} {#3} {#4}  
    }
    {
      \IfInstanceExistTF { xfrac } {#1}
        { \UseInstance { xfrac } {#1} }
        { 
          \IfInstanceExistTF { xfrac } { \f@family }
            { \UseInstance { xfrac } { \f@family } }
            { \UseInstance { xfrac } { default } }
        }
      {#2} {#3} {#4}   
    }
}
%    \end{macrocode}